\usepackage[acronym,shortcuts]{glossaries}

\makeglossaries

\glsaddkey
{longsp}% key
{}% default value
{\glsentrylongsp}% new command analogous to \glsentrylong
{\Glsentrylongsp}% new command analogous to \Glsentrylong
{\acrlongsp}% new command analogous to \acrlong
{\Acrlongsp}% new command analogous to \Acrlong
{\ACRlongsp}% new command analogous to \ACRlong

% Add new key for plural long Spanish form:
\glsaddkey
{longplsp}% key
{}% default value
{\glsentrylongplsp}% new command analogous to \glsentrylongpl
{\Glsentrylongplsp}% new command analogous to \Glsentrylongpl
{\acrlongplsp}% new command analogous to \acrlongpl
{\Acrlongplsp}% new command analogous to \Acrlongpl
{\ACRlongplsp}% new command analogous to \ACRlongpl

% Provide conditional to test if longsp/longplsp has been set
\newcommand*{\glsifhaslongsp}[3]{%
	\ifcsempty{glo@#1@longsp}{#3}{#2}%
}
\newcommand*{\glsifhaslongplsp}[3]{%
	\ifcsempty{glo@#1@longplsp}{#3}{#2}%
}

% Define new acronym style:

\newacronymstyle{spanish}
{% base the display style on 'long-short'
	\GlsUseAcrEntryDispStyle{long-short}%
}%
{% base the definitions on 'long-short'
	\GlsUseAcrStyleDefs{long-short}%  
	% Make some custom modifications for the first use display.
	% Singular, no case change:
	\renewcommand*{\genacrfullformat}[2]{%
		\glsifhaslongsp{##1}%
		{% has Spanish version:
			\glsentrylongsp{##1}##2\space
			(\firstacronymfont{\glsentryshort{##1}}, \glsentrylong{##1})%
		}%
		{%
			\glsentrylong{##1}##2\space
			(\firstacronymfont{\glsentryshort{##1}})%
		}%
	}%
	% Singular, first letter upper case:
	\renewcommand*{\Genacrfullformat}[2]{%
		\glsifhaslongsp{##1}%
		{% has Spanish version:
			\Glsentrylongsp{##1}##2\space
			(\firstacronymfont{\glsentryshort{##1}}, \glsentrylong{##1})%
		}%
		{%
			\Glsentrylong{##1}##2\space
			(\firstacronymfont{\glsentryshort{##1}})%
		}%
	}%
	% Plural, no case change:
	\renewcommand*{\genplacrfullformat}[2]{%
		\glsifhaslongplsp{##1}%
		{% has Spanish version:
			\glsentrylongplsp{##1}##2\space
			(\firstacronymfont{\glsentryshortpl{##1}}, \glsentrylongpl{##1})%
		}%
		{%
			\glsentrylongpl{##1}##2\space
			(\firstacronymfont{\glsentryshortpl{##1}})%
		}%
	}%
	% Plural, first letter upper case:
	\renewcommand*{\Genplacrfullformat}[2]{%
		\glsifhaslongplsp{##1}%
		{% has Spanish version:
			\Glsentrylongplsp{##1}##2\space
			(\firstacronymfont{\glsentryshortpl{##1}}, \glsentrylongpl{##1})%
		}%
		{%
			\Glsentrylongpl{##1}##2\space
			(\firstacronymfont{\glsentryshortpl{##1}})%
		}%
	}%
}

% switch to the new style:
\setacronymstyle{spanish}

% Define a new glossary style that checks for the existence of
% the longsp field.
\newglossarystyle{listsp}{%
	\setglossarystyle{list}% base style on the list style
	\renewcommand*{\glossentry}[2]{%
		\item[\glsentryitem{##1}%
		\glstarget{##1}{\glossentryname{##1}}]
		\glossentrydesc{##1}%
		\glsifhaslongsp{##1}{\space(\glsentrylongsp{##1})}{}%
		\glspostdescription\space ##2}%
}


% Definiciones
\newglossaryentry{car-following}{
	name=car-following,
	description={Modelo de control de conductor que basa su comportamiento en el vehículo del mismo carril que le precede.}
}
\newglossaryentry{sumo}{
	name=SUMO,
	description={(\textit{Simulation of Urban MObility}). Entorno de micro y mesosimulaón multiagente desarrollado por el instituto de sistemas de transporte del DLR (Centro Aeroespacial Alemán). Url: \url{http://www.dlr.de}.}
}
\newglossaryentry{matsim}{
	name=MatSIM,
	description={(\textit{Multi-Agent Transport Simulation}). Software de microsimulación multiagente desarrollado en la ETH Zürich. Url: \url{http://matsim.org}.}
}
\newglossaryentry{movsim}{
	name=MovSim,
	description={(\textit{Multi-model Open-source Vehicular-traffic Simulator}). Software de microsimulación que implementa tanto modelo multiagente y como modelo basado en \glsentryshortpl{ca}. Url: \url{http://www.movsim.org/}.}
}
\newglossaryentry{mitsim}{
	name=MovSim,
	description={(\textit{MIcroscopic Traffic SIMulator}). Software de microsimulación desarrollado por el laboratorio de sistemas inteligentes de transporte del MIT. Url: \url{https://its.mit.edu/software/mitsimlab}.}
}
\newglossaryentry{aorta}{
	name=AORTA,
	description={(\textit{Approximately Orchestrated Routing and Transportation Analyzer}). Entorno de microsimulación multiagente de tráfico desarrollado en el departamento de Ciencias de la Computación de la Universidad de Austin (Texas). Url: \url{http://www.aorta-traffic.org/}.}
}
\newglossaryentry{gpl}{
	name=GPL,
	description={(\textit{General Public License}). Licencia de software que garantiza las libertades del software. Asegura que cualquier versión, extensión o software derivado de éste permanecerá siendo software libre. Su última versión es la 3.0. Url: \url{https://www.gnu.org/licenses/gpl-3.0.html}.}
}
\newglossaryentry{traci}{
	name=TraCI,
	description={(\textit{Traffic Control Interface}). Término que sirve tanto para denominar al protocolo de comunicación ofrecido por \gls{sumo} en modo servidor para la interacción remota con la simulación como para denominar a la librería desarrollada para abstraer dicho protocolo cuando se trabaja desde \gls{python}. Url: \url{http://www.sumo.dlr.de/wiki/TraCI}.}
}
\newglossaryentry{traas}{
	name=TraaS,
	description={(\textit{TraCI as a Service}). \glsentryshortpl{api} basado en \glsentryshortpl{soap} para interactuar con el simulador \gls{sumo} cuando está funcionando en modo servidor. Url: \url{http://traas.sourceforge.net/cms/}.}
}
\newglossaryentry{torcs}{
	name=TORCS,
	description={(\textit{The Open Racing Car Simulator}). Software de simulación creado en un principio como videojuego de carreras y que ha evolucionado hacia plataforma de simulación de técnicas de \glsentrylongsp{ai} aplicadas a la conducción. Url: \url{http://torcs.sourceforge.net/}.}
}
\newglossaryentry{python}{
	name=Python,
	description={Lenguaje de programación \glsentryshort{oss}. Url: \url{https://www.python.org/}.}
}

% Acrónimos
\newacronym{v2v}{V2V}{Vehicle-to-Vehicle}
\newacronym{api}{API}{Applcation Programming Interface}
\newacronym{v2i}{V2I}{Vehicle-to-Infraestructure}
\newacronym{dvu}{DVU}{Driver-Vehicle Unit}
\newacronym{acl}{ACL}{Agent Communications Language}
\newacronym{kqml}{KQML}{Knowledge Query and Manipulation Language}
\newacronym{sc}{SC}{Soft Computing}
\newacronym{soap}{SOAP}{Simple Object Access Protocol}
\newacronym{hc}{HC}{Hard Computing}
% Acrónimos con traducción
\newacronym[longsp=Estudio Naturalista de Conducción,longplsp=Estudios Naturalistas de Conducción,longplural=Naturalistic Driving Studies]{nds}{NDS}{Naturalistic Driving Study}
\newacronym[longsp=Reconocimento del Entorno Intravehicular,longplsp=Reconocimientos del Entorno Intravehiculares,longplural=Intra-vehicular Context Awarenesses]{ivca}{IvCA}{Intra-vehicular Context Awareness}
\newacronym[longsp=Sistema Avanzado de Ayuda a la Conducción,longplsp=Sistemas Avanzados de ayuda a la Conducción,longplural=Advanced Driver Assistance Systems]{adas}{ADAS}{Advanced Driver Assistance System}
\newacronym[longsp=Procesamiento Complejo de Eventos,longplsp=Procesamientos Complejos de Eventos,longplural=Complex Event Processings]{cep}{CEP}{Complex Event Processing}
\newacronym[longsp=Inteligencia Artificial,longplsp=Inteligencias Artificiales]{ai}{AI}{Artificial Intelligence}
\newacronym[longsp=Inteligencia Computacional,longplsp=Inteligencias Computacionales,longplural=Computational Intelligences]{ci}{CI}{Computational Intelligence}
\newacronym[longsp=Sistema Inteligente de Transporte,longplsp=Sistemas Inteligentes de Transporte,longplural=Intelligent Transport Systems]{its}{ITS}{Intelligent Transport System}
\newacronym[longsp=Red Neuronal Artificial,longplsp=Redes Neuronales Artificiales,longplural=Artificial Neural Networks]{ann}{ANN}{Artificial Neural Network}
\newacronym[longsp=Logica Difusa]{fl}{FL}{Fuzzy Logic}
\newacronym[longsp=Computación Evolutiva]{cev}{EC}{Evolutionary Computation}
\newacronym[longsp=Aprendizaje Automático]{ml}{ML}{Machine Learning}
\newacronym[longsp=Algoritmo Genético,longplsp=Algoritmos Genéticos,longplural=Genetic Algorithms]{ga}{GA}{Genetic Algorithm}
\newacronym[longsp=Sistema de Recomendación,longplsp=Sistemas de Recomendación,longplural=Recommender Systems]{rs}{RS}{Recommender System}
\newacronym[longsp=Procesamiento de Lenguaje Natural]{nlp}{NLP}{Natural Language Processing}
\newacronym[longsp=Sistema de Inferencia Difusa,longplsp=Sistemas de Inferencia Difusa, longplural=Fuzzy Inference systems]{fis}{FIS}{Fuzzy Inference System}
\newacronym[longsp=Sistema de Control Difuso difusa, longplsp=Sistemas de Control Difuso, longplural=Fuzzy Control systems]{fcs}{FCS}{Fuzzy Control System}
\newacronym[longsp=Sistema Multiagente,longplsp=Sistemas Multiagente, longplural=Multi-Agent Systems]{mas}{MAS}{Multi-Agent System}
\newacronym[longsp=Autómata Celular, longplsp=Autómatas Celulares, longplural=Cellular Automata]{ca}{CA}{Cellular Automaton}
\newacronym[longsp=Software Abierto]{oss}{OSS}{Open Source Software}
\newacronym[longsp=Inteligencia Artificial Distribuida]{dai}{DAI}{Distributed Artificial Intelligence}
\newacronym[longsp=Inteligencia de enjambre]{si}{SI}{Swarm Intelligence}
\newacronym[longsp=Estrategia Evolutiva, longplsp=Estrategias Evolutivas, longplural=Evolution Strategies]{es}{ES}{Evolution Strategy}
\newacronym[longsp=Programación Genética]{gp}{GP}{Genetic Programming}
\newacronym[longsp=Programación Genética Guiada por Gramáticas]{gggp}{GGGP}{Grammar Guide Genetic Programming}
\newacronym{ghr}{GHR}{Gazis-Herman-Rothery}
