\documentclass[a4paper,nobib,makeidx,justified,twoside,symmetric]{tufte-book}

\usepackage[utf8]{inputenc}
\usepackage[spanish]{babel}
\usepackage{ntheorem}
\usepackage{units}
\usepackage{listingsutf8}

\setlength{\parskip}{0.5\baselineskip}%
\setlength{\parindent}{0pt}%

% Package ntheorem.
\theoremseparator{:}
\newtheorem{hyp}{Hipótesis}
\newtheorem{definition}{Definición}

% Package listingutf8
\lstset
{
	language=[LaTeX]TeX,
	breaklines=true,
	basicstyle=\tt\scriptsize,
	keywordstyle=\color{blue},
	identifierstyle=\color{magenta},
}

\def\thetitle{Modelado de comportamiento de conductores con técnicas de Inteligencia Computacional}
\def\theauthor{Alberto Díaz Álvarez}
\def\thedate{\today}

\hypersetup{colorlinks}
\title{\thetitle}
\author[\theauthor]{\theauthor}
\authortitle{Máster en Ciencias y Tecnologías de la Computación}
\department{Departamento de Inteligencia Artificial}
\school{Escuela Técnica Superior de Ingenieros Informáticos}
\university{Universidad Politécnica de Madrid}
\advisor{Dr. Francisco Serradilla García}
\advisortitle{Doctor en Inteligencia Artificial}

\IfFileExists{bergamo.sty}{\usepackage[osf]{bergamo}}{} % Bembo
\IfFileExists{chantill.sty}{\usepackage{chantill}}{} % Gill Sans

%\usepackage{microtype}
\usepackage{booktabs}
\usepackage{graphicx}

\setkeys{Gin}{width=\linewidth,totalheight=\textheight,keepaspectratio}
\graphicspath{{images/}}

\usepackage{fancyvrb}
\fvset{fontsize=\normalsize}

\usepackage{xspace}

% Inserts a blank page
\newcommand{\blankpage}{\newpage\hbox{}\thispagestyle{empty}\newpage}

\newcommand{\TODO}{\textcolor{red}{\bf TODO!}\xspace}


\usepackage{marginfix}
\usepackage{mparhack}

\usepackage[acronym,shortcuts]{glossaries}

\makeglossaries

\glsaddkey
{longsp}% key
{}% default value
{\glsentrylongsp}% new command analogous to \glsentrylong
{\Glsentrylongsp}% new command analogous to \Glsentrylong
{\acrlongsp}% new command analogous to \acrlong
{\Acrlongsp}% new command analogous to \Acrlong
{\ACRlongsp}% new command analogous to \ACRlong

% Add new key for plural long Spanish form:
\glsaddkey
{longplsp}% key
{}% default value
{\glsentrylongplsp}% new command analogous to \glsentrylongpl
{\Glsentrylongplsp}% new command analogous to \Glsentrylongpl
{\acrlongplsp}% new command analogous to \acrlongpl
{\Acrlongplsp}% new command analogous to \Acrlongpl
{\ACRlongplsp}% new command analogous to \ACRlongpl

% Provide conditional to test if longsp/longplsp has been set
\newcommand*{\glsifhaslongsp}[3]{%
	\ifcsempty{glo@#1@longsp}{#3}{#2}%
}
\newcommand*{\glsifhaslongplsp}[3]{%
	\ifcsempty{glo@#1@longplsp}{#3}{#2}%
}

% Define new acronym style:

\newacronymstyle{spanish}
{% base the display style on 'long-short'
	\GlsUseAcrEntryDispStyle{long-short}%
}%
{% base the definitions on 'long-short'
	\GlsUseAcrStyleDefs{long-short}%  
	% Make some custom modifications for the first use display.
	% Singular, no case change:
	\renewcommand*{\genacrfullformat}[2]{%
		\glsifhaslongsp{##1}%
		{% has Spanish version:
			\glsentrylongsp{##1}##2\space
			(\firstacronymfont{\glsentryshort{##1}}, \glsentrylong{##1})%
		}%
		{%
			\glsentrylong{##1}##2\space
			(\firstacronymfont{\glsentryshort{##1}})%
		}%
	}%
	% Singular, first letter upper case:
	\renewcommand*{\Genacrfullformat}[2]{%
		\glsifhaslongsp{##1}%
		{% has Spanish version:
			\Glsentrylongsp{##1}##2\space
			(\firstacronymfont{\glsentryshort{##1}}, \glsentrylong{##1})%
		}%
		{%
			\Glsentrylong{##1}##2\space
			(\firstacronymfont{\glsentryshort{##1}})%
		}%
	}%
	% Plural, no case change:
	\renewcommand*{\genplacrfullformat}[2]{%
		\glsifhaslongplsp{##1}%
		{% has Spanish version:
			\glsentrylongplsp{##1}##2\space
			(\firstacronymfont{\glsentryshortpl{##1}}, \glsentrylongpl{##1})%
		}%
		{%
			\glsentrylongpl{##1}##2\space
			(\firstacronymfont{\glsentryshortpl{##1}})%
		}%
	}%
	% Plural, first letter upper case:
	\renewcommand*{\Genplacrfullformat}[2]{%
		\glsifhaslongplsp{##1}%
		{% has Spanish version:
			\Glsentrylongplsp{##1}##2\space
			(\firstacronymfont{\glsentryshortpl{##1}}, \glsentrylongpl{##1})%
		}%
		{%
			\Glsentrylongpl{##1}##2\space
			(\firstacronymfont{\glsentryshortpl{##1}})%
		}%
	}%
}

% switch to the new style:
\setacronymstyle{spanish}

% Define a new glossary style that checks for the existence of
% the longsp field.
\newglossarystyle{listsp}{%
	\setglossarystyle{list}% base style on the list style
	\renewcommand*{\glossentry}[2]{%
		\item[\glsentryitem{##1}%
		\glstarget{##1}{\glossentryname{##1}}]
		\glossentrydesc{##1}%
		\glsifhaslongsp{##1}{\space(\glsentrylongsp{##1})}{}%
		\glspostdescription\space ##2}%
}


% Aquí todas las definiciones y acrónimos
\newglossaryentry{latex}{
	name=latex,
	description={Is a mark up language specially suited for scientific documents}
}

\newacronym[longsp=Estudio Naturalista de Conducción,longplsp=Estudios Naturalistas de Conducción,longplural=Naturalistic driving studies]{nds}{NDS}{Naturalistic driving study}
\newacronym[longsp=Reconocimento el Entorno Intravehicular,longplsp=Reconocimientos del Entorno Intravehiculares,longplural=Intra-vehicular Context Awarenesses]{ivca}{IvCA}{Intra-vehicular Context Awareness}
\newacronym[longsp=Sistema Avanzado de Ayuda a la Conducción,longplsp=Sistemas Avanzados de ayuda a la Conducción,longplural=Advanced Driver Assistance Systems]{adas}{ADAS}{Advanced Driver Assistance System}
\newacronym{torcs}{TORCS}{The Simulated Car Racing Championship}
\newacronym[longsp=Procesamiento Complejo de Eventos,longplsp=Procesamientos Complejos de Eventos,longplural=Complex Event Processings]{cep}{CEP}{Complex Event Processing}
\newacronym[longsp=Inteligencia Artificial,longplsp=Inteligencias Artificiales]{ai}{AI}{Artificial Intelligence}
\newacronym[longsp=Inteligencia Computacional,longplsp=Inteligencias Computacionales,longplural=Computational Intelligences]{ci}{CI}{Computational Intelligence}
\newacronym[longsp=Sistema Inteligente de Transporte,longplsp=Sistemas Inteligentes de Transporte,longplural=Intelligent Transport Systems]{its}{ITS}{Intelligent Transport System}
\newacronym[longsp=Sistema Multi-Agente,longplsp=Sistemas Multi-Agentes,longplural=Multi-Agent Systems]{mas}{MAS}{Multi-Agent System}
\newacronym[longsp=red neuronal artificial,longplsp=redes neuronales artificiales,longplural=artificial neural networks]{ann}{ANN}{artificial neural network}
\newacronym[longsp=lógica difusa]{fl}{FL}{fuzzy logic}
\newacronym[longsp=Computación Evolutiva]{cev}{EC}{Evolutionary Computation}
\newacronym{sc}{SC}{Soft-Computing}
\newacronym{hc}{HC}{Hard-Computing}
\newacronym{sumo}{SUMO}{Simulation of Urban MObility}
\newacronym{traci}{TraCI}{Traffic Control Interface}
\newacronym{gpl}{GPL}{general public license}
\newacronym[longsp=aprendizaje automático]{ml}{ML}{machine learning}
\newacronym[longsp=algoritmo genético,longplsp=algoritmos genéticos,longplural=genetic algorithms]{ga}{GA}{genetic algorithm}
\newacronym[longsp=sistema de recomendación,longplsp=sistemas de recomendación,longplural=recommender systems]{rs}{RS}{recommender system}
\newacronym[longsp=procesamiento de lenguaje natural]{nlp}{NLP}{natural language processing}

\makeglossaries

\usepackage{smartdiagram}
\usepackage[caption=false]{subfig}

\usepackage{tikz}
\usetikzlibrary{matrix,arrows,arrows.meta,shapes,trees,mindmap}
\usepackage{smartdiagram}

\usepackage{pgfplots}

\usepackage{todonotes}

\usepackage{pifont}
\newcommand{\yep}{\textcolor{ForestGreen}{\ding{51}}}
\newcommand{\nop}{\textcolor{Red}{\ding{55}}}
\newcommand{\na}{\textcolor{Dandelion}{--}}
 
\usepackage[
	type={CC},
	modifier={by-nc-sa},
	version={3.0},
]{doclicense}

%\setcounter{secnumdepth}{2}
\titleformat{\chapter}%
[display]% shape
{\relax\ifthenelse{\NOT\boolean{@tufte@symmetric}}{\begin{fullwidth}}{}}% format applied to label+text
	{\itshape\huge\chaptertitlename~\thechapter}% label
	{0pt}% horizontal separation between label and title body
	{\huge\rmfamily\itshape}% before the title body
	[\ifthenelse{\NOT\boolean{@tufte@symmetric}}{\end{fullwidth}}{}]% after the title body

\usepackage{imakeidx}
\makeindex
\newcommand\idx[1]{#1\index{#1}}

% Para definición de ecuaciones por partes (dos partes)
\newcommand{\twopartf}[4]
{
	\left\{
	\begin{array}{ll}
		#1 & \mbox{si } #2 \\
		#3 & \mbox{si } #4
	\end{array}
	\right.
}

\usepackage{morefloats}

\begin{document}

\hyphenpenalty=10

\frontmatter
\maketitle
\cleardoublepage
\begin{fullwidth}
~\vfill
\thispagestyle{empty}
\setlength{\parindent}{0pt}
\setlength{\parskip}{\baselineskip}
\theauthor

\par{
	\textit{\thetitle}\\*
	Tesis doctoral, \thedate\\*
	Revisores: Rev1, Rev2 y Rev3\\*
	Director: \thanklessadvisor
	}

\par{
	\textbf{Instituto Universitario de Investigación del Automóvil}\\*
	Universidad Politécnica de Madrid\\*
	Campus Sur UPM, Carretera de Valencia (A-3), km7\\*
	28031 Madrid
}

\par{
	\doclicenseThis
}

\end{fullwidth}

\cleardoublepage
~\vfill
\thispagestyle{empty}
\begin{doublespace}
\noindent\fontsize{16}{20}\selectfont\itshape
\nohyphenation .
\end{doublespace}
\vfill
\vfill


\cleardoublepage

\tableofcontents
\printglossaries
\listoffigures
\listoftables

\mainmatter
\part{Introducción}
\chapter{Introducción}
\label{ch:intro}

Es un hecho que la \ac{ai} en general (y la \ac{ci} en particular) como área de conocimiento ha experimentado un creciente interés en los últimos años. Esto no siempre ha sido así, ya que después de un nacimiento muy esperanzador, con mucho optimismo, le siguieron unas épocas de apenas avance (ver cuadro~\ref{tbl:ai-timeline}). Sin embargo, en la actualidad es muy difícil encontrar un campo que no se beneficie directamente de sus técnicas.

\begin{margintable}
	\caption{Línea temporal de los principales hitos en la \glsentrytext{ai}. Actualmente la \glsentrytext{ai} está ofreciendo resultados muy prometedores áreas como la conducción autónoma, el procesamiento del lenguaje natural o el análisis de sentimiento entre muchos otros.}
	\label{tbl:ai-timeline}
	\centering
	\begin{minipage}[t]{\linewidth}
		\newcommand\ytl[2]{
			\parbox[b]{1cm}{\hfill{\color{cyan}\bfseries\sffamily #1}~$\cdots$~}
			\parbox[c]{3.8cm}{\vspace{7pt}\raggedright\sffamily #2.\\[0pt]}\\[0pt]}
		\color{gray}

		\ytl{1956}{Conferencia de Dartmouth. Nace el campo de la \ac{ai}. Se destinan muchos recursos debido al potencial del nuevo campo.}
		\ytl{1974}{No llegan los resultados esperados. Se suspenden financiaciones y se deja de investigar en muchas áreas (\textit{AI Winter})}
		\ytl{1980}{Aparecen los sistemas expertos. Muy prometedores. Acaparan la práctica totalidad de la investigación en \ac{ai}.}
		\ytl{1987}{De nuevo los resultados no son lo esperado y vuelve a dinsminuir el trabajo en \ac{ai}.}
		\ytl{1990}{La mejora de las prestaciones, la ubicuidad de los ordenadores y nuevos conceptos (e.g. agentes) hacen que la investigación en el área vuelva a crecer. Se replantea el concepto de \ac{ai}.}
		\ytl{2000}{Se retoman investigaciones relacionadas con el aprendizaje profundo. Aumenta la investigación en el área de las redes neuronales y redes bayesianas.}
		\ytl{2006}{Conferencia de Dartmouth. Se analizan los avances y se debate sobre la \ac{ai} a 50 años vista. Crece la expectación y el interés en el campo}
		\ytl{2007}{Crece el interés en el aprendizaje automático debido a los resultados obtenidos y el aumento de las fuentes de datos}
		\bigskip
	\end{minipage}%
\end{margintable}

Una de sus razones es su caracter multidisciplinar ya que, si bien se la define como área perteneciente al campo de la informática, es transversal a muy diferentes campos, como pueden ser por ejemplo la biología, neurología o la psicología, entre otros.

Dentro del área de la \acrlongsp{ai} es común diferenciar dos tipos de aproximaciones a la hora de hablar de cómo representar el conocimiento: la \textbf{\acrlongsp{ai} clásica}, que postula que el conocimiento como tal se puede reducir a un conjunto de símbolos con operadores para su manipulación, y la \textbf{\acrlongsp{ci}}, que defiende que al conocimiento se llega a través del aprendizaje, y que basa sus esfuerzos en la simulación de elementos de bajo nivel que subyacen a los comportamientos inteligentes esperando que el conocimiento \enquote{emerja} de éstos.

El límite entre ambos conjuntos no está perfectamente definido, máxime si tenemos en cuenta las diferentes terminologías existentes, las sinergias entre distintas técnicas dentro del área y los diferentes puntos de vista sobre éstas por parte de los autores. Sin embargo, una de las principales diferencias de ambos paradigmas es el punto de vista a la hora de solucionar problemas, siendo la aproximación \textbf{top-down} la usada en problemas de \acrshort{ai} clásica y la \textbf{bottom-up} la típica usada en la \acrshort{ci}. Revisaremos las diferencias entre conceptos de diferentes autores en el capítulo \ref{ch:sota-ci}\sidenote{Una aproximación \textit{top-down} a los problemas funciona definiendo primero el algoritmos que resuelve el problema para posteriormente ejecutarlo y llegar así a la solución exacta. Por otro lado, una aproximación \textit{bottom-up} el algoritmo de resolución no se programa, sino que se aprende, llegando él sólo a soluciones no necesariamente exactas pero sí lo suficientemente buenas para ser aceptadas.}.

Uno de los campos de aplicación es el de los \acp{its}. Éstos se definen como un conjunto de aplicaciones orientadas a gestionar el transporte en todos sus aspectos y granularidades (e.g. conducción eficiente, diseño de automóviles, gestión del tráfico o señalización en redes de carreteras) para hacerlos más eficientes y seguros. El interés es tal que en el año 2010 se publicó la directiva 2010/40/UE (ver \cite{parliament2010directive}) donde se estableció el marco de implantación de los ITS en la Unión Europea\footnote{En esta directiva, los ITS se definen como \textit{aplicaciones avanzadas que, sin incluir la inteligencia como tal, proporcionan servicios innovadores en relación con los diferentes modos de transporte y la gestión del tráfico y permiten a los distintos usuarios estar mejor informados y hacer un uso más seguro, más coordinado y «más inteligente» de las redes de transporte.}}.

En el caso concreto del comportamiento al volante, es interesante la evaluación de los conductores para conocer su manera de actuar en determinados escenarios, y poder extraer información de éstos que nos permitan, por ejemplo, detectar qué factores pueden afectar más o menos a determinados indicadores (e.g. el consumo estimado para una ruta en concreto). Sin embargo, la evaluación en distintos escenarios puede no ser interesante debido a limitaciones existentes, como pueden ser, por ejemplo, el tiempo, el dinero o la peligrosidad del escenario.

Los simuladores de tráfico son una solución para muchas de estas limitaciones, pero suelen basar su funcionamiento en conductores y vehículos (normalmente concebidos como una única entidad) basándose en modelos de conductor que responden a funciones más o menos complejas, además con pocas o ningunas opciones de personalización. Esto provoca que dichos modelos se adapten poco al modelo de un conductor en concreto.

Esta tesis pretende explorar el tema de la generación de modelos de conductor para simuladores que respondan al comportamiento de conductores reales usando, para ello, técnicas pertenecientes al campo de la \ac{ci}.

Concretamente pretende desarrollar un método para el análisis de la eficiencia de los conductores realizando, para ello, un modelo del perfil de conducción a partir de técnicas de la \ac{ci} y aplicándolo a un entorno de simulación basado en \Acp{mas}. Así, una vez configurado el entorno, se podrán estudiar aspectos generales como la evolución del tráfico con determinados perfiles o particulares como el estilo de conducción o el impacto de los sistemas de asistencia.

\section{Motivación}

Los conceptos introducidos al comienzo del capítulo obedecen a una \textit{necesidad} (aquí como eufemismo de problema) de la sociedad en la que vivimos, y que afecta tanto a nuestra generación como afectará a las venideras: la eficiencia en la conducción. Dado que es imprescindible saber que existe un problema para arreglarlo, nada mejor que puntualizar algunos hechos de sobra conocidos:

\begin{itemize}
	\item En el año 2014, el número de vehículos a nivel mundial asciende a más de $1.200$ millones, con una tendencia creciente \cite{oica2014motrate}. Reducir en un pequeño porcentaje el consumo durante la conducción evita la emisión de toneladas de gases considerados nocivos para el medio ambiente y el ser humano\footnote{Uno puede argumentar que el parque automovilístico se recicla con nuevos vehículos eléctricos categorizados \enquote{de consumo 0}. La triste realidad es que estos vehículos consumen la electricidad generada actualmente de una mayoría de centrales de combustibles fósiles y nucleares. Además, mientras que en países desarrollados el crecimiento ha sido en torno al 4-7\%, en países subdesarrollados, donde no existe aun infraestructura para la recarga de vehículos eléctricos, dicho crecimiento ha superado el 120\%.}.
	\item Debemos abandonar los combustibles fósiles antes de que éstos nos abandonen a nosotros. Si bien es cierto que existen diferentes puntos de vista acerca de cuándo se agotarán las reservas de petróleo, también lo es que los combustibles fósiles son recursos \textbf{finitos}. Lo más probable es que no se llegue a agotar debido a la ley de la oferta y la demanda, pero hay que recordar que el petróleo se usa como base para la producción de otros muchos tipos de productos, como por ejemplo la vaselina, el asfalto o los plásticos.
	\item Independientemente del momento en el que se agoten los recursos, hay que hacer notar que la emisión de gases está correlacionada con el aumento de la temperatura del planeta, hecho que se ilustra en la figura~\ref{fig:co2-global-warm-correlation}. De seguir con el ritmo de consumo actual, se teme llegar a un punto de no retorno con consecuencias catastróficas para la vida en el planeta.

\begin{marginfigure}
	\centering
	\includegraphics{images/co2-global-warm-correlation}
	\caption{Desde el comienzo de la revolución industrial, el uso masivo de combustibles fósiles y el crecimiento de la población propició un aumento desproporcionado de $CO_2$ a la atmósfera, tendencia que sigue en aumento aún con la (lenta) adopción del vehículo eléctrico. La gráfica muestra cómo ambos valores parecen estar correlacionados. Fuente: Environmental Defense Fund (\url{edf.org}).}
	\label{fig:co2-global-warm-correlation}
\end{marginfigure}

	\item Algo más cercano en el tiempo. La conducción eficiente afecta directamente a factores correlacionados con el número de accidentes de tráfico. Un factor de sobra conocido es el de la velocidad, factor relacionado no sólo con el número sino con la gravedad de los accidentes\cite{imprialou2016re}. Otros indicadores son las aceleraciones, deceleraciones y maniobras de cambio de dirección, cuya frecuencia es inversamente proporcional a la eficiencia en la conducción y directamente proporcional a la agresividad, falta de seguridad y accidentes (\cite{dingus2006100} y \cite{lerner2010exploration}).
\end{itemize}

Estos hechos son solo algunos que ponen de manifiesto la necesidad de centrarse en el problema de cómo hacer de la conducción una actividad más eficiente y segura.

La \textbf{conducción eficiente} o \textit{eco-driving} es definida como la aplicación de una serie de reglas de conducción con el objetivo de reducir el consumo de combustible, independientemente del tipo (e.g. electricidad, gasolina, gas natural, \ldots).

Ser capaces de identificar o al menos estimar qué conductores son considerados como no eficientes es importante debido a que, de esta manera, se pueden identificar los hábitos recurrentes en este tipo de perfil y adecuar la formación para eliminar dichos hábitos. Más aún teniendo en cuenta la relación existente entre la peligrosidad y algunas conductas agresivas. Un ejemplo donde la identificación de perfiles no eficientes pueden tener impacto claro económico y social es el de las empresas cuya actividad se basa en el transporte de mercancías o de personas.

Sin embargo, identificar la conducta de un conductor no es fácil, dado que su comportamiento se ve condicionado por numerosos factores como el estado de la ruta, el del tráfico o el estado físico o anímico. Además, la ambigüedad de las situaciones dificulta todavía más la identificación. Por ejemplo, un conductor puede ser clasificado en un momento como agresivo o no eficiente en una situación únicamente porque su comportamiento ha sido condicionado por las malas reacciones por parte de los demás conductores.

El análisis de todos los posibles casos es una tarea prácticamente imposible. Por ello, las simulaciones pueden dar una estimación de los posibles resultados de un estudio en el mundo real. Las simulaciones con \acp{mas} representan a los conductores como agentes permitiendo la evaluación del comportamiento tanto individual como general del sistema en base a sus individuos a través de iteraciones discretas de tiempo.

Si dichos agentes son obtenidos mediante la modelización de conductores a partir de sus datos reales, su comportamiento en la simulación podría ser considerado como fuente de datos para condiciones de tráfico y/o rutas no contempladas en el mundo real. De esta forma, se dispondría de un marco de trabajo para la comparación de diferentes conductores sin necesidad de exponerlos a todos y cada uno de los posibles eventos posibles. También sería factible evaluar sistemas de asistencia evitando los problemas de no comparabilidad de condiciones del entorno entre pruebas.

Demostrar que la evaluación de un modelo del conductor en entornos simulados es equivalente a la evaluación de conductores en entornos reales implica que se pueden comparar dos conductores usando un criterio objetivo, es decir, sin depender del estado del resto de factores a la hora de realizar la prueba de campo. Dicho de otro modo, implicaría que es posible comparar la eficiencia de dos conductores independientemente del estado del tráfico e, incluso, sobre rutas diferentes.

\section{Objetivos}
\label{ch:intro:objectives}

El objetivo de esta tesis doctoral es la de demostrar la hipótesis~\ref{hyp:hypothesis-1}, quedando dicha demostración dentro de los límites impuestos por los supuestos y restricciones indicados más adelante.

\begin{hyp}[H\ref{hyp:hypothesis-1}] \label{hyp:hypothesis-1}
	La aplicación de técnicas pertenecientes al campo de la \ac{ci} con datos extraídos de un entorno de microsimulación de espacio continuo y tiempo discreto basado en sistemas multiagentes permitirá modelar, de manera fiel a la realidad, el comportamiento de conductores reales.
\end{hyp}

Por tanto, el objetivo de la tesis es el de simular el comportamiento de conductores en entornos de micro-simulación a partir de su comportamiento en entornos reales usando técnicas de \ac{ci}. Para ello se consideran los siguientes objetivos específicos:

\begin{itemize}
	\item Estudiar y aplicar técnicas de la \ac{ci} sobre el área de la conducción.
	\item Realizar un \gls{nds}\sidenote{Los \gls{nds} basan su funcionamiento en la captura masiva de datos de conducción, normalmente involucrando una gran cantidad de sensores, para analizar el comportamiento del conductor, las características del vehículo, la vía, etcétera. La cantidad de sensores y la velocidad de captura hacen que la tarea de analizar y extraer conclusiones sea una tarea prácticamemte imposible para un humano, por lo que es necesario el uso de técnicas de análisis de datos que suelen recaer en los campos de la estadística y del aprendizaje automático.} sobre conductores reales para:
	\begin{enumerate}
		\item Generar modelos personalizados de conductor a partir de los datos de conducción obtenidos.
		\item Aplicar modelos de conductores a entornos de simulación multiagente.
		\item Validar los modelos de conductor contra conductores reales.
	\end{enumerate}
	\item Estudiar la efectividad de sistemas de asistencia encaminados a mejorar la eficiencia y analizar el comportamiento de conductor.
\end{itemize}

\subsection{Supuestos}

\begin{itemize}
	\item Suponemos que el comportamiento de un conductor es el comportamiento en línea y el comportamiento de cambio de carril\footnote{Son conocidos en la literatura como \textit{car-following} y \textit{lane-changing} respectivamente. Entraremos en detalle sobre ambos conceptos en el capítulo~\nameref{ch:sota-behavior-models}}.
	\item La circulación se realizará por la dereccha.
	\item Los datos de los que extraer el comportamiento se corresponderán con lecturas realizadas durante el día, con buena visibilidad y sin lluvia.
	\item El tipo de vehículo sobre el que modelar el comportamiento será el de un utilitario.
	\item El conductor a modelar pertenecerá al grupo más representativo de conductores. Esto se corresponde con varón de $35$ a $39$ años (ver figura~\ref{fig:drivers-census}).
\end{itemize}

\begin{marginfigure}
	\resizebox {\linewidth} {!} {
	\begin{tikzpicture}
	\begin{axis}[legend style={anchor=north east},
	symbolic x coords={15--17,18--20,21--24,25--29,30--34,35--39,40--44,45--49,50--54,55--59,60--64,65--69,70--74,74+}, xtick=data, x tick label style={rotate=90,anchor=east}]
	\addlegendentry{Hombres}
	\addplot[mark=*,thick,blue] coordinates {
		(15--17,39341)
		(18--20,318037)
		(21--24,729846)
		(25--29,1097874)
		(30--34,1472038)
		(35--39,1828905)
		(40--44,1768957)
		(45--49,1688069)
		(50--54,1460193)
		(55--59,1256212)
		(60--64,1082591)
		(65--69,974768)
		(70--74,709285)
		(74+,1190514)
	};
	
	\addlegendentry{Mujeres}
	\addplot[mark=diamond*,thick,red] coordinates {
		(15--17,15697)
		(18--20,238534)
		(21--24,651961)
		(25--29,1054377)
		(30--34,1337432)
		(35--39,1568926)
		(40--44,1445740)
		(45--49,1321330)
		(50--54,1055472)
		(55--59,810168)
		(60--64,569461)
		(65--69,387158)
		(70--74,189829)
		(74+,1138945)
	};
	
	
	\end{axis}
	\end{tikzpicture}
	}
	\caption{Último censo de conductores según género segmentado por edades. Fuente: Dirección General de Tráfico (\url{dgt.es}).}
	\label{fig:drivers-census}
\end{marginfigure}
\subsection{Restricciones}

\begin{itemize}
	\item El sistema multiagente hará uso de \gls{dvu} como agentes, es decir, usando la tupla (conductor, vehículo) como un todo.
	\item La resolución máxima del modelo creado es de 1Hz.
	\item En el caso de los modelos que hacen uso de redes neuronales artificiales, no se pueden explicar las razones del comportamiento inferido.
\end{itemize}

\section{Estructura de la tesis}
\label{ch:intro:structure}

La tesis está estructurada de la siguiente manera:

En los capítulos \ref{ch:sota-traffic-simulators-and-mas}, \ref{ch:sota-ci} y \ref{ch:sota-behavior-models} se expone la revisión realizada del estado de la cuestión donde se explica en qué punto se encuentra la literatura de los temas en los que se apoya la presente tesis.

En el capítulo XXX se explica el método seguido para la confirmación de la hipótesis describiendo además las instrumentaciones, los conjuntos de datos obtenidos, las técnicas utilizadas y las aplicaciones desarrolladas.

Por último en el capítulo \ref{ch:conclusions} se exponen los resultados y las conclusiones extraídas de la tesis. Además, tras las conclusiones se indican una serie de posibles líneas futuras de trabajo consideradas interesantes tras la realización de la tesis.

\part{Estado de la cuestión}
\chapter{\glsentrylongsp{ci}}
\label{ch:sota-ci}

El comportamiento de una persona se ve influenciado por una gran cantidad de variables. Identificar las relaciones entre éstas es en la mayoría de las ocasiones una tarea que va de lo muy difícil a lo imposible, más aún si añadimos que éstas son muy numerosas y pueden llegar a ser imposibles de cuantificar o incluso de detectar.

La \ac{ci} engloba un conjunto de técnicas que facilitan enormemente estas tareas. En este capítulo se ofrece una perspectiva de la literatura actual sobre las técnicas de la inteligencia computacional que son de interés para esta tesis. Desarrollaremos el concepto de \ac{ci} y la noción de \enquote{aprendizaje} como método de resolución de problemas en la \ac{ci}. Introduciremos algunas técnicas usadas dentro de la \ac{ci} y desarrollaremos las tres principales y sobre las que se trabaja en esta tesis: \glsentrylongsp{ann}, \glsentrylongsp{fl} y \glsentrylongsp{cev}.

\section{\glsentrylongsp{ai} vs. \glsentrylongsp{ci}}

¿Qué es la \ac{ci}? Para entender el significado de éste término tenemos que entender cómo ha evolucionado el término \ac{ai} a lo largo de los años.

El primer concepto a introducir es el de "conexionismo". Se puede considerar a Santiago Ramón y Cajal como el principal precursor de esta idea por sus trabajos acerca de la estructura de las neuronas y su conexión (e.g. \cite{y1888estructura} y~\cite{ramon1904textura}). Otros prefieren citar el trabajo de Donald Hebb acerca de la Teoría del aprendizaje~\cite{hebb19680} como el primer trabajo sobre este tema. Independientemente, en el conexionismo se considera que la mente y el conocimiento surgen surgen de redes formadas por unidades sencillas interconectadas (e.g. neuronas).

Por otro lado, en 1950, Alan Turing publicó un artículo que comenzaba con la frase \textit{\enquote{Can machines think?\sidenote{Es aquí donde introduce el Test de Turing como método para probar si una máquina es capaz de exhibir un comportamiento inteligente similar al de un ser humano. El funcionamiento es el siguiente: hay tres participantes, dos humanos y una máquina, todos separados entre sí pero pudiendo intercambiarse mensajes de texto. Uno de los humanos le hace preguntas al otro humano y a la máquina, y éstos le responden. Si el humano que hace las preguntas no es capaz de discernir qué respuestas vienen de la máquina y qué respuestas del otro humano se puede concluir que la máquina es inteligente.}}}~\cite{turing1950computing}. Se puede considerar este momento como el punto donde se estableció el objetivo a largo plazo del campo de la \ac{ai}, ya que en el artículo Turing propuso un método para determinar si una máquina era capaz de pensar o no. Sin embargo, no fue hasta $1956$ en la Conferencia de Dartmouth~\cite{mccarthy1956dartmouth} donde John McCarthy acuñó el término~\ac{ai} a la vez que presentó el tema de la conferencia: "¿puede una máquina pensar\sidenote{Hay que destacar que el propio concepto de \enquote{pensar} es en sí un tema controvertido en el propio ser humano: ¿pensar es algo inherentemente biológico? ¿surje de la mente? Tanto si sí como si no, ¿de qué forma lo hace?. El experimento mental de la habitación china (ver~\cite{preston2002views}) propuesto por John Searle nace precisamente para rebatir la validez del Test de Turing. En esencia trata de un Test de Turing donde la máquina ha aprendido a hablar chino y es reemplazada por una persona que no sabe nada del idioma pero que va equipada con un listado de correspondencias del tipo "cada vez que entra esta secuencia de ideogramas, devuelve esta secuencia de ideogramas". Cuando una persona le manda mensajes en chino, esta otra responde, pero ¿podemos decir que dicha persona sabe chino? Evidentemente no. Entonces, cómo podemos asegurar que la máquina ha \enquote{aprendido} chino. Y lo que es más intrigante, si la máquina es capaz de realizar una acción sin entender lo que hace y por qué lo hace, ¿qué garantías tenemos de que el humano sí es capaz? Si los ordenadores operan sobre símbolos sin comprender el verdadero contenido de éstos, ¿hasta qué punto los humanos lo hacen de forma diferente}?".

En este punto la investigación en~\ac{ai} recibió muchísima atención por parte de investigadores y gobiernos, lo que se tradujo en financiación. Los estudios estaban dominados por aquellos relacionados con la idea del conexionismo hasta que aproximadamente en $1969$ se publicó el libro \textit{Perceptrons}~\cite{minsky1969perceptrons} de Marvin Minsky y Seymour Papert donde se expusieron las limitaciones de los modelos de redes neuronales desarrollados hasta la fecha. El impacto fue tal que se abandonó prácticamente por completo el campo de conexionismo, y por tanto una gran parte de la investigación en la \ac{ai}. Es lo que se conoce como \textit{AI Winter}\footnote{Indicar que también hubo otros factores como los ecnómicos, la crisis del software, etcétera. https://en.wikipedia.org/wiki/AI\_winter\#Lasting\_effects\_of\_the\_AI\_winters}.

Debido al \textit{AI Winter}, el conexionismo no estuvo presente en la literatura científica durante prácticamente dos décadas. Fue en $1980$ cuando apareció un nuevo concepto dentro de la \ac{ai} que acaparó el interés por el campo de nuevo y que se considera como el primer caso de éxisto dentro del campo: los Sistemas Expertos~\cite{russell2003artificial}. A finales de la década, sin embargo, empezaron a resurgir los enfoques conexionistas, en gran parte por el surgimiento de nuevas formas de entrenamiento de redes multicapa o por el concepto de funciones de activación no lineales (e.g. trabajos como~\cite{rumelhart1985learning} o~\cite{cybenko1989approximation}). En este momento los sistemas expertos empezaron a perder interés\footnote{A esta década se la conoce como segundo \textit{AI Winter} dado que la investigación sobre Sistemas Expertos se empieza a abandonar. Sin embargo no fue un abandono tan acusado como el del primer AI Winter.} frente al nuevo avance del conexionismo.

Frente al avance del conexionismo, algunas voces se alzaron contra lo que se consideraba como \enquote{el enfoque incorrecto} de la \ac{ai}\footnote{Es comprensible ya que el método clásico produce modelos fáciles de interpretar mientras que el enfoque conexionista produce modelos cuyo funcionamiento en general no es del todo deducible.}. Mientras que el enfoque clásico de la \ac{ai} postulaba que la mente operaba de la misma manera que una máquina de Turing, es decir, mediante operaciones sobre un lenguaje de símbolos, el enfoque del conexionismo postulaba que la mente, el comportamiento inteligente emergía de modelos a más bajo nivel. Además, otras técnicas no pertenecientes al conexionismo pero sí alienadas a éste debido a su enfoque de comportamiento emergente y aproximación (e.g. lógica difusa o algoritmos genéticos) ganaban popularidad y alimentaban el éxito de este tipo de técnicas que no cumplian el ideal de la aproximación exacta y simbólica de la \ac{ai}.

Esto provocó una explosión de terminologías para diferenciar las investigaciones de la propia~\ref{ai} clásica. Por un lado, se evitaba el conflicto nombrando el trabajo con un término más acorde con el comportamiento o técnica utilizada. Por otro, se separaba de las connotaciones negativas de \enquote{promesas incumplidas} que el término había adquirido con el paso de los años.

\begin{figure}
	\includegraphics{different-povs-ai}
	\caption{Objetivos que persigue la~\glsentrylong{ai}. Las filas diferencian si lo que se persigue es pensamiento o comportamiento mientras que las columnas separan si se persigue la inteligencia entendida como la humana o como el ideal de inteligencia (inteligencia racional).}
	\label{fig:different-povs-ai}
\end{figure}

Lo verdaderamente interesante es ver la evolución de la literatura, y por tanto de los objetivos de la \ac{ai} durante estos años. En el nacimiento del campo, se buscan literalmente máquinas que piensen como humanos, aunque en el conexionismo se habla de algo más general como lo es la mente. Con el paso de los años y los continuos choques con la realidad, la literatura gira hacia la búsqueda de conductas y comportamientos inteligentes, sin necesidad de que cubran todos los aspectos que hacen de un ser un ente inteligente. Es más, en el momento en que la cantidad de diferente terminología empieza a elevarse, se pone de manifiesto que existen diferentes puntos de vista para el mismo campo. En~\cite{russell2003artificial}, tras un análisis de las definiciones existentes en la literatura por parte de diferentes autores, se hace énfasis en este hecho mostrando los diferentes puntos de vista a la hora de hablar de lo que es la \ac{ai} (estos son sistemas que piensan como humanos, que actúan como humanos, que piensan racionalmente y que actúan racionalmente).

Volviendo al tema de la terminología, muchas de las diferentes técnicas se fueron agrupando dentro de diferentes áreas. Una de ellas es la conocida como \glsentrylong{ci}. Dado que persigue el mismo objetivo a largo plazo y que surje de la propia \ac{ai} parece lógico mantenerla como un subconjunto y no como un nuevo campo del conocimiento humano. La principal diferencia es la generalización de esa disputa o diferentes puntos de vista, la \ac{ai} clásica y los métodos no clásicos.

Podemos definir la \ac{ci} como la rama de la \ac{ai} que explora la búsqueda de conocimiento a partir del aprendizaje a partir de datos experimentales. A diferencia de la aproximación clásica de la \ac{ai}, buscan aproximaciones a la soluciones y no las soluciones exactas \textit{per se}. Esto es debido de la necesidad de otros puntos de vista en resolución de problemas cuando éstos son muy complejos, cuando los datos están incompletos o contienen ruido o cuando no pueden ser traducidos al lenguaje binario.

Se puede fijar el año $1994$ como el que el término \ac{ci} nace como área, coincidiendo con el cambio de nombre del \textit{IEEE Neural Networks Council} a \textit{IEEE Computational Intelligence Society}\footnote{\url{http://cis.ieee.org/}.}. Poco antes, en $1993$, Bob Marks en su trabajo~\cite{bezdek1993intelligence} presentó las que él consideraba diferencias fundamentales entre la \ac{ai} y la \ac{ci} para posteriormente resumirlas en la siguiente frase.

\blockquote{"Neural networks, genetic algorithms, fuzzy systems, evolutionary programming, and artificial life are the building blocks of CI.}

Durante estos años ganaba popularidad el concepto del \ac{sc}. Éste engloba las técnicas que buscan resolver problemas que manejan información incompleta o con ruido. Debido a que el conjunto de técnicas definidas como consituyentes del \ac{sc} son las mismas que las de la \ac{ci} algunos autores consideran el \ac{sc} como la rama que ocupa estas técnicas mientras que otros consideran ambos términos equivalentes. Nosotros consideramos que el \ac{sc} es un punto de vista de la computación más que de la \ac{ai} en contraposición con el \ac{hc}, y que la \ac{ci} hace uso de métodos del \ac{sc}.

\TODO Aquí dibujo de una taxonomía donde así: (inteligencia (clásica (hard computing)) (computacional (soft computing (lógica difusa, computación evolutiva, redes neuronales))))

"AI ranges from machines truly capable of thinking to search algorithms used to play board games"
"It has applications in nearly every way we use computers in society."
"The term artificial intelligence was first coined by John McCarthy in 1956 when he held the first academic conference on the subject"
"¿Puede una máquina pensar?. 
"Is the problem simply that we haven’t focused
enough resources on basic research, as is seen in the AI winter section, or is the complexity of AI one that we
haven’t yet come to grasp yet? (And instead, like in the case of computer Chess, we focus on much more
specialized problems rather than understanding the notion of ‘understanding’ in a problem domain.)"

Inteligencia computacional: habilidad de un ordenador de aprender una tarea específica a partir de datos u observaciones experimentales (esto es sacado de la wikipedia. Hay que buscar más definiciones en la literatura). Destacar \textbf{tarea específica} y \textbf{a partir de datos experimentales}.

Algunos autores lo consideran sinónimo de Soft Computing, pero no como no hay definición comúnmente aceptada, pues na.

La cosa es que hay procesos matemáticos muy complejos, con mucha interdependencia entre variables y factores que hacen que los problemas sean muy difíciles de modelar, más aún cuando el problema en su naturaleza es estocástico (poner algún ejemplo de algo en la naturaleza que se comporte de forma estocástica~\cite{siddique2013computational}).

But generally, computational intelligence is a set of nature-inspired computational methodologies and approaches to address complex real-world problems to which mathematical or traditional modelling can be useless for a few reasons: the processes might be too complex for mathematical reasoning, it might contain some uncertainties during the process, or the process might simply be stochastic in nature.[1] Indeed, many real-life problems cannot be translated into binary language (unique values of 0 and 1) for computers to process it. Computational Intelligence therefore provides solutions for such problems.

\section{Concepto de agente}

\section{Aprendizaje}

Técnicas de entrenamiento de modelos: supervisado, no supervisado, semisupervisado, refuerzo, ...
Técnicas de funcionamiendo: online y offline

\section{Ténicas en la \Glsentrylongsp{ci}}

¿Qué técnicas se usan actualmente y sobre qué problemas?

\section{\Glsentrylongplsp{ann}}

Técnicas de la inteligencia computacional usadas en esta tesis (redes neuronales artificiales(perceptrón multicapa, recurrentes y lstm), lógica difusa y computación evolutiva)

\section{\Glsentrylongsp{fl}}

Explicar lógica difusa y control difuso. Indicar los controladores difusos de segundo, tercer y sucesivos niveles.

\section{\Glsentrylongsp{cev}}

Explicar qué es la cev, la evolución de conceptos distintos a distintas escuelas de pensamiento del mismo concepto. 
\chapter{Simulación de tráfico}
\label{ch:sota-traffic-simulation}

El tráfico es un sistema caótico en el que intervienen un número muy elevado de diferentes variables, muchas de las cuales relacionadas entre si. Debido a esto, obtener modelos exactos de tráfico es una tarea prácticamente imposible y es por ello que la mayoría del trabajo cuyo objetivo es la predicción se realice en base a simuladores.

Los simuladores de tráfico son herramientas de software que, usando diferentes modelos, describen el tráfico como sistema, permitiendo:

\begin{itemize}
	\item Extraer resultados y conclusiones de escenarios de tráfico determinados.
	\item Implementar técnicas de tráfico sin necesidad de alterar el tráfico real.
	\item Reproducir exactamente un escenario.
	\item Introducir modificaciones en puntos determinados (e.g. espaciales o temoprales) de un escenario conocido para evaluar la divergencia en su evolución.
\end{itemize}

Los objetivos en la simulación de tráfico son los de hacer que los modelos se parezcan lo máximo posible a la realidad. En este capítulo vamos a ver cuál es la realidad actual de este tipo de simuladores, cuáles son sus diferentes tipologías y maneras de modelar los diferentes aspectos del tráfico y, posteriormente, realizaremos una evaluación de cuáles son los idóneos para nuestro trabajo.

Limitaremos el estudio no obstante a los simuladores de vehículos, obviando otros tipos de simulación de tráfico que no tienen que ver con esta temática, como por ejemplo los orientados a la evaluación de sistemas de señalización inteligentes (e.g.~\cite{jin2016evaluation}) o a la estimación de emisiones (e.g.~\cite{quaassdorff2016microscale}).

\section{Clasificación de simuladores de tráfico}

Los aspectos simulables y medibles del problema tráfico son muy diversos, dependiendo sobre todo del nivel de complejidad del tráfico\sidenote{Modelar una vía por la que circula un centenar de coches no es lo mismo que modelar una ciudad donde circulan millones.}, de qué queremos medir\sidenote{Evaluar a un conductor en una situación determinada o evaluar la evolución del flujo de tráfico en un cuello de botella causado por un accidente.} y de cómo lo modelamos\sidenote{Un autómata celular se modela de forma diferente a un modelo lineal de vías o carriles.}.

El resto de la sección ofrece una visión de las diferentes categorías en las que se clasifican los simuladores de tráfico.

\subsection{Tipos de simulador en función de la complejidad}

La complejidad en una simulación se refiere al nivel de detalle al que queremos llegar a la hora de modelar nuestra solución. Es evidente que según aumentamos el detalle en la simulación aumenta la cantidad de cálculo. Por ejemplo, si queremos modelar el comportamiento de $10$ billones de canicas cayendo por un tubo es considerablemente más eficiente modelarlas como un fluido con una serie de propiedades características que como una colección de elementos individuales, cada uno con sus propiedades (e.g. masa, aceleración, ...) e interaccionando entre sí.

\begin{figure}
	\centering
	\includegraphics{images/granularities-in-traffic-simulation}
	\caption{Taxonomía clásica de los simuladores en función de la granularidad (complejidad) de la simulación.}
	\label{fig:granularities-in-traffic-simulation}
\end{figure}

En el caso de los simuladores de tráfico es lo mismo. En éstos existe un amplio intervalo de granularidades, desde por ejemplo el flujo de entrada en una autovía hasta el consumo de carburante de un vehículo en ciudad. Lo más común es clasificar los simuladores dentro de dos grandes grupos, los cuales se ilustran en la Figura~\ref{fig:granularities-in-traffic-simulation}:

\begin{itemize}
	\item \textbf{Microsimulación} o simulación de tipo \textbf{micro}. Su objetivo es estudiar, desde un punto de vista de granularidad fina como puede ser vehículos o peatones, las micropropiedades del flujo de tráfico como, por ejemplo, los cambios de carril, las aproximaciones a vehículos delanteros o los adelantamientos, para evaluar su comportamiento. Tiene dos principales ventajas, la posibilidad de estudiar el tráfico como un todo a partir de sus elementos más simples (ofreciendo una representación más fiel de éste) y la posibilidad de estudiar cada elemento por separado. Sin embargo, la principal desventaja de este tipo de modelos es que cada elemento de la simulación requiere de cómputo independiente y por tanto simulaciones con alto contenido de elementos pueden llegar a ser inviables\sidenote{Existen técnicas de computación distribuida que superan ampliamente los límites impuestos por la computación en un único nodo, por ejemplo, el simulador de IBM \textit{Megaffic}. Éste implementa un modelo de granularidad micro donde cada elemento es un agente independiente (i.e. sistema multiagente) usando para ello entornos con cientos de núcleos de proceso que proveen de capacidad suficiente para modelar ciudades enteras como Tokio (ver~\cite{Osogami2012} y~\cite{Suzumura2012}).}.
	\item \textbf{Macrosimulación} o simulación de tipo \textbf{macro}. Este tipo de modelos centran su esfuerzo en estudiar el flujo de tráfico como un todo, explorando sus macropropiedades (e.g. evolución del tráfico, efectos onda, velocidad media o flujo en vías). Su ventaja principal es que a nivel macroscópico permiten estudiar propiedades que a nivel microscópico requeriría una cantidad ingente de recursos. Sin embargo, con este modelo es imposible obtener información precisa de un elemento en particular del tráfico.
\end{itemize}

Aunque esta es la categorización típica de modelos, en la literatura aparecen otros tipos de modelo con granularidades que pueden considerarse no pertenecientes a ninguno de estos dos conjuntos. Este es el caso de los simuladores de tipos \textbf{sub-micro} y los \textbf{meso} (ver figura~\ref{fig:mesoscopic-and-submicroscopic-simulation}). 

\begin{figure}
	\centering
	\includegraphics{images/mesoscopic-and-submicroscopic-simulation}
	\caption{Aproximaciones alternativas de modelos en función de la complejidad. Ejemplo de mesosimulación como ventana de microsimulación dentro de un flujo en un macrosimulador (e.g. \cite{munoz2001integrated}) y ejemplo de submicrosimulación donde se modelan componentes internos del vehículo.}
	\label{fig:mesoscopic-and-submicroscopic-simulation}
\end{figure}

Los \textbf{sub-micromodelos} especifican granularidades por debajo del nivel de \enquote{vehículo} o \enquote{peatón}. Por ejemplo, en (\cite{Minderhoud1999}) trabaja a nivel de funcionamiento del control de crucero inteligente de un vehículo en función del entorno del vehículo.

Por otro lado los \textbf{mesosimuladores} (e.g.~\cite{munoz2001integrated} o~\cite{casas2011need}) nacen para amortiguar los problemas inherentes a la complejidad en los micromodelos y a la falta de resolución en los macromodelos.

Dado que en nuestro discurso trabajaremos en la evaluación de modelos de comportamiento de conductores, nos ceñiremos al uso de simuladores que modelen un nivel de granularidad \textbf{micro}.

\subsection{Tipos de simulador en función del espacio y el tiempo}

Existen otras dos dimensiones que generan dos agrupaciones cada una dependiendo de cómo evolucionan los factores espacio y tiempo a lo largo de la simulación. Sin embargo, el tiop de simulador en funcińo de la complejidad (i.e. \textit{micro} vs. \textit{macro}) determina en gran medida la evolución de estos factores.

En el caso del tiempo, si éste transcurre en forma de intervalos variables pero discretos se habla de \textbf{simulación de tiempo discreto} o de \textit{eventos discretos}. Si por el contrario el tiempo es un factor más de un modelo de ecuaciones, generalmente diferenciales, estamos hablando de una \textbf{simulación de tiempo continuo}. En general, las simulaciones de tiempo continuo trabajan sobre macrosimulación, por lo que es de menor interés para nosotros que los simuladores de tiempo discreto, donde se cuantifica el tiempo de la simulación (e.g. pasos a una frecuencia de $1Hz$ o $10Hz$).

En el caso del espacio, la clasificación es similar. Si la simulación se mueve por un espacio discreto, hablamos de una \textbf{simulación de espacio discreto}, y en caso de que el espacio sea continuo, de \textbf{simulación de espacio continuo}. En este caso, para nuestro estudio consideramos que los discretos pierden demasiado detalle. Dado un instante $t$, independientemente de si el simulador es de tiempo discreto o continuo, nos es más interesante conocer la situación exacta del vehículo y no una situación aproximada en una separación discreta del espacio de al simulación.

\subsection{Modelos en microsimulación}

Dentro de la microsimulación, he visto que hay diferentes aproximaciones. Autómatas celulares, sistemas de partículas, sistemas multiagentes. Buscarmás y describir aquí. El caso es que por lo que veo, de esostres sólo nos interesan los mas. Explicarlos aquí.


\section{Elección de software para las simulaciones}

En este apartado se facilita la comparativa realizada para la elección de simulador sobre el que basar los escenarios a plantear en las simulaciones de los modelos de conductor.

Hoy en día existe una oferta muy amplia de simuladores en el mercado, cada uno implementando uno o varios modelos diferentes y bajo diferentes licencias.

Cada simulador tiene sus ventajas e inconvenientes, y es por ello importante realizar un estudio previo para conocerlos y no llevarse sorpresas una vez se llega a estadios más avanzados del estudio. No se trata de realizar una comparativa en busca del mejor simlador de tráfico del mercado, sino en encontrar el simulador que más se adecúa a los criterios concretos para los propósitos de esta tesis.

\subsection{Entornos de simulación a estudiar}

El siguiente listado muestra la lista de simuladores de tráfico sobre los que se realizarán las comparativas. Por motivos de espacio no se han incluido todos los simuladores encontrados en la listeratura, sino que se han seleccionado únicamente aquellos que (i) aun existen y se pueden adquirir, y (ii) son entornos de microsimulación.

\begin{enumerate}
	\item \textbf{AIMSUN}. Entorno de simulación de granularidad micro, meso y macro desarrollado por la empresa \textit{Transport Simulation Systems}. Url: \url{http://www.aimsun.com/}.
	\item \textbf{TSIS-CORSIM}. Entorno de microsimulación compuesto de dos simuladores para distintos modos de tráfico (NETSIM para entornos urbanos y FRESIM para entornos interurbanos) desarrollado dentro de la Universidad de Florida por el centro \textit{McTrans}. Url: \url{http://mctrans.ce.ufl.edu/featured/tsis/}.
\end{enumerate}

Simuladores de pago:

Quadstone paramics (microscopic)
VISSUM (macroscopic)
VISSIM (microscopic)
ARCHISIM

Simuladores gratuitos:

Matsim
SUMO (microscopic)
Repast
MAINSIM
Synchro

Ni puta idea:

CUBE
SATURN
PARAMICS
TRANSIMS


\subsection{Criterios de selección}

Los criterios se muestran ordenados alfabéticamente:

\begin{enumerate}
	\item \textbf{Activo}. Si el simulador está activaemnte desarrollado o si por el contrario se trata de un proyecto con poca actividad por parte de sus autores. Es interesante hacer uso de un simulador que esté siendo activamente desarrollado porque eso favorece la aparición de parches y mejoras sobre el software.
	\item \textbf{Extensibilidad}. Si el simulador permite extender sus funcionalidades de alguna manera. Aunque se puede considerar que si es Open Software, es posible modificar su comportamiento para adcuarlo a los modelos desarrollados, es mejor que el propio software ofrezca los mecanismos necesarios para la integración sin necesidad de tocar el núcleo.
	\item \textbf{Granularidad}. Si el simulador es de tipo micro, meso o macro. Para nuestras necesidades es necesario un simulador que implemente microsimulación, ya que es el único tipo de granularidad que permite evaluar el comportamiento de un conductor independientemente del resto de la simulación.
	\item \textbf{Licencia}. Especifica con qué tipo de licencia se distribuye el software. Es preferible una licencia de tipo Open Software (\TODO hay que ver si esto está bien dicho o no) ya que de esta manera es posible modificar el software en caso de encontrar algún error o falta de funcionalidad que el fabricante no tenga pensado codificar.
	\item \textbf{Sistema operativo}. Sobre qué sistemas operativos está soportado el entorno de simulación. Es imprescindible que el software se ejecute sobre sistemas operativos GNU/Linux por la configuración de los sistemas sobre los que se trabaja, aunque es interesante también su funcionamiento en entornos tipo OS-X.
	\item \textbf{Tipo de simulación}. Qué modelo interno usa el motor para la simulación (e.g. automatas celurares, sistemas multiagentes, ...).
	particle system simulation).
\end{enumerate}

\subsection{Comparativa}

Al haber tal cantidad de simuladores, la comparativa se ha realizado descartando aquellos simuladores que no cuentan con características necesarias o que son de una tipología no deseada. A continuación se enumera la lista de razones por las que se han descartado simuladores junto con aquellos afectados por la decisión:

\begin{enumerate}
	\item Debe ser (o al menos soportar) un entorno de microsimulación.
	\item Debe ofrecer un entorno de simulación de tráfico general, no sólo casos particulares como congestión o colisiones.
	\item \ldots
\end{enumerate}

\begin{center}
	\footnotesize
	\begin{tabular}{lllll}
		\toprule
		& Aimsun & \acrshort{sumo} & TSIS-CORSIM & \\
		\midrule
		Activo & sí & sí & sí & \na \\
		\addlinespace
		Extensibilidad & \na & sí & no & \na \\
		\addlinespace
		Granularidad & & & & \\
		\quad Micro          & sí     & sí             & \na  & \na \\
		\quad Meso           & sí     & no             & \na  & \na \\
		\quad Macro          & sí     & no             & \na  & \na \\
		\addlinespace
		Licencia & & & & \\
		\quad Propietaria    & sí     & sí             & \na  & \na \\
		\quad Open Software  & no     & no             & \na  & \na \\
		\quad Compatible GPL & no     & no             & \na  & \na \\
		\addlinespace
		Sistema operativo & & & & \\
		\quad GNU/Linux      & sí     & no             & \na  & \na \\
		\quad OS X           & sí     & no             & \na  & \na \\
		\quad Windows        & sí     & sí             & \na  & \na \\
		\addlinespace
		Tipo de simulación   & \na    & \acrshort{mas} & italics & upright, caps \\
		\bottomrule
	\end{tabular}
\end{center}

\subsection{Entorno seleccionado: \acrshort{sumo}}

En definitiva, el simulador que más se adapta a nuestras necesidades y el que se usará como simulador base en el desarrollo de esta tesis será \gls{sumo}\sidenote{Sus principales publicaciones son~\cite{krajzewicz2002sumo}, \cite{behrisch2011sumo} y \cite{krajzewicz2012recent}.}. \gls{sumo} es un entorno de microsimulación de código abierto\sidenote{Licenciado bajo la \gls{gpl}, concretamente la versión $3.0$.} que implementa un modelo discreto en el tiempo y continuo en el espacio.

Además de simulación clásica, \gls{sumo} provee de una interfaz gráfica (se puede ver un pantallazo en la figura~\ref{fig:sumo-simulator}) donde se puede ver el comportamiento de cada vehículo durante la simulación. Es interesante para obtener de un vistazo información acerca del funcionamiento del modelo en concreto a controlar. Otras de las características que el simulador ofrece son las siguientes:

\begin{figure}
	\includegraphics{sumo-simulator}
	\caption{Ejemplo de pantalla del simulador \gls{sumo}. Además de entorno de simulación propiamente dicho, \gls{sumo} provee de una interfaz gráfica que permite una visualización general, de zonas y de elementos en concreto a la vez que permite la variación de configuración de la simulación durante el desarrollo de la misma.}
	\label{fig:sumo-simulator}
\end{figure}

\begin{itemize}
	\item Multimodalidad permitiendo modelar no sólo tráfico de vehículos sino de peatones, bicicletas, trenes e incluso de barcos.
	\item Vehículos de diferentes tipologías, Simulación con y sin colisiones de vehículos.
	\item Diferentes tipos de vehículos y de carreteras, cada una con diferentes carriles y éstas con diferentes subdivisiones de subcarriles (diseño conceptual para permitir las simulaciones )
\end{itemize}

Al estar licenciado bajo la licencia \gls{gpl}, su distribución implica a su vez la distribución de su código fuente. Esto permite la modeificación de su comportamiento y el desarrollo de nuevos modelos integrados dentro del simulador. Sin embargo nosotros no haremos uso de esta característica, sino que usaremos \gls{sumo} como aplicación servidor y el módulo \gls{traci} como aplicación cliente desde donde gestionar todos los aspectos de cada simulación.

\subsection{La interfaz \glsentrylong{traci}}
\chapter{Modelos de comportamiento}
\label{ch:sota-behavior-models}

El objetivo que persigue la simulación de tráfico es hacer cada vez más realistas los modelos generados. En un simulador de tipo \glsentrylongsp{mas} donde cada uno de los agentes modela, entre otros, a conductores, el realismo aumenta cuanto más se parecen el comportamiento\sidenote{
	¿A qué nos referimos cuando hablamos del \textit{comportamiento al volante}? \textit{comportar} (segun la RAE) se define como \textit{actuar de una manera determinada}. Por tanto, comportamiento al volante lo tratamos en este estudio como las acciones o tareas que se ejecutan durante la conducción, es decir, la manera de actuar de un individuo mientras conduce.
} del agente al del conductor real.

Conducir implica la ejecución de múltiples tareas en paralelo, cada una de ellas pertenecientes a un nivel cognitivo. Además, las acciones no están limitadas a la interacción con el vehículo; el conductor ha de tener en cuenta otros factores como, por ejemplo las señales, los peatones o los \glspl{adas}.

\begin{figure}
	\centering
	\begin{tikzpicture}
	\tikzset{
		centered/.style = { align=center, anchor=center },
		arrow/.style = { black!20, arrows={-Triangle} },
		dblarrow/.style = { black!20, arrows={Triangle-Triangle} },
	}
	
	\matrix (m) [
			matrix of nodes,
			column sep      = 2em,
			row sep         = 1em,
			column 1/.style = { nodes = { font=\sffamily\scriptsize, fill=black!20, centered}},
			column 2/.style = { nodes = { font=\sffamily, centered, fill=orange!20, text width=3cm }},
			column 3/.style = { nodes = { font=\sffamily, centered, text width=2.2cm }},
			column 4/.style = { nodes = { font=\sffamily, centered }},
		] {
			& Nivel estratégico & planes generales & $\gg s.$\\
			Entorno & Nivel táctico  & patrones de acción & $s.$ \\
			Entorno  & Nivel de control & comportamientos automáticos & $ms.$ \\
		};
		\draw[arrow] (m-1-2) -> (m-1-3);
		\draw[arrow] (m-2-1) -> (m-2-2);
		\draw[dblarrow] (m-2-2) -> (m-2-3);
		\draw[arrow] (m-3-1) -> (m-3-2);
		\draw[dblarrow] (m-3-2) -> (m-3-3);
		\draw[dblarrow] (m-2-2) -> (m-1-2);
		\draw[dblarrow] (m-3-2) -> (m-2-2);
	\end{tikzpicture}
	\caption{Los tres niveles jerárquicos que describen la tarea de conducción según~\cite{michon1985critical}: \textit{estrategia} (i.e. las decisiones generales), la \textit{maniobra} (i.e. decisiones durante la conducción de más corto plazo) y \textit{control} (i.e. automatismos).}
	\label{fig:three-levels-of-human-driving}
\end{figure}

\cite{michon1985critical} divide en tres los niveles de abstracción de las tareas: el de \textbf{control}, que se ocupa de las tareas de más bajo nivel destinadas éstas a mantener la conducción como son la aceleración o los cambios de marcha, el de \textbf{maniobra} o táctico, donde sus tareas son las encargadas de mantener la interacción con el entorno como los cambios de carril o el control de las señales y demás estímulos externos, y el \textbf{estratégico}, que engloba las tareas de más alto nivel como el razonamiento y la planificación de rutas (ver figura~\ref{fig:three-levels-of-human-driving})\sidenote{Algunos estudios llegan incluso a definir intervalos temporales de tiempo de razonamiento para las tareas de cada nivel. Por ejemplo, en \cite{Alexiadis2004} se establecen los siguientes tiempos: alrededor de $30$ segundos para las tareas del nivel de planificación, de $5$ a $30$ segundos para las tareas de nivel táctico y por debajo de los $5$ segundos para las tareas de control.}.

El comportamiento de un conductor al volante tiene una relación directa con el nivel de abstracción de maniobra o táctico. Ésta se puede concebir como la encargada de planificar acciones a corto plazo para conseguir objetivos a corto plazo. Las tareas de control son automáticas e influyen poco o nada en las tomas de decisión relacionadas con tareas del estilo de cuánto acelerar en esta situación o cuándo cambiar de carril. Las tareas estratégicas estan a un nivel más alto de abstracción (e.g. la ruta a seguir hasta mi destino) y tampoco afectan demasiado al comportamiento en situaciones concretas\sidenote{No obstante algunos trabajos han demostrado que en ocasiones la planificación de la ruta sí afecta a decisiones normalmente asociadas el nivel táctico como por ejemplo la preferencia de un conductor por uno u otro carril de la vía \cite{Wei2000, Toledo2003}.}.

El resto del capítulo introducirá los modelos de comportamiento más conocidos y hará especial hincapié en el estado más reciente de modelos basados en técnicas de la \glsentrylong{ci}\sidenote{
	Otros usos de las técnicas de la \gls{ci} como la caracterización de los conductores no serán explorados. Algunos estudios de interés son \cite{sekizawa2007modeling, terada2010multi} donde se hace uso de regresión parcial sobre datos extraídos de simuladores para la caracterización del comportamiento y \cite{DiazAlvarez2014} donde se aplican redes neuronales a datos de conducción reales para la predicción del consumo y la identificación de comportamientos agresivos.
}.

En~\cite{sekizawa2007modeling} describen modelos supervisados offline para capturar el comportamiento del conductor basados en auto-regresión a trozos. Más adelante lo extienden en~\cite{terada2010multi}, aunque los datos de entrenamiento son extraídos directamente de simulaciones, no del \enquote{mundo real\textsuperscript{TM}}.

\section{Tipos de maniobra}

En el nivel táctico del comportamiento, las tareas que se realizan durante la conducción son las relacionadas a circular con el vehículo dentro del flujo de tráfico, sin bajar demasiado de detalle. En la literatura, estas tareas se centran en dos clases generales de problema diferentes (figura \ref{fig:behavior-model-classification}): el de \textbf{aceleración}, que se ocupa de modelar el comportamiento de un conductor en el carril en el que se encuentra y el de \textbf{cambio de carril}, encargado de decidir y ejecutar los cambios de carril.

\begin{figure}
	\centering
	\begin{tikzpicture}
	\tikzset{every concept/.style={minimum size=2cm, text width=2cm}}
	\path[mindmap,concept color=MidnightBlue, text=white]
	node[concept] {modelos} [clockwise from=300]
	child[level distance=100, concept color=RoyalBlue] {
		node[concept] {aceleración} [clockwise from=30]
		child[concept color=Peach] {
			node[concept] {free-flow}
		}
		child[concept color=Peach] {
			node[concept] {car-following}
		}
	}
	child[level distance=100, concept color=RoyalBlue] {
		node[concept] {lane-changing} [clockwise from=210]
		child[concept color=Peach] {
			node[concept] {lane-selection}
		}
		child[concept color=Peach] {
			node[concept] {gap-acceptance}
		}
	};
	\end{tikzpicture}
	\caption{Las diferentes tareas para modelar el comportamiento de un conductor al volante. Están clasificadas en dos tipos, de aceleración, encargadas de definir cómo acelera el conductor en casos con y sin vehículo delantero y de lane-changing, encargadas de las tareas relacionadas con el cambio de carril.\TODO{Volver a echar un ojo a modelos de aceleración porque lo he cambiado.}}
	\label{fig:behavior-model-classification}
\end{figure}

Dentro de éstas, los autores en función del alcance y el objetivo del estudio identifican diferentes clases de sub-problemas. Algunos de éstos pueden ser selección decarril, cambio de carril, adelantamiento, adaptación a velocidad de vehículo frontal, etcétera (\cite{Aycin1999}). Mencionaremos estos tipos de problema a lo largo de la sección.

\subsection{Modelos de aceleración}

Se ocupan de gestionar el comportamiento del conductor sobre la aceleración (positiva o negativa) en un entorno lineal como lo es un carril de tráfico.

Las primeras menciones sobre modelos de aceleración se atribuyen a Reuschel (1950) y \cite{Pipes1953} por sus trabajos sobre el concepto de \textit{\idx{car-following}}.

\newthought{Un vehículo está en una situación de \idx{car-following}} cuando su velocidad está condicionado por el vehículo que se encuentra frente a él. El primer trabajo concreto es el de \cite{Pipes1953}, en el que el comportamiento responde a tratar de mantener un espacio variable en función de la velocidad\sidenote{En este modelo en concreto, el espacio viene determinado a partir de la ecuación de la velocidad cuando el tiempo no baja de $1,02$ segundos.}. Este modelo se puede considerar de una clase que denominaremos \textit{mantenimiento de medida} dado que su objetivo es manmtener constantemente una distancia segura. Otros trabajos trabajan con el mantenimiento de otras medidas como distancia relativa al parachoque delantero o trasero.

Posteriormente, en $1958$, se presentó el modelo GM\sidenote{Debido a que el desarrollo del modelo se realizó dentro de la \textit{General Motors Corporation}.} (\cite{Chandler1958}), el cual sirvió como base para el desarrollo de numerosos modelos posteriores. Este modelo se caracteriza por el uso del concepto \textit{estímulo $\leftarrow$ respuesta}, donde la acción (respuesta) del vehículo es debida a la activación de un estímulo tras pasar un tiempo de retardo  $\tau$. En el modelo de \cite{Chandler1958} la respuesta es el cambio de tasa de aceleración en función de la variación de la distancia al vehículo delantero. Algunas modificaciones sobre el algoritmo original son, entre otras, la asimetría en la tasa de cambio de aceleración y deceleración o la inclusión de segundos coches delanteros (\cite{Gazis1959}, \cite{Bexelius1968}).

\begin{figure}
	\missingfigure[figheight=4cm]{Una clasificación entre los tres tipos identificados de car-following: mantenimiento de medidas, estímulo-respuesta y psicofísicos}
	\caption{Los tres tipos generales de modelo de \textit{car-following}: basados en mantenimiento de medidas, basados en estímulo-respuesta y psicofísicos.}
	\label{fig:car-following-there-different-models}
\end{figure}

Los métodos de estas dos clases de modelo suelen ser sencillos de implementar, pero tienen un problema principal: suponen que el conductor es capaz de percibir todo cambio, incluso el más mínimo, en el coche delantero cuando esto en realidad no es así. En $1974$ apareció una nueva clase de modelos de \textit{\idx{car-following}}, denominados posteriormente como \textbf{psicofísicos} (\cite{wiedemann1974simulation} and \cite{Leutzbach1988}) donde se introduce el concepto de \textit{umbral perceptual}\sidenote{El \textit{umbral perceptual} de una medida es el límite a partir del cual se percibe un cambio en dicha medida. Mediante el uso de umbrales perceptuales, se limitan las acciones de los coches a cambios perceptibles en los coches delanteros.} como medida para superar la limitación de los otros dos tipos de modelo.

Sucesivos trabajos sobre el concepto de \textit{umbral perceptual} y los modelos psico-físicos llevaron a conclusión de que el \textit{\idx{car-following}} no era más que una clase más de un conjunto más amplio de modelos de aceleración (ver imagen~\ref{fig:acceleration-model-classes}). En \cite{wiedemann1992microscopic} se proponen hasta cuatro clases diferentes de modelos de aceleración en función de las posiciones y velocidades relativas entre el vehículo sujeto y el siguiente: \textit{free-flow}, donde el vehículo no se ve afectado por el comportamiento del siguiente vehículo y se basa en velocidad que quiere alcanzar sin impedimentos (más allá de los impuestos por el tipo y condición de la vía), \textit{car-following} cuando el comportamiento del vehículo se ve influenciado por el vehículo delantero, obligando a disminuir la velocidad deseada en el conductor en cuestión, \textit{approaching} como situación intermedia entre las dos anteriores y \textit{emergency} cuando la situación es crítica (e.g. colisión inminente). Otros autores posteriormente (e.g. \cite{Toledo2003} o \cite{Liu2013}) diferencian otras situaciones como el \textit{close-following} o \textit{stop-and-go}

\begin{figure}
	\missingfigure[figheight=4cm]{Diferentes clases de modelos de aceleración}
	\caption{El modelo \textit{car-following} sólo es uno entre muchas clases de modelos de aceleración.}
	\label{fig:acceleration-model-classes}
\end{figure}

\newthought{Agrupar en un mismo modelo} las diferentes clases de modelos de aceleración es un trabajo que se empezó a desarrollar a partir de los años $1990$. Los principales problemas son la complejidad de estos modelos, ya que aumentar las clases implica la generalización de multitud de factores. Algunos trabajos a este respecto son el modelo de Gipps (\cite{Gipps1981}) el cual agrupa las clases \textit{\idx{free-flow}} y \textit{\idx{car-following}}, el modelo de Yang et. al (\cite{Yang1996}) que agrupa las de \textit{emergency}, \textit{car-following} y \textit{free-flow} o el modelo Optimal Velocity (\cite{Bando1998}) que agrupa las de \textit{\idx{free-flow}}, \textit{\idx{car-following}} y \textit{stop-and-go}.

Una característica de todos los modelos hasta el momento es que no capturan los comportamientos de los diferentes tipos de conductor o vehículo. Sin embargo, el comportamiento puede variar en función de los comportamientos de los vehículos que se encuentran en su entorno (\cite{Tordeux2010}). Algunos estudios recientes como los de Simonelli et al. (2009), Colombaroni and Fusco, 2013 o Zheng et al., 2013 se puede decir que incorporan el comportamiento del entorno, pero únicamente porque entrenan redes neuronales con información tanto del coche como del entorno y por tanto éstas pueden haber aprendido detalles de éste. Más adelante se hablará de los modelos basados en técniacs de la glsentrylong{ci}.

...

SUMO usa (al menos así lo indican en el paper del 2002) el modelo Gipps\cite{krajzewicz2002sumo}. No sé si ellos han hecho una extensión del modelo o están referenciando la extensión y ellos sólo la implementan. En el paper del 2012 citan que el modelo car-following que usan por defecto es el desarrollado por Stefan Krauß\cite{jin2016evaluation}, debido a su simplicidad y su velocidad de ejecución. El modelo ha probado ser válido, pero tiene algunos defectos, por lo que existe un API para implementar otros modelos. En la actualidad están incluidos en el sistema los modelos IDM\cite{treiber2000congested} (\textit{Intelligent Driver Model}), el modelo de tres fases de Kerner\cite{kerner2008testbed} y el modelo de Wiedemann\cite{wiedemann1974simulation}.

(Barcelo et al., 1996, PETRI: A parallel environment for a real-time traffic management and information system) describen el simulador AIMSUN. El comportamiento de cada vehículo en la simulación es modelado a través de múltiples modelos de comportamiento (e.g. car following, lane changing, gap acceptance). El modelo de cambio de carril es el usado en el modelo de Gipps, aunque el propio simulador permite la modelización de incidentes por lo que debe existir alguna variación del modelo original o un nuevo modelo para ese caso concreto.

\subsection{Modelos de cambio de carril}

El tráfico real no está compuesto por un sólo carril, sino por varios. El cambio de carril ocurre cuando un vehículo se mueve del carril que está usando a un carril destino, ya sea porque quiere mejorar su circulación (e.g. quiere realizar un adelantamiento) o porque su ruta lo requiere (e.g. está próxima la rampa de salida que quiere tomar en una autopista).

Este comportamiento mejora la velocidad media del flujo de la vía en situaciones de poca carga de tráfico, aunque pueden afectar al tráfico en formas de ondas de choque (\cite{Sasoh2002}, \cite{Jin2006}) e interferir en éste. En situaciones de carga de tráfico media-alta o de congestión pueden llegar a interferir en el tráfico incluso más aun que los modelos de \textit{\idx{car-following}} (\cite{Laval2006}).

\begin{figure}
	\missingfigure[figheight=4cm]{Dos ilustraciones, una con la seleccción de cambio de carril y otra con el cambio.}
	\caption{El cambio de carril se divide tradicionalmente en una operación que involucra dos pasos. La selección de carril (\textit{lane-selection}) al que cambiarse y la ejecución del cambio (\textit{merging}).}
	\label{fig:lane-selection-plus-merging}
\end{figure}

Los modelos de cambio de carril o \textit{\idx{lane-change}} se ocupan de determinar cuándo un vehículo quiere desplazarse de un carril a otro (denominado \textbf{\idx{lane-selection}}) y de la ejecución del mismo (denominado \textbf{merging}). Esta división fue introducida en \cite{Sparmann1978}, donde el autor además distinguía entre cambios hacia la izquierda (motivados por obstrucciones como por ejemplo coches lentos) y hacia la derecha (motivados por, por ejemplo, no obstrucciones), estando determinada la ejecución del cambio por el espacio en el carril objetivo.

En \cite{Gipps1986} se introduce el concepto de cambio de carril \textbf{obligatorio}, ejecutado cuando es obligatorio abandonar el carril actual o acceder al carril objetivo y \textbf{discrecional} cuando el cambio se ve motivado para mejorar la situación actual de conducción (ver figura~\ref{fig:lane-change-mandatory-vs-discretional}). Este trabajo propone un framework para el problema de cambio de carril al aproximarse a un cambio de dirección. El modelo identifica tres distancias que caracterizan el comportamiento del conductor en función de cómo de lejos está dicho punto: (i) textbf{lejos}, en el que no existe condicionamiento en la decisión de cambio de carril, (ii) textbf{medio}, donde el conductor empieza a ignorar los cambio que dan ventaja de velocidad si no hacia carriles distanciados del de salida y (iii) \textbf{cerca} donde los vahículos deben estar en el carril de cambio de salida. El modelo tiene los problemas de que los factores son evaluados secuencialmente (por lo que no se evalúan factores más bajos en la jerarquía si no es necesario) y de que supone que el cambio de carril ocurre sin forzar a los vehículos del carril de destino a modificar su comportamiento como disminuir la velocidad o parar. El modelo de \cite{Hidas2002} advierte de esta situación indicando que el cambio en situaciones de congestión ha de ser o bien forzado o bien a través de colaboración.

\begin{figure}
	\missingfigure[figheight=4cm]{Dos ilustraciones, una con un cambio de carril obligatorio (que se acabe el carril por ejemplo) y otro discrecional (que se vea un vehículo lento como un camión y el coche adelantando).}
	\caption{Los cambios de carril se clasifican en la literatura como aquellos necesarios para continuar con la conducción u \textbf{obligatorios} y aquellos útiles para mejorar la situación de conducción o \textbf{discrecionales}.}
	\label{fig:lane-change-mandatory-vs-discretional}
\end{figure}

En \cite{wiedemann1992microscopic} se desarrolla un framework que tiene en cuenta los cambios de carril lento a rápido (debido, por ejemplo a alguna obstrucción como accidente o un vehículo lento) y de rápido a lento (por ejemplo debido a las condiciones de la ruta). Ya que el modelo está afectado por la influencia del entorno, divide esta en dos tipos: (i) actual, las características de los vehículos de alrededor y (ii) potencial, la estimación de las características del entorno en momentos posteriores al actual.

En \cite{Hidas2002} se presenta un modelo similar al de \cite{Gipps1986} apoyándose también en los dos tipos de cambio de carril, \textbf{obligatorio} y \textbf{discrecional}. Se presentan una serie de factores (también evaluados en secuencia) que provocan situaciones de cambio de carril. Cuando las situaciones son del tipo de mejora de la situación de conducción, se dispara un cambio discrecional, pero cualquier situación que dispara un cambio obligatorio descarta toda decisión de cambio discrecional por considerarse prioritario.

Más trabajos en la línea de los anteriores son \cite{Halati1997, Yang1996, Ahmed1999}, los cuales fijan la separación jerárquica entre cambios obligatorios y cambios discrecionales, o \cite{Toledo2003, Wei2000} donde salvan dicha limitación.

\cite{Ahmed1999} propone un modelo probabilístico que divide la secuencia de cambio de carril en tres fases: decisión de camboi, selección de carril y ejecución (i.e. aceptación del hueco y merge). Además, a las clases de cambio de carril \textbf{obligatorio} y \textbf{discrecional} definidas en \cite{Gipps1986} añade una nueva, \textbf{forzado}, definida para situaciones de mucha congestión de tráfico, cuando se abre un hueco suficiente para ejecutar el cambio. Se usa \gls{mitsim} como plataforma en la que realizar los tests del modelo.

En \cite{Toledo2003, Toledo2009} Plantean un modelo en el que, manteniendo la clasificación de los cambios de carril en obligatorio y discrecional, separan el adelantamiento en selección de carril y gap-acceptance. El modelo es probabilístico y clasifica las entradas en cuatro grupos de variables: las de vecindad (e.g. huecos y velocidades), las de planificación de ruta (e.g. distancia a la salida objetivo), las relacionadas con la experiencia (e.g. evitar un determinado carril en un determinado tramo) y las de estilo de conducción.

Tanto \cite{Ahmed1999} como \cite{Toledo2003, Toledo2009} prueban sus modelos en \gls{mitsim}


\newthought{La viabilidad en un cambio de carril} se determina haciendo uso de modelos denominados \textit{\idx{gap acceptance}}, donde los vehículos calculan si caben o no en un determinado huevo. Estos modelos fueron creados inicialmente para situaciones emplazadas en intersecciones aunque posteriormente se comprobó su utilidad para determinar si es posible o no el cambio a un carril basándose principalmente en el espacio hueco existente en el carril destino.

\begin{equation}
f_{g_l}(t) = \twopartf {0} {g_l(t) < g^{crit}_l(t)} {1} {g_l(t) \geq g^{crit}_l(t)}
\label{eq:gap-acceptance-model}
\end{equation}

En la ecuación~\ref{eq:gap-acceptance-model} se describe el modelo típico de un modelo de \idx{gap accceptance}. En un momento $t$, el cambio a un carril $l$ es viable ($f_{g_l}(t) = 1$) o no ($f_{g_l}(t) = 0$) dependiendo de si el espacio en el carril destino $g_l(t)$ es mayor o menor que un \enquote{hueco crítico} (en inglés \idx{critical gap}) $g^{crit}_l(t)$. Diferentes autores determinan el hueco crítico basándose en diferentes parámetros (e.g. \cite{Miller1972} o \cite{Cassidy1995})

En \cite{Gipps1986} y \cite{Ahmed1996} dividen sin embargo el hueco crítico en dos, hasta el vehículo delantero y hasta el vehículo trasero. Ambos deben ser aceptables y en dichos huecos influye además la velocidad relativa respecto a los vehículos delantero y trasero. Otras variaciones a la hora de determinar el hueco crítico incluyen la variación del tamaño en función de si es obligatorio o discrecional (\cite{Toledo2003}), velocidades relativas (\cite{Ahmed1999}) o cooperación entre conductor realizando el cambio y conductores en carril destino (\cite{Ahmed1999}, \cite{Hidas2002}).

...


\newthought{Un cambio de carril no tiene por qué involucrar al conductor que lo ejecuta}....

...

En \cite{Fritzsche1994} se describe un modelo de microsimulación para analizar cuellos de botella (e.g. un accidente donde se bloquea uno de los carriles). Es un caso típico donde los vehículos no pueden cambiar de carril sin la participación activa del resto de vehículos (colaboración). El modelo lo describe de una maera muy sucinta y no considera comportamiento colaborativo en el cambio de carril.

En (Yang and Koutsopoulos, 1996, A Microscopic Traffic Simulator for Evaluation of Dynamic Traffic Management Systems) presentan el simulador MITSIM, desarrollado por el MIT (creo) en el que se habla específicamente de comportamiento colaborativo en cambio de carriles haciendo uso de lo que denominan "courtesy yielding function" (algo así como función de cesión de paso de cortesía), la cual se usa para para hacer espacio al vehículo que va a incorporarse al carril. Sin embargo, los detalles de dicho proceso no están especificados en el paper.

...

Sparmann (\cite{Sparmann1978}) propone modelo psico-físico a las velocidades y distancias relativas.

\subsection{Modelos mixtos}

Los modelos de aceleración y de cambio de carril han sido tratados tradicionalmente como modelos independientes, siendo estos primeros mucho más estudiados que los segundos\sidenote{Este hecho es motivado por la dificultad en la captura de datos en los cambios de carril y, por tanto, por su escasez.}.

\begin{figure}
	\includegraphics{toledo-2007-behavior-model-tree}
	\caption{Estructura del modelo de comportamiento de los vehículos en \cite{Toledo2007}. El agente comprueba constantemente se desea o no cambiar de carril, y una vez decidido comprueba la viabilidad. En cualquier caso, toda decisión finaliza con la comprobación de la aceleración. Fuente: \cite{Toledo2007}.\TODO{Quizá la imagen es muy grande y quedaría mejor si la hiciésemos más pequeña.}}
	\label{fig:toledo-2007-behavior-model-tree}
\end{figure}

Sin embargo, a partir de los años $90$ se ha tendido al desarrollo de modelos de comportamiento que combinan los modelos de aceleración y de cambio de carril (\cite{Ma2004}). Un ejemplo actual de este tipo de modelos es el descrito en \cite{Toledo2007}, el cual se basa en el concepto de \enquote{objetivo a corto plazo} para elaborar un \enquote{plan a corto plazo} apoyándose en un árbol de decisión que determina la accinó a realizar (ver figura~\ref{fig:toledo-2007-behavior-model-tree}).

...

\section{La \glsentrylong{ci} en los modelos de conducción}

Hasta ahora, los modelos que hemos explorado en la descripción de maniobras pertenencen al dominio del  \glsentrylong{hc}, es decir, están basados en fórmulas matemáticas y reglas de la lógica convencional con parámetros que se ajustan a partir de la observación de datos reales.

Sin embargo, desde mediados de los años $90$ empezó a crecer el interés por las técnicas de la \gls{ci} debido, entre otras a las siguientes razones:

\begin{itemize}
	\item A mediados del interés volvió el interés de la \gls{ai} debido a los éxistos cosechados por las técnicas de la \gls{ci}, y por tanto se comenzaron a usar en todas las áreas, incluída la de las its.
	\item El rápido desarrollo de la tecnología ha hecho posible la existencia de conjuntos de datos masivos (i.e. mayor capacidad de captura y almacenamiento), con más calidad (e.g. sensores más precisos) y más cantidad de fuentes (e.g. GPS, acelerómetros, giroscopios, \ldots). Esto no sólo permite el ajuste de los modelos existentes o las pequeñas modificaciones, sino que son una fuente de datos muy interesante para técnicas de \glsentrylongsp{ml}, rama que pertenece a la \gls{ci}.
\end{itemize}

Algunos autores incluso sugieren que el futuro de las \gls{its} pasa por cambiar del paradigma hacia técnicas basadas en \gls{ml}. Es decir, pasar del desarrollo convencional de sistemas al desarrollo con técnicas basadas en procesado y aprendizaje de datos (\cite{Zhang2011}).

Las técnicas de \gls{ci} se usan principalmente para dos áreas: la caracterización y el modelado\sidenote{
	No es en estas áreas exclusivamente. La \gls{ci} se usa prácticamente en cualquier problema que tenga que ver con detección de patrones, predicción, planificación, etcétera. Por ello, es fácil encontrar aplicaciones de sus técnicas a prácticamente cualquier tema. Por ejemplo, en \cite{terroso2015complex} hacen uso de un \gls{cep} que procesa los diferentes sensores del vehículo para detectar y analizar patrones característicos con técnicas tales como clasificadores basados en \gls{fl} y, de esta manera, predecir no sólo el número de ocupantes, sino la tipología o clase de ocupante (e.g. niños, adultos, \ldots).
}.

\newthought{La caracterización} de conductores es interesante porque permite identificar perfiles de conducción y clasificar a los conductores de acuerdo a indicadores extraídos de su manera de conducir. Por ejemplo, en \cite{van2013driver} hacen uso de datos extraídos de sensores de inercia para construir un perfil de conductor, concluyendo que frenar y girar son mejores indicadores para la caracterización que la aceleración.

En~\cite{bando2013unsupervised} describen otro modelo no supervisado offline basado en un modelo bayesiano no paramétrico para la clusterización, combinado con un LDA (Latent Dirichlet Allocation, sea lo que sea esto) para la clusterización a más alto nivel. (\TODO ¿este método usa datos reales?)

Estos trabajos (\cite{sekizawa2007modeling}, \cite{terada2010multi} y \cite{bando2013unsupervised}) tienen la desventaja de ser computacionalmente muy costosos y con poca precisión en el caso de variar mucho los escenarios de entrenamiento y de test.

En~\cite{maye2011bayesian} se presenta un modelo online donde se infiere el comportamiento del conductor haciendo uso de una IMU (Intertial Measurement Unit) y una cámara. Primero de la IMU se sacan datos que se separan en fragmentos para luego relacionarlos con las imágenes obtenidas de la cámara. (\TODO ¿Hacia dónde mira la cámara?). Los modelos propuestos en~\cite{johnson2011driving} y~\cite{van2013driver} también se apoyan en el funcionamiento de clasificar las segmentaciones de una IMU, pero con técnicas distintas y sin cámara. Además hacen uso de señales externas y umbrales de activación para hacer más efectiva la clasificación (\TODO corroborar).

En el artículo~\cite{bender2015unsupervised} se usa también un modelo no supervisado online con una aproximación bayesiana para identificar los puntos de cambio sin depender de parámetros externos (e.g. umbrales o señales). Se basa también en (1) segmentar los datos de conducción y (2) asignar estos datos a clases que se corresponde n con comportamientos de conducción. Tiene la ventaja de ser más eficiente y rubusto que los anteriores.

La idea de estos métodos desde el~\cite{sekizawa2007modeling} hasta el~\cite{bender2015unsupervised} creo que es el de un sistema que traduzca datos en crudo a datos de más alto nivel. Esto es debido a que la cantidad de datos que se pueden generar en un sólo coche (no digamos ya una flota de ellos) es tan grande que para determinados sistemas disponer de información de más alto nivel haría más sencillo su desempeño (por ejemplo, \gls{adas} que funcionasen con datos de \enquote{adelantando} que sus valores de giro, aceleración en una ventana temporal).

En~\cite{satzoda2013towards}, haciendo uso de la información combinada de bus CAN, cámaras, GPS e información digitalizada el mapa donde se circula se determina una amplio abanico de información crítica en diferentes condiciones de la carretera. La información que sacan es: número de cambios de carril a la derecha, a la izquierda, tiempo en autopista y carretera urbana, distancia recorrida, velocidades medias en autopista y urbano, paradas, giros a la derecha, a la izquierda, incormporaciones y salidas de autopsta, tiempo gastado en un sólo carril, curvas a la izquierda, curvas a la derecha y distancia media al carril central


\newthought{}

En \cite{Dia2002} Los parámetros relativos al comportamiento, es decir, los que definen características, forma de razonar, etcétera son extraídos de encuestas a conductores reales. De acuerdo a los autores, cada conductor con sus propias características su forma de razonar sus perceciones y sus objetivos se puede modelar como un agente.

\newthought{En microsimulación}, las técnicas dominantes de los modelos de conducción son las \glsentrylong{ann} y la \glsentrylong{fl}. Las primeras debido a ser una de las técnicas principales en la rama del \gls{ml}, y la segunda por ser una manera sencilla y cercana a la manera de razonar del ser humano.

...

Para el cambio de carril

Los modelos estudiados no tienen en cuenta las inconsistencias y la incertidumbre de las percepciones y decisiones de un conductor humano \cite{McDonald1997}. Dichos modelos pertenencen a la clase \glsentrylong{hc}, esto es, valores fijos, ecuaciones y reglas de la lógica convencional para representar el conocimiento, las percepciones y las decisiones de los conductores. La única manera que tienen los autores de añadir incertidumbre en los modelos es mediante la introducción de términos aleatorios.

Sin embargo, los conductores basan sus decisiones en sus percepciones y éstas, aunque imprecisas, no tienen por qué seguir una distribución aleatoria con distribución clásica (e.g. normal). Las técnicas basadas en \gls{ci} palian los inconvenientes típicos de estas técnicas al pertenencer sus técnicas a soluciones de la clase \glsentrylong{sc}.

\cite{Das2009} una nueva metodología para cambio de carril basada en lógica difusa. El simulador se denomina Autonomous Agent SIMulation Package (AASIM). La motivación es que los sistemas difusos son capaces de modelar sistemas no lineales complejos como  reglas de la forma \texttt{if \ldots then \ldots} y que además la lógica difusa es ideal para modelar la incertidumbre del mundo real y por tanto de las percepciones de los conductores. Clasifican los cambios de carril en MLC y DLC. En MLC las reglas tienen en cuenta la distancia al siguiente punto característico (e.g. una salida) (\TODO{a lo mejor en la ilustración del principuio podemos ponerle nombre aesto y asi'erferirnos a ello en el resto del texto}) y el número de cambios de carril necesarios. En DLC deciden si cambiar o no basándose en el nivel de satisfacción del conductor y en el nivel de congestión en los carriles adjacentes.

\cite{McDonald1997, Wu2003} desarrollan otro modelo de simulación denominado Fuzzy LOgic motorWay SIMulation (FLOWSIM) con similares características donde establecen conjuntos difusos para el modelo. Dso categorías de cambio de carril, al lento (principalmente para evitar incordiar a los vehículos que se aproximan por detrás a velocidades superiores, usan dos variables, presión del vehículo trasero y satisfacción en el gap del carril destino) y al rápido (para ganar velocidad, variables: velocidad ganada con el cambio y oportunidad, es decir, seguridad y confort con el cambio).

\TODO{Aquí hay que meter el cuadro ese de las reglas difusas para ejemplo}

\TODO{Echar un vistazo a la tabla resumen}

\subsection{El enfoque de agentes}

...

\cite{Hidas2002} --> También aquí hablan de los DVOs (driver-vehicle objects) (¿esto es un clon de \glspl{dvu}?. Tienen características individuales como (i) tipo de vehículo, (ii) magnitudes físicas (tamaño, velocidad máxima, ...), (iii) tipo de conduictor y (iv) nivel de conocimiento de la red (porque afecta en la elección de ruta). Tienen un objetivo, llegar del origen al destino tan rápido como puedan. Esto implica un conjunto de decisiones a hacer en intervalos periódicos durante su funcionamiento (i) selección de ruta cada vez que se entra en un nuevo tramo y (ii) cálculo de la aceleración en cada intervalo).

En (Chaib-draa and Levesque, 1996, ) proponen un framework para trabajar con tres tipos diferentes de situaciones (rutina, familiar y no familiar) en sistemas multiagente, demostrando la aplicación en escenarios de microsimulacinó urbana). Su modelo se basa en una estructura jerárquica definida por los niveles de comportamiento humano y de técnicas de razonamiento propuestas por (Rasmussen, 1986, ) (skill-rule-knowledge). El comportamiento basado en habilidades (skills) se refiere a las actividades completamente automatizadas (percepción--ejecución) usadas típicamente en situaciones rutinarias. comportamiento basado en reglas se refiere a situaciones esteorotipadas (percepción--reconocimiento de la situación--planificación--ejecución) aplicable en su mayoría en situaciones familiares. El comportamiento basado en el conocimiento se refiere a actividades conscientes que implican trabajo de resolución de problemas y toma de decisiones (perception--reconocimiento de la situación--toma e decisión--planificación de la ejecución) que suelen ser necesarias en situaciones poco familiares. En SITRAS se ven estas diferencias claramente en el cálculo de la aceleración. (i) si no hay ninguna otra restriccinó, llegar a la máxima velocidad es acelerar hasta la máxima velocidad (skill), (ii) si hay ua luz roja más adelante, se va frenando hasta para el coche (rule) y (ii) si se recibe información de coches alrededor (por ejemplo se está incorporando un nuevo coche a nuestro carril) se requiere un conocimiento más complejo (knowledge). Los DVO aquí tienen las siguientes debilidades: no tienen memoria (sólo planean el segundo siguiente de acuerdo a la información actual) y tienen poco contacto directo con los demás DVOs de alrededor (saben del de delante y del de detrás, pero no de los lados).

En~\cite{al2001framework} describen un framework para la modelización de comportamiento de conductores dentro de simuladores. Se basa en cuatro unidades de funcionamiento interconectadas, la de percepción (percibe el entorno en términos locales y globales), la de emoción (cómo responde emocionalmete al entorno), la de decision-making (investiga posibles acciones que podrían servir a las necesidades del módulo emocional) y la de decision-implementation (intenta implementar las decisiones cuando emergen condiciones de tráfico lo suficientemente seguras para llevarlas a cabo). Tengo que volver a leerlo después de hacer una primera introducción en el tema de agentes, porque me parece poner nombrecitos a un tipo de agente que funciona de esa misma manera, pero lo mismo no. En \cite{Kuge2000} proponen otro framework de este palo.


En \cite{Das} se realiza una simulación de comportamiento de vehículos en autopista. Los agentes basan su comportamiento en un sistema difuso donde las reglas definen la conclusión en autopista (i.e. car-following y lane-changing). Llaman a este simulador AASIM (Autonomous Agent Simulator).,

En \cite{Ehlert2001} (ver si este otro paper del autor cuenta lo mismo y me quito de una de las dos referencias: \cite{Ehlert2001-2}), simulación donde los agentes son de tipo reactivo. Además, poseen diferentes estuilos de conduccińo. El agente cntinuamente va realizando decisiones de control para mantenerse en la via y llegar a su destino.


\subsection{Modelos basados en lógica difusa}

Los modelos de conducción basados en \glsentrylong{fl} parten de la hipótesis de que la información que maneja el conductor a la hora de tomar decisiones proviene de un análisis no demasiado detallado de la situación que le rodea; es decir, la percepción y el comportamiento humanos son estímulos percibidos de manera aproximada. Por tanto, el resultado debe ser fruto de un proceso de razonamiento que tenga en cuenta esa imprecisión en los estímulos.

El primer trabajo documentado en \gls{fl} es \cite{Kikuchi1992}, donde los autores aplicaron la lógica difusa sobre un modelo de aceleración de tipo \textit{\idx{car-following}}. Utilizaron el modelo \gls{ghr}\sidenote{
	El modelo \gls{ghr} (\cite{Chandler1958}) es el modelo más conocido antes de la introducción del modelo de Gipps. Desarrollado a finales de los años $50$, calcula el valor de la aceleración $a$ en un instante $t$ como:
	
	\begin{equation}
	a(t) = c v^m(t) \frac{\Delta v(t - \tau)}{\Delta x^l(t - \tau)}
	\label{eq:ghr-car-following-model}
	\end{equation}
	
	Siendo $t$ es el instante actual, $a(t)$ la aceleración del vehículo, $\delta v(t)$ y $\delta x(t)$ son la velocidad y distancia relativas al siguiente coche respectivamente, $v$ la velocidad del vehículo y $c$, $m$, $l$ y $\tau$ constantes, siendo ésta última el tiempo de reacción del conductor.
} como base y determinaron las entradas al modelo como valores de pertenencia a conjuntos difusos. Las entradas del modelo eran las distancias y velocidades relativas entre el vehículo modelado y el delantero y las variaciones en la aceleración del vehículo delantero, y como salidael cambio en la tasa de aceleración sobre el vehículo modelado. Más adelante los mismos autores aplicaron el mismo razonamiento sobre el modelo de General Motors (Modelo GMC)~\cite{Chakroborty1999}.

Otros trabajos en la línea de investigación de~\cite{Kikuchi1992} y \cite{Chakroborty1999}
...

Data-driven approaches have already been used in developing a fully adaptive cruise control system (Simonelli et al., 2009; Bifulco et al., 2013)

Otros trabajos que trabajan con lógica difusa: (Calibrating the membership functions of the fuzzy inference system: instantiated by car-following data, A FUZZY LOGIC MODEL OF FREEWAY DRIVER BEHAVIOR, The Modeling and Simulation of the Car-following Behavior Based on Fuzzy Inference, Fuzzy parameters estimation for car-following modelling, A Fuzzy Logic Approach for Car-Following Modelling, Variable response time lag module for car-following models, Development of a fuzzy logic based microscopic motorway simulation model, Establishment of car following theory based on fuzzy-based sensitivity parameters. Advances in multimedia modelling, Application of fuzzy systems in the car-following behaviour analysis, The validation of a microscopic simulation model: A methodological case study). Ninguno de estos estudios considera diferentes tipologías de vehículos para el car following

Problemas:

\begin{enumerate}
	\item ¿Qué reglas usan los humanos para modelar su comportamiento? Desconocerlas implica modelos no realistas. En (The validation of a microscopic simulation model: A methodological case study) intenta suplir este problema con encuestas a conductores.
	\item Los problemas inherentes de los controladores difusos. ¿Cómo validar las funciones de pertenencia? ¿cómo determinar las reglas difusas?
\end{enumerate}

\subsection{Modelos basados en redes neuronales artificiales}

Las redes neuronales se han aplicado mucho sobre el campo de las ITS en general, y sobre la conducción autónoma y el análisis del comportamiento de los conductores (A review of neural networks applied to transport).

En (Modelling human performance with neural networks) implementaron un controlador basado en redes neuronales para el comportamiento del car following en un microsimulador entrenando dicho modelo previamente con datos extraídos de un conductor en dicho simulador.

En (The  use  of  neural  networks  to  recognise  and
predict traffic congestion) usan redes para determinar el nivel de congestión en la vía.

(Develop a car-following model using data collected by ‘five-wheel
system’) son los primers en usar datos reales de un coche instrumentado usando el método Five-Wheel-System, que está especificado en su paper de aquella manera y que no me entero de nada. A partir de las entradas correspondientes a velocidad relativa, espacio relativo, velocidad y velocidad deseada (para ello, clasifican al conductor de agresivo, normal, conservador) determinan la aceleración/deceleración del vehículo. No lo aplican a ningún simulador, sólo que los valores se ajustan.

(Neural agent car-following models) redes neuronales usando el dataset de (Traffic simulation supporting urban control system development) desarrollan una red neuronal para mantener la distancia con el siguiente vehículo. Este modelo sí se evaluó en el simulador AIMSUN, y los resultados muestran una buena correspondiencia entre los datos y la realidad. No replica sin embargo el comportamiento de frenar hasta parar o de acelerar desde parado.

A Modified Car-Following Model Based on a Neural Network Model of the Human Driver Effects


Problemas:

\begin{enumerate}
	\item Es imposible determinar por qué la red funciona como está funcionando.
	\item Los clásicos de los problemas de redes, el no aprendizaje y la especialización.
\end{enumerate}

...

En (Hunt and Lyons, 1994, Modelling of dual-carriageway lane-changing using neural networks) desarrollan un modelo de decisión usando redes neuronales. El modelo funciona a partir de presentarle una entrada visual del entorno alrededor del vehículo que quiere cambiar de carril. Sin embargo, no considera la cooperación entre vehículos.

Simonelli et al. (2009) have applied neural networks to develop a real-time learning mode for capturing car-following behavior taking into consideration individual drivers’ characteristics. Bifulco et al. (2013) extended the work of Simonelli et al. (2009) into a framework for reproducing spacing in adaptive cruise control applications. While in this research we have used data derived from the same experiment as Simonelli et al. (2009), the scope and level of complexity of the studies is very different. While all studies adopt a data-driven approach, in this paper the objective is to create a simple and practical methodology for speed estimation using car-following models for use in a microscopic traffic simulator.

\subsection{Otras técnicas}

En \cite{Hou2011} proponen una técnica basada en modelos ocultos de Markov para la identificación de posibles cambios de carril. Los conductores conducen vehículos en un simulador de autopistas. A partir del ángulo de giro del volante el modelo es capaz de estimar, con una precisión del $0,95$, si el conductor va a realizar un cambio a la derecha, a la izquierda o si va a mantenerse en el carril. Creo que \cite{Berndt2008} es también por el estilo. De hecho puede que el primero es haya \enquote{inspirado} mucho.

\subsection{Aproximaciones híbridas}

(Simulation of car-following decision using fuzzy neural network system) y (Toward an integrated car-following and lane-changing model based on neural-fuzzy approach) usan aproximaciones de fuzzy neural networks y neuro-fuzzy respectivamente. No proveen sin embargo de documentación y no se investiga la aplicación de estos modelos a microsimuladores de tráfico.

(Exploring a local linear model tree approach to car-following) usan el modelo de árboles lineales locales (LOLIMOT, (Nonlinear system identification: From classical approaches to neural networks and fuzzy models)) que no deja de ser una aproximación neuro-fuzzy del comportamiento. Intenta incorporar imperfecciones perceptuales en un modelo de car following. El modelo está basado usando datasets reales y los resultados indican que se ajusta lo predicho con la realidad, pero no hay pruebas realizadas en microsimuladores.

En~\cite{Ma2004}, bajo la suposición de que el ser humano toma múltiples decisiones relacionadas basándose en su percepción (imprecisa) del entorno propone un método masado en un sistema de inferencia difusa para tomar decisiones tanto para el problema del \textit{car following} como para el del \textit{lane changing}, calibrando y ajustando dicho controlador mediante el uso de redes neuronales (aproximación Neuro-Fuzzy).

Modeling car-following behavior via artificial neural networks (Colombaroni and Fusco, 2013; Chong et al., 2013; Zheng et al., 2013)
\part{Desarrollo de la tesis}
\chapter{Sistemas desarrollados}
\label{ch:developed-software}

Este capítulo describe todos los sistemas y el software desarrollados e implementados para realizar la tesis. Éstos son tanto los encargados de la captura de datos de los conductores, los que trabajan directamente con el simulador para integrar los controladores generados y los desarrollos para la generación de Software.

\section{ScanBUS}

ScanBUS es un software para la identificación de paquetes enviados por dispositivos a través del Bus CAN del vehículo.

\section{Sistema para la captura de datos multidispositivo}

Para la obtención de los datos de conducción se ha desarrollado un sistema que permite la conexión a múltiples dispositivos desde diferentes interfaces. Las razones para su desarrollo son las siguientes

\begin{itemize}
	\item Sincronización automática de datos de dispositivo en intervalos configurables de tiempo. El sistema permite la configuración de la recuencia de captura sincronizando los datos recibidos a esa frecuencia.
	\item Diseño extensible a otros dispositivos. Es software está diseñado para facilitar en la medida de los posible la introducción de nuevos dispositivos usando, para ello, las interfaces apropiadas.
	\item Hardware compacto. El sistema está integrado en un ordenador de tipo Raspberry PI, aunque es factible su integración en otros sistemas siempre y cuando funcionen con un sistema GNU/Linux e incluyan el hardware necesario para las capturas.
\end{itemize}

\section{Biblioteca para la incorporación de modelos de conductor personalizados en SUMO}

Hace uso de \gls{traci}.

\section{Modelos de comportamiento}

\subsection{Entrenamiento de controladores difusos mediante \gls{cev}}


\chapter{Estudio de modelos de comportamiento}
\label{ch:behavior-models-study}

...

Creo que además de crear modelos adaptados al conductor de car following y de lane change, estaría bien adaptar los parámetros de los modelos existentes y comparar.

Cosas que se me ocurren. Habría que tener un módulo que se plantea qué acción tomar en función del entorno y qué acciones ocurren alrededor nuestro. Quiero cambiar de carril, está entrando un usuario al carril,

Lo mismo es una chorrada, pero en \cite{Alexiadis2004} el autor se saca de la manga un intervalo temporal para diferenciar las tareas de niveles estratégico, táctico y de control. Estos intervalos podrían ser una variable lingüística y sus particiones variar en función del conductor.

En \cite{Toledo2007} (o \cite{Toledo2007-3}), plantea cosas a mejorar en el estado de la cuestión: (i) más regímenes de conducción y mejor determinación de sus límites y sus transiciones, tanto en modelos de aceleración como en modelos de cambio de carril, mejorando así el realismo de la simulación, (ii) posicionamiento estratégico, es decir, incorporar el ptah planning y cosas por el estilo del comportamiento de planificación al táctico preposicionando el coche en los carriles más convenientes, (iii) más campo de visión, no sólo los delanteros y traseros del carril actual y carril destino, (iv) interdependencia entre modelos y regímenes de modelos, (v) planificación y anticipación,sobre todo esto último en el tema de qué van a hacer los vehículos de alrededor. Si tiramos de algo de aquí, vendría bien mirar el paper porque da referencias a trabajos en estas direcciones.

En un paper que me ha pasado Felipe para revisar (A Novel Method for Predicting Vehicle State in Internet of Vehicle) plantean un modelo basado en árboles de clasificación para identificar el estado del vehículo. El caso es que plantean comportamientos del estilo "evitar jam", "intento de adelantamiento" (que por cierto, se basan únicamente en el tamaño y velocidad del vehículo delantero). Echarle un ojo más adelante.

...

Proponer el modelo en cuatro partes. Car following, free flow, lane exit, decisión de cambio de carril y ejecución de cambio de carril. Todos ellos se ejecutarán con un árbol molón en el que primero se realizarán o no los cambios de carril y luego se decidirá la velocidad del vehículo (yéndose por una de las dos ramas del modelo longitudinal).

\begin{enumerate}
	\item **Car following**. En este caso, el modelo se definirá como controladores difusos ajustados a los conductores con lo que he estado haciendo de representar el controlador difuso como grafo computacional. Las entradas el modelo serán la distancia al vehículo siguiente (hasta un máximo), la diferencia de velocidad con éste y ¿la aceleración actual?. La salida será la aceleración.
	\item **Free flow**. Este caso creo que lo mejor será representarlo como un perceptrón multicapa donde las entradas sean la velocidad máxima de la vía, la velocidad actual del vehículo y la aceleración de esta. Tanto en el caso anterior como en este, la salida deberá ser la aceleración.
	\item **Lane exit**. Este caso es similar al siguiente en entradas, pero la salida es la aceleración a aplicar.
	\item **Decisión de cambio de carril**. Para decidir el cambio de carril debemos saber el estado de los carriles de alrededor, si es posible o no desde esos carriles seguir por la siguente salida y la distancia al comienzo y al final de la siguiente salida. La salida del modelo debe ser "quiero cambiar a la izquierda, a la derecha o no cambiar".
	\item **Ejecución de cambio de carril**. Este es el que hecho y parte de las entradas del entorno y del módulo anterior. Esto lo tengo que describir de forma similar a la del paper publicado.
\end{enumerate}

El orden de los módulos es modelo más grande cuya salida son cuatro valores: la aceleración (como valor de coma flotante) a aplicar, cambio a la izquierda (0 o 1) y cambio a la derecha (0 o 1).

Las entradas pasarán, por un lado a los modelos longitudinales (fx, ff y le) y el resultado será el del menor de los tres y por otro al modelo de cambio de carril.

\section{El vehículo como agente inteligente}

En nuestra tesis el trabajo se basa en la representación del comportamiento de un conductor (\gls{dvu}).

\section{Resultados}
\label{ch:behavior-models-study:results}

Realizar comparativa de modelos existentes contra modelos de parámetros ajustados contra modelos creados.
\chapter{Resultados}
\label{ch:results}

\chapter{Conclusiones}
\label{ch:conclusions}

\section{Aportaciones}
\label{ch:conclusions:contributions}

\section{Fururas líneas de investigación}
\label{ch:conclusions:future-work}

¿A lo mejor se podría tirar por el campo de las V2X desde esta tesis?

¿Entrar en el tema de la mesosimulación?
\backmatter

\bibliography{thesis}
\bibliographystyle{apalike}
\printindex

\cleardoublepage
\begin{fullwidth}
~\vfill
\thispagestyle{empty}
\setlength{\parindent}{0pt}
\setlength{\parskip}{\baselineskip}
\par{
	\textbf{Acerca del código fuente}
	
	La presente tesis lleva consigo numerosas horas de programación, lo que implica varios miles de líneas de código. Sin embargo, esta nota existiria aún con sólo unas pocas decenas. Se ha decidido no proveer de forma impresa el código fuente y en su lugar distribuirlo en formato electrónico, una forma más manejable para su consulta y a la vez respetuosa con el medio ambiente. No obstante sí es posible que existan pequeños fragmentos de código para apoyar explicaciones. En caso de necesitar los fuentes y no ser capaces de conseguirlos, se puede contactar directamente conmigo, el autor, en \href{mailto:alberto.diaz@upm.es}.
	}

\par{
	\textbf{Cómo citar esta tesis}
	
	Si deseas citar esta tesis, lo primero gracias. Me alegro de que te sirva para tu investigación. Si lo deseas, incluye el siguiente código en bibtex:
	
	\textbf{TODO A ver cómo coño meto en el paquete listings caracteres acentuados...}
	}


\end{fullwidth}




\end{document}