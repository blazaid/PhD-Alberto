\chapter{Resultados}
\label{ch:results}

Mínimo de resultados, los datasets, porque además en materia de cambio de carril hay poco.

Frase para poner en los resultados a raíz de los que estoy obteniendo. Quizá si hablo un poco más de este caso relleno más y parece más consistente: La configuración de la salida en el caso de los cambios de carril es cuanto menos curiosa. Cuando la salida se realiza de forma que el no-cambio de carril se encuentra en un extremo (e.g. 0, 1, 2 siendo 0 el no cambio y 2 el cambio), el entrenamiento devuelve modelos peores que en el caso en el que el no-cambio de carril se encuentra en el medio (e.g. -1, 0, 1). Me aventuro a predecir que en los casos en los que la salida al problema a modelar son los estados con transiciones entre ellos dentro del problema, mantener la relación de cercanía entre transiciones de estado mejora los resultados del entrenamiento y del modelo aprendido (en nuestro caso, -1, 0, 1 mantiene una distancia de 1 transición, mientras que en el caso 0, 1, 2 hay una distancia que no se corresponde con la distancia en transiciones). Esta predicción es una generalización de la experiencia obtenida en estos experimentos y requiere de más estudio.