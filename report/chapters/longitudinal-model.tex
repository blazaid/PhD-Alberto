\chapter{Comportamiento longitudinal}
\label{ch:longitudinal-model}

El comportamiento longitudinal es un problema de regresión sobre cómo ha de comportarse la aceleración del \ac{dvu} en función de la información existente alrededor. El perfil de aceleración para el conjunto general de conductores tanto del conjunto de test como del de entrenamiento se ilustran en la figura~\ref{fig:acceleration-profiles}.

\begin{figure}[!b]
	\centering
	\subfloat[Entrenamiento]{\includegraphics[width=\textwidth]{acceleration-profiles-training}}\qquad
	\subfloat[Test]{\includegraphics[width=\textwidth]{acceleration-profiles-test}}
	\caption[Perfiles de aceleración a ajustar por los modelos. Conjuntos de entrenamiento y de test]{Perfiles de aceleración durante los experimentos: (a) entrenamiento y (b) test. Estos son los conjuntos de datos que se intentarán ajustar con los modelos.}
	\label{fig:acceleration-profiles}
\end{figure}

Por la naturaleza del problema se han seleccionado dos técnicas diferentes para el ajuste del modelo:

\begin{enumerate}
	\item \ac{mlp}. Al ser un problema de regresión, el uso de \acrlongplsp{mlp} en el comportamiento longitudinal está justificado a ser considerados éstos aproximadores universales~\cite{hornik1991approximation}\sidenote{El \textit{Teorema de aproximación Universal} postula que un \ac{mlp} con al menos una capa oculta es capaz de aproximar cualquier función si dispone de suficientes neuronas en ésta.}
	\item \ac{fcs}. La formulación de este problema se ajusta muy bien al funcionamiento de estos modelos, donde en función de las entradas y de acuerdo a una serie de reglas, el controlador toma una decisión para la salida. Además, los \acrlongplsp{fcs} tienen la ventaja de que se puede explicar cómo funcionan, cosa que no es posible para un \acrlongsp{mlp} con una o más capas ocultas.
\end{enumerate}

Ambos modelos se ajustarán con un procedimiento basado en el descenso del gradiente denominado ADAM~\cite{kingma2014adam}. Su aplicación a los \acrlongplsp{mlp} es directa (después de todo es una técnica desarrollada dentro de área de las \acrlongplsp{ann}\index{red neuronal artificial}), pero para los \acrlongplsp{fcs}\index{sistema de control difuso} es necesaria una representación que permita el uso de este método para su optimización. Esta representación es una de las aportaciones de esta tesis y se se explica en el apéndice \nameref{ch:fuzzy-controller-adjustment}.

El error que se trata de minimizar es el cuadrado del \gls{rmse}\index{RMSE} entre los valores de aceleración del conjunto reales y los ajustados por el modelo, ya que es equivalente. Los resultados no obstante se presentan con el \gls{rmse}\index{RMSE} ya que éste es directamente interpretable en las medidas de la variable sobre la que se aplica (en nuestro caso, \SI{}{\metre\per\square\second}).

\subsection{Descripción de los datasets}

Del conjunto de datos descrito en la Tabla~\ref{tbl:main-variables} del capítulo~\nameref{ch:methodology} seleccionamos aquellos indicadores de interés: \textit{Distancia a líder}, \textit{Distancia a siguiente TLS}, \textit{Estado de siguiente TLS}, \textit{Velocidad}, \textit{Velocidad a líder} y \textit{Aceleración} (este último como salida a ajustar por el modelo). Estos valores conformarán los conjuntos de entrenamiento y de test tanto para cada uno de los sujetos $S_1$, $S_2$ y $S_3$, como para el conjunto global de éstos $S_A$.

Dado que nuestros modelos se basan en un esquema \textit{feed-forward}, existe el inconveniente de que para ellos es imposible mantener una memoria del orden en el que se están sucediendo las entradas. Sin embargo, contamos con las derivadas de la posición respecto al líder (la aceleración y la velocidad al líder) y la velocidad, por lo que consideramos que disponemos de información temporal suficiente para este problema en concreto.

El resultado de la generación de los datasets de entrenamiento y test para todos los sujetos se resume en la Tabla~\ref{tbl:cf-datasets-description}

\begin{table}
	\centering
	\caption[Descripción de los conjuntos de datos]{Descripción de los conjuntos de datos para el entrenamiento de los modelos. $CF_{S_A}$ se corresponde con el modelo de conductor global, mientras que cada $CF_{S_i}$ es la porción de datos correspondiente al sujeto $S_i$.}
	\label{tbl:cf-datasets-description}
	\begin{tabular}{ccccc}
		\toprule
		& & & \multicolumn{2}{c}{Tamaño} \\
		& Entradas & Salidas & Entrenamiento & Test \\
		\midrule
		\rowcolor{black!20} $CF_{S_1}$ & $7$ & $1$ & $1089$ & $543$ \\
		$CF_{S_2}$ & $7$ & $1$ & $1313$ & $560$ \\
		\rowcolor{black!20} $CF_{S_3}$ & $7$ & $1$ & $2067$ & $668$ \\
		$CF_{S_A}$ & $7$ & $1$ & $4469$ & $1771$ \\
		\bottomrule
	\end{tabular}
\end{table}

\section{Modelo basado en \acrlongplsp{fcs}}

Se han realizado entrenamientos sobre arquitecturas con diferente número de particiones difusas en las variables. Las arquitecturas que se han considerado más relevantes (tras un proceso de ensayo y error con diferente número de particiones difusas) se describen en la Tabla~\ref{tbl:cf-fcs-architectures}. 

\begin{table}[t]
	\centering
	\small
	\caption[Resumen de las arquitecturas \ac{fcs} para el modelo longitudinal]{Resumen de las arquitecturas seleccionadas. La posición de cada número de la arquitectura indica a qué variable lingüística se refiere (\textit{Distancia al líder}, \textit{Distancia a siguiente TLS}, \textit{TLS en verde}, \textit{TLS en amarillo}, \textit{TLS en rojo}, \textit{Velocidad} y \textit{Velocidad de aproximación al líder} respectivamente). Su valor se corresponde al número de conjuntos difusos de la partición de cada una de ellas.}
	\label{tbl:cf-fcs-architectures}
	\begin{tabular}{cccccc}
		\toprule
		\multirow{2}{*}{} & \multirow{2}{*}{Arquitectura} & \multirow{2}{*}{Epochs} & \multicolumn{3}{c}{RMS}      \\ 
		& & & Entrenamiento & Validación & Test \\
		\midrule
		\rowcolor{black!20} $FCS_1$ & $2, 2, 2, 2, 2, 2, 2$ & $10^5$ & $0.059$ & $0.064$ & $0.062$  \\
		$FCS_2$ & $3, 3, 2, 2, 2, 3, 3$ & $10^5$ & $0.073$ & $0.079$ & $0.080$  \\
		\rowcolor{black!20} $FCS_3$ & $4, 3, 2, 2, 2, 3, 3$ & $10^5$ & $0.072$ & $0.078$ & $0.088$  \\
		$FCS_4$ & $5, 5, 2, 2, 2, 5, 5$ & $10^5$ & $0.063$ & $0.068$ & $0.109$  \\
		\bottomrule
	\end{tabular}
\end{table}

Todas las arquitecturas mantienen el número de conjuntos difusos asociados a las variables de estado del semáforo (verde, ámbar y rojo) a $2$. La razón es debido a que dicha variable sólo puede tomar dos valores, $0$ o $1$, y en la inicialización de sus respectivas particiones, estas variables ya toman dichos valores con un grado de pertenencia de $1$ en sus dominios.

Cada uno de los controladores se ha entrenado durante $250.000$ epochs. En un principio, el proceso de entrenamiento fue el habitual, ajustando todas las variables a la vez (particiones difusas y pesos asociados a las reglas).

\begin{figure}[!b]
	\centering
	\subfloat[Entrenamiento y validación]{\includegraphics[width=.45\textwidth]{lm-fcs-rmse-all-training-and-validation-detail}}\qquad
	\subfloat[Test]{\includegraphics[width=.45\textwidth]{lm-fcs-rmse-all-test-detail}}
	\caption[Evolución del error durante el entrenamiento en las arquitecturas de \acrshort{fcs} para el modelo longitudinal]{Visión en detalle de la evolución del error en los conjuntos de entrenamiento y validación (izquierda) y de test (derecha). Para cada arquitectura, el color más transparente se corresponde al error en el conjunto de validación. El error de test se muestra debido a que nos ofrece puede ofrecer intuición de qué forma aprende la red, pero su evolución no se ha considerado para determinar las arquitecturas.}
	\label{fig:lm-fcs-rmse-all-comparisons}
\end{figure}

Sin embargo, tras varias pruebas, se ha comprobado que el ajuste de las variables que determinan las particiones difusas parece suceder más rápido que el ajuste de los pesos asociados a las reglas. Por ello, en lugar entrenar a la vez, se ha particionado el entrenamiento en secuencias sucesivas de entrenamiento de reglas y entrenamiento de particiones difusas.

Concretamente, los $250.000$ epochs de entrenamiento han sido divididos en $250$ iteraciones de:

\begin{enumerate}
	\item $800$ epochs ajustando sólo las reglas con una tasa de entrenamiento de $0.01$
	\item $200$ epochs ajustando sólo las variables de las particiones difusas con una tasa de entrenamiento de $0.001$.
\end{enumerate}

\begin{figure}[t]
	\centering
	\subfloat[Distancia a líder]{\includegraphics[width=.45\textwidth]{lm-fcs-best-architecture-leader-distance-variable-partition}}\qquad
	\subfloat[Distancia a siguiente semáforo]{\includegraphics[width=.45\textwidth]{lm-fcs-best-architecture-next-tls-distance-variable-partition}}\qquad
	\subfloat[Velocidad]{\includegraphics[width=.45\textwidth]{lm-fcs-best-architecture-speed-variable-partition}}\qquad
	\subfloat[Velocidad a líder]{\includegraphics[width=.45\textwidth]{lm-fcs-best-architecture-speed-to-leader-variable-partition}}
	\caption[Particiones difusas después de la operación de ajuste en el modelo longitudinal basado en \acrshort{fcs}]{Particiones difusas despueés de la operación de ajuste en el modelo longitudinal basado en \ac{fcs}. Las funciones de pertenencia correspondientes al estado del semáforo no se incluyen ya que no se vieron modificadas en ningún entrenamiento. Después de todo, estas variables eran binarias, y las particiones difusas se inicializan de tal manera que dichos valores siempre alcanzan un valor de pertenencia $1$. El de la variable \textit{Velocidad a líder} es cuanto menos, curioso. Tras varias ejecuciones sobre esta y otras arquitecturas, ha mantenido este comportamiento, por lo que intuimos que la variable tiene un efecto mínimo o nulo sobre el comportamiento en el modelo longitudinal.}
	\label{fig:adjusted-fuzzy-partitions}
\end{figure}

Empíricamente (y para este problema en concreto) se ha podido observar que el entrenamiento realizado de esta manera hace que el \ac{rmse} descienda más rápido en el mismo número de iteraciones.

En la Figura~\ref{fig:lm-fcs-rmse-all-comparisons} se puede observar la evolución en general y un detalle de la disminución del error durante el entrenamiento de los controladores. El error en test se ofrece a título informativo, y no ha sido usado para determinar las arquitecturas seleccionadas.

Se puede observar que la arquitectura $FCS_1$ es la que mejor error en test ha alcanzado. En las pruebas realizadas para el ajuste de controladores se ha observado además que los errores en entrenamiento y en test tienden a ser similares cuando los problemas son representables con un \acrlongsp{fcs}.

Éste será por tanto, el modelo seleccionado para la comparativa final. Las funciones de pertenencia ajustadas se muestran en la Figura~\ref{fig:adjusted-fuzzy-partitions} contrastadas con las versiones de antes de comenzar el entrenamiento. Las reglas, sin embargo, son muy numerosas para ser incluidas en el texto, ya que la mayoría de reglas posibles en el sistema (aproximadamente $60$) tienen un peso asociado de más de $0.5$.

Un detalle que merece la pena destacar es el aprendizaje de la variable lingüística \textit{Velocidad a Líder}. En todos los entrenamientos conducidos con las arquitecturas de la Tabla~\ref{tbl:cf-fcs-architectures} el comportamiento ha sido el mismo. Esto nos hace pensar que esta variable tiene un efecto mínimo o nulo sobre el comportamiento del modelo longitudinal.

En la Figura~\ref{fig:fcs-test-comparisons} se muestran los perfiles de la aceleración estimada por los controladores frente la aceleración real en cada momento. 

\begin{figure}[t]
	\centering
	\subfloat[Perfil de aceleración]{\includegraphics[width=\textwidth]{lm-fcs-outs-all-test-comparison}}\qquad
	\subfloat[Detalle entre frames de $750$ a $1050$]{\includegraphics[width=\textwidth]{lm-fcs-outs-all-test-comparison-detail}}
	\caption[Comparación del perfil de aceleración real y el inferido por los modelos entrenados]{Comparación del perfil de aceleración real y el inferido por los modelos entrenados. En la visión general se puede observar, en transparente, el perfil real. A la derecha se amplía una pequeña sección del perfil para mostrar los diferentes ajustes de los modelos entrenados y cómo difieren del valor real.}
	\label{fig:fcs-test-comparisons}
\end{figure}

\section{Modelo basado en \acrlongplsp{mlp}}

Para determinar el modelo óptimo de \acrlongsp{mlp} en comportamiento longitudinal, se han realizado entrenamientos sobre arquitecturas con diferente cantidad de neuronas y capas ocultas. Las arquitecturas más ilustrativas de todas las probadas se resumen en la tabla~\ref{tbl:cf-mlp-architectures}.

El modelo de funciones de activación que se ha utilizado es de tipo tangente hiperbólica en todas las neuronas salvo en la última, que se ha utilizado una activación lineal\sidenote{Se han utilizado también funciones de activación de tipo \acrshort{relu}, pero las tasas de error tras el entrenamiento eran notablemente más altas por lo que se ha optado al final por el uso de activación basada en tangente hiperbólica.}. Los pesos de la red han sido inicializados con una muestra aleatoria uniforme de valores reales en el intervalo $(-0.25, 0.25)$.

\begin{table}
	\centering
	\caption[Resumen de las arquitecturas de \acrlongplsp{mlp} para el modelo longitudinal]{Resumen de las arquitecturas de de \acrlongplsp{mlp} para el modelo longitudinal. La posición de cada número de la topología indica el índice de la capa oculta y su valor el número de nodos (neuronas) que incluye dicha capa. Las arquitecturas seleccionadas en esta tabla son aquellas consideradas relevantes tras un proceso manual de ensayo y error.}
	\label{tbl:cf-mlp-architectures}
	\begin{tabular}{cccccc}
		\toprule
		\multirow{2}{*}{} & \multirow{2}{*}{Topología} & \multirow{2}{*}{Epochs} & \multicolumn{3}{c}{RMS} \\
		& & & Entrenamiento & Validación & Test \\
		\midrule
		\rowcolor{black!20} $MLP_1$ & $16$        & $10^5$ & $0.057$ & $0.057$ & $0.069$  \\
		$MLP_2$ & $8, 2$      & $10^5$ & $0.061$ & $0.061$ & $0.056$  \\
		\rowcolor{black!20} $MLP_3$ & $16, 8$     & $10^5$ & $0.051$ & $0.051$ & $0.060$  \\
		$MLP_4$ & $16, 16, 8$ & $10^5$ & $0.046$ & $0.046$ & $0.061$  \\
		\bottomrule
	\end{tabular}
\end{table}

La Figura~\ref{fig:lm-mlp-rmse-all-comparisons} muestra la evolución del \gls{rmse} durante el proceso de entrenamiento. También se ilustra el error de test durante el mismo, aunque éste no se ha utilizado para decidir las arquitecturas y es simplemente informativo.

\begin{figure}[t]
	\centering
	\subfloat[Entrenamiento y validación]{\includegraphics[width=.45\textwidth]{lm-mlp-rmse-all-training-and-validation-detail}}\qquad
	\subfloat[Test]{\includegraphics[width=.45\textwidth]{lm-mlp-rmse-all-test-detail}}
	\caption[Evolución del error durante el entrenamiento en las arquitecturas de \acrshort{mlp} para el modelo longitudinal]{Visión en detalle de la evolución del error en los conjuntos de entrenamiento y validación (izquierda) y de test (derecha). Para cada arquitectura, el color más transparente se corresponde al error en el conjunto de validación. El error de test se muestra debido a que nos ofrece puede ofrecer intuición de qué forma aprende la red, pero su evolución no se ha considerado para determinar las arquitecturas.}
	\label{fig:lm-mlp-rmse-all-comparisons}
\end{figure}

\begin{figure}[!b]
	\centering
	\subfloat[Perfil completo]{\includegraphics[width=\textwidth]{lm-mlp-outs-all-test-comparison}}\qquad
	\subfloat[Detalle entre frames de $750$ a $1050$]{\includegraphics[width=\textwidth]{lm-mlp-outs-all-test-comparison-detail}}
	\caption[Comparación del perfil de aceleración real y el inferido por los modelos entrenados]{Comparación del perfil de aceleración real y el inferido por los modelos entrenados. En la visión general se puede observar, en transparente, el perfil real. A la derecha se amplía una pequeña sección del perfil para mostrar los diferentes ajustes de los modelos entrenados y cómo difieren del valor real.}
	\label{fig:cf-mlp-test-comparisons}
\end{figure}

Estos errores se encuentran entre los \SI{0.05}{\metre\per\square\second} y los \SI{0.07}{\metre\per\square\second}, lo cual consideramos que es una aproximación aceptable. Una particularidad del problema ha sido la inestabilidad de los entrenamientos, esto es, la alta sensibilidad a los valores de inicialización de los parámetros. La intuición tras ver la evolución de los entrenamientos es que la función de error del problema tiene muchos mínimos locales o mesetas.

Al contrastar los errores de test, podemos determinar que la arquitectura que parece que mejor generaliza es la $MLP_2$ (arquitectura $7, 8, 2, 1$).

Una visión de detalle del ajuste de estas arquitecturas al conjunto de test se puede ver en la figura~\ref{fig:cf-mlp-test-comparisons}, donde se muestra el perfil de aceleración del conjunto de test y los perfiles de aceleración de las redes entrenadas.

A la vista de los resultados, y dado que las arquitecturas se ajustan bien a los datos reales, es razonable elegir el modelo $MLP_2$ debido a que es el que aparentemente mejor generaliza los comportamientos del conjunto de conductores.

\section{Comparación entre modelos}

Las mejores arquitecturas de ambos modelos han sido la $MLP_2$ para los \acrlongplsp{mlp} y $FCS_1$ para los \acrlongplsp{fcs}. Los errores y el perfil de aceleración para ambos modelos se muestran en la figura~\ref{fig:cf-comparison-between-best-mlp-and-fcs-architecture}.

\begin{figure}
	\centering
	\subfloat[Error en test]{\includegraphics[width=.46\textwidth]{comparison-between-best-mlp-and-fcs-architecture-rms}}\qquad
	\subfloat[Perfil de aceleración]{\includegraphics[width=.46\textwidth]{comparison-between-best-mlp-and-fcs-architecture-acceleration-profile}}
	\caption[Comparación entre los dos tipos de modelo longitudinal]{Comparación de la mejor arquitectura de \acrlongplsp{mlp} frente a la mejor arquitectura de \acrlongplsp{fcs}: (a) diferencia entre los errores cuadráticos medios de ambas arquitecturas y (b) perfiles de aceleración en el conjunto de test.}
	\label{fig:cf-comparison-between-best-mlp-and-fcs-architecture}
\end{figure}

Aunque el modelo basado en \acrlongplsp{fcs} arroja un error bajo en test y parece que tiende a ajustarse al perfil de aceleración, parece que el problema es suficientemente complejo como para no poder representarse como un simple \acrlongsp{fcs}.

Además, el error arrojado por el \acrlongsp{mlp} es sustancialmente menor y, por tanto, la arquitectura elegida para el modelo longitudinal serála $MLP_2$.

