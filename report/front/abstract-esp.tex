\cleardoublepage
\thispagestyle{empty}
\chapter*{Resumen}
\begin{fullwidth}
	El tráfico se puede concebir como un sistema complejo cuyo estudio requiere, o bien la inmersión en el sistema real o bien su simulación. Y si el objetivo es una mayor fidelidad en los comportamientos o evaluar un elemento concreto de este entorno, la opción es mediante el uso de micro-simuladores, comúnmente basados en arquitecturas de \acp{mas} donde cada elemento es modelado por separado. El incremento de los recursos computacionales nos permiten aumentar la complejidad de las simulaciones según pasa el tiempo, pero los agentes que modelan a los conductores se suelen basar en funciones más o menos complejas que responden poco o nada al comportamiento de un conductor real.
	
	Esto provoca que dichos modelos se adapten poco al comportamiento de un conductor en concreto. Por otro lado, estamos viviendo una época donde los avances de la \ac{ci} en materia de predicción, clasificación y modelización están siendo apabullantes. Esto hace pensar que la aplicación de estas técnicas en agentes de conductores pueda hacer que se comporten \textit{\enquote{mejor}} (en el sentido \textit{\enquote{más humano}}), permitiendo un mejor análisis tanto del tráfico como de los agentes que conducen en él.
	
	Esta tesis explora cómo la \ac{ci}, aplicada a la modelización de conductores reales, puede mejorar el desempeño de las simulaciones de tráfico en el sentido de hacer más reales sus resultados. Concretamente, se centra en técnicas basadas en \acp{ann} y \acp{fcs} para, a partir de datos de conductores reales, estudiar de qué manera las diferentes técnicas se comportan a la hora de reproducir comportamientos humanos en micro-simuladores.
	
	El estudio se centra en dos características existentes dentro de este tipo de simulaciones: el comportamiento longitudinal y el de cambio de carril. Tras el estudio, se concluye que las técnicas basadas en \acp{ann} son claramente superiores a la hora de modelar el comportamiento humano en estas tareas, y que dentro de ellas, las \acp{cnn} en la tarea del cambio de carril son idóneas frente a los \acp{mlp} a la hora de interpretar el entorno circundante del vehículo.
	
	Las implicaciones de estos resultados son variadas. Aun con las limitaciones que nos imponen los micro-simuladores, es posible modelar estos dos comportamientos a partir de datos reales con precisión suficiente como para decir que los agentes se comportan de manera más humana. Además, estos comportamientos son dependientes del perfil del que fueron obtenidos los datos, por lo que es posible su uso no sólo para modelar comportamiento más humanos, sino para evaluar determinados perfiles dentro de escenarios.
	
	Tras la investigación, se proponen tres líneas de trabajo futuras: (i) la aplicación de las técnicas estudiadas a simuladores que modelen escenarios más realistas para evaluar más aspectos de comportamiento, tanto para agentes modelados como para conductores reales inmersos en estos entornos, (ii) el modelado no sólo de comportamientos de conductor sino de otros agentes (e.g. peatones), o de aspectos como las emisiones de vehículos para diferenciarlos de modelos de consumo genéricos, y (iii) la aplicación de arquitecturas temporales de \ac{ann} para contrastar si éstas son capaces de aprender de una manera más humana situaciones tales como velocidad de aproximación de objetos o factores de impaciencia  a partir de los datos reales de conducción.
\end{fullwidth}