\cleardoublepage
\thispagestyle{empty}
\chapter*{Resumen}
\begin{fullwidth}
	El tráfico es un sistema complejo cuyo estudio requiere, o bien la inmersión en el sistema real o bien su simulación. Si el objetivo es una mayor fidelidad en los comportamientos o evaluar un elemento concreto de este entorno, la opción es mediante el uso de micro-simuladores. En general este tipo de simuladores están basados en arquitecturas de \Acrfullpl{mas} donde cada elemento es modelado por separado. El incremento de los recursos computacionales nos permite aumentar la complejidad de las simulaciones, pero los agentes que modelan a los conductores se suelen basar en funciones más o menos complejas que responden poco o nada al comportamiento real de un conductor. Esto provoca que dichos modelos se adapten poco al comportamiento de un conductor en concreto.
	
	Por otro lado, estamos viviendo una época donde los avances de la \Acrfull{ci} en predicción, clasificación y modelización están siendo abrumadores. Esto hace pensar que la aplicación de estas técnicas en agentes conductores pueda hacer que se comporten \enquote{mejor} (en el sentido \enquote{más humano}), permitiendo un mejor análisis del tráfico y sus componentes.
	
	Esta tesis explora cómo la \acrlongsp{ci}, aplicada a la modelización de conductores reales, puede mejorar el desempeño de las simulaciones de tráfico, en el sentido de hacer más reales sus resultados. Concretamente, se centra en técnicas basadas en \Acrfullpl{ann} y \Acrfullpl{fcs} para, a partir de datos de conductores reales, estudiar de qué manera las diferentes técnicas se comportan a la hora de reproducir comportamientos humanos en micro-simuladores.
	
	El estudio se centra en dos características existentes en este tipo de simulaciones: el comportamiento longitudinal y el cambio de carril. Se concluye que las técnicas basadas en \acrlongplsp{ann} son claramente superiores a la hora de modelar el comportamiento en estas tareas, y dentro de ellas, las \Acrfullpl{cnn} son idóneas frente a los \Acrfullpl{mlp} interpretando el entorno del vehículo.
	
	Las implicaciones de estos resultados son variadas. Aún con las limitaciones que nos imponen los micro-simuladores, es posible modelar estos dos comportamientos a partir de datos reales con precisión suficiente como para decir que los agentes se comportan de manera más humana. Además, estos comportamientos son dependientes del perfil del que fueron obtenidos los datos, por lo que es posible su uso no sólo para modelar comportamiento más humanos, sino para evaluar determinados perfiles dentro de escenarios.
	
	Tras la investigación, se proponen tres líneas de trabajo futuras: (i) aplicar las técnicas estudiadas a simuladores que modelen escenarios más realistas para evaluar más aspectos de comportamiento, tanto para los modelos como para conductores reales, (ii) modelar no sólo comportamientos de conductor sino otros agentes (e.g. peatones), o aspectos como emisiones de vehículos para diferenciarlos de modelos de consumo genéricos, y (iii) aplicar arquitecturas temporales de \acrlongplsp{ann} para comprobar si éstas son capaces de aprender de una manera más humana situaciones tales como velocidades de aproximación o factores de impaciencia a partir de los datos reales de conducción.
\end{fullwidth}