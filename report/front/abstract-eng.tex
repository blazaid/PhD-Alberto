\cleardoublepage
\thispagestyle{empty}
\chapter*{Abstract}
\begin{fullwidth}
	Traffic might be conceived as a complex system, the study of which requires immersion in the real system or its simulation. If the goal is to obtain greater behavior trustworthiness or to evaluate a specific element in this environment, the better choice is the use of micro-simulators, commonly based in \Acrlongpl{mas} architectures, in which each element is modelled separately. The increase in computational resources allows us to enhance the complexity of the simulations over time, but modelled driver agents are typically based on functions, more or less complex, that which have nothing to do with a true driver’s behaviour. 
	
	This causes that those models does not adapt to a particular driver. On the other hand, we are living in a thriving era with regard to \Acrlong{ci} progress, such as prediction, classification and modelization. This makes us wonder whether the application of this techniques in driver agents could yield better results (\textit{\enquote{better}} as \textit{\enquote{more human}}), allowing us to obtain a better traffic and driver agents analysis.
	
	The present thesis explores how real drivers modelled \acrlong{ci} could improve traffic simulation making their results more close to reality as possible. Specifically, it focuses on \Acrlong{ann} based techniques and \Acrlongpl{fcs}. It will study how different techniques mimic human behaviour inside micro-simulators, based on real driver’s data.
	
	The study focuses into two existent characteristics inside this type of simulations: the longitudinal and lane-changing behaviours. The study concludes that \acrlong{ann} based techniques are far superior in modelling human behaviour. In lane-changing the best among them are \acrlongpl{cnn}, which are better in perceiving the surrounding environment patterns than \acrlongpl{mlp}.
	
	The implications of these findings are varied. Micro-simulators have some limitations, but it is possible to accurately modellate those behaviours based on real data in order to agents act more humanly. In addition, those behaviours are driver profile dependents, consequently it is possible to use them not only for modelling more human behaviour, but also to evaluate those profiles inside certain situations.
	
	After the investigation, three future lines of work are proposed: (i) To apply studied techniques into more realistic scenarios to evaluate other driver behaviour aspects, both for simulated and real drivers inmersed on these environments, (ii) Other agents modelation (e.g. pedestrian) or other aspects such as car emissions, and (iii) The \acrlong{ann} temporal architecture application to know if they are able to learn more as a human in situations such objects approximation or impatience factors based on real driver data.
\end{fullwidth}