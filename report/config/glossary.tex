\makeglossaries
\renewcommand*{\acronymname}{Abreviaturas}

% Glossary terms
\newglossaryentry{aorta}{
	name=AORTA,
	description={(\textit{Approximately Orchestrated Routing and Transportation Analyzer}). Entorno de microsimulación multiagente de tráfico desarrollado en el departamento de Ciencias de la Computación de la Universidad de Austin (Texas). Url: \url{http://www.aorta-traffic.org/}}
}
\newglossaryentry{autoencoder}{
	name=autoencoder,
	plural=autoencoders,
	description={Técnica de aprendizaje para reducir el tamaño de un vector de entrada transformándolo en una representación de ésta en una menor dimensión. En \acrshort{ml} se utilizan tanto para reducción de dimensión de entradas, recuperación de datos a partir de vectores de características o reducción de ruido en señales entre otros}
}

\newglossaryentry{car-following}{
	name=car-following,
	description={Modelo de control de conductor longitudinal que basa su comportamiento en el del vehículo del mismo carril que le precede}
}
\newglossaryentry{dai}{
	name={Inteligencia Artificial Distribuida},
	description={Rama de la \gls{ai} donde se estudian las técnicas de aplicación, coordinación y distribución de conocimiento en un entorno multiagente},
}
\newglossaryentry{dbn}{
	name={DBN},
	description={\textit{Deep Belief Network} o Redes Profundas de Creencia, se tratan de modelos probabilísticos para aprender representaciones jerárquicas de datos de manera no supervisada. Están compuestas de múltiples capas de \acrshortpl{rnn} conectadas entre si dos a dos},
}
\newglossaryentry{esys}{
	name={Sistema Experto},
	description={Sistema que emula a un humano en la toma de decisiones de un dominio en el que es experto},
	plural={Sistemas Expertos}
}
\newglossaryentry{framework}
{
	name=framework,
	description={Conjunto estandarizado de herramientas, conceptos, normas y metodología para solucionar problemas de características similares de una manera organizada}
}
\newglossaryentry{free-flow}{
	name=free-flow,
	description={Modelo de control de conductor longitudinal que genera su comportamiento de acuerdo al estado de la vía cuando no hay vehículo que le preceda}
}
\newglossaryentry{gpl}{
	name=GPL,
	description={(\textit{General Public License}). Licencia de software que garantiza las libertades del software. Asegura que cualquier versión, extensión o software derivado de éste permanecerá siendo software libre. Su última versión es la 3.0. Url: \url{https://www.gnu.org/licenses/gpl-3.0.html}}
}
\newglossaryentry{gpu}{
	name=GPU,
	description={(\textit{Graphics Processing Unit}). Componente similar a una CPU especializado en el cálculo de operaciones sobre matrices para el procesamiento gráfico en videojuegos, también usada para el trabajo con grandes cantidades de datos que requieren ese tipo de operaciones (e.g. \acrlongsp{ml})}
}
\newglossaryentry{hbefa}{
	name=HBEFA,
	description={\textit{The Handbook Emission Factors for Road Transport} es una guía que propone unas tablas de emisiones específicas en \SI{}{\gram\per\kilo\meter} para todas las categorías actuales de vehículos de carretera (concretamente automóviles de pasajeros, vehículos ligeros, vehículos pesados, autobuses y motocicletas).}
}
\newglossaryentry{hc}{
	name=Hard Computing,
	description={Forma de denominar a la computación clásica en contraposición al término \gls{sc}}
}
\newglossaryentry{matsim}{
	name=MatSIM,
	description={(\textit{Multi-Agent Transport Simulation}). Software de microsimulación multiagente desarrollado en la ETH Zürich. Url: \url{http://matsim.org}}
}
\newglossaryentry{mitsim}{
	name=MitSIM,
	description={(\textit{MIcroscopic Traffic SIMulator}). Software de microsimulación desarrollado por el laboratorio de sistemas inteligentes de transporte del MIT. Url: \url{https://its.mit.edu/software/mitsimlab}}
}
\newglossaryentry{movsim}{
	name=MovSim,
	description={(\textit{Multi-model Open-source Vehicular-traffic Simulator}). Software de microsimulación que implementa tanto modelo multiagente y como modelo basado en \glsentryshortpl{ca}. Url: \url{http://www.movsim.org/}}
}
\newglossaryentry{nmea}{
	name=NMEA,
	description={(\textit{National Marine Electronics Association}). Nombre que recibe la una especificación de formato de mensajes propuesta uy desarrollada por la organización de la que toma el nombre}
}
\newglossaryentry{osi}{
	name=OSI,
	description={Estándar de comunicaciones en red desarrollado por la ISO en 1980 para servir como modelo de referencia para la implementación de protocolos de red basados en arquitectura en capas}
}
\newglossaryentry{python}{
	name=Python,
	description={Lenguaje de programación \glsentryshort{oss}. Url: \url{https://www.python.org/}}
}
\newglossaryentry{rmse}{
	name=RMSE,
	description={
		\textit{Root Mean Squared Error}, una métrica para el cálculo del error entre dos conjuntos de medidas sobre los mismos elementos y definida como
		$$
		\sqrt{\frac{\sum_{i=1}^{n}(y_i - \hat{y_i})}{n}}
		$$
		Siendo $y_i$ e $\hat{y_i}$ cada una de las $n$ medidas tomadas sobre el mismo conjunto
	}
}
\newglossaryentry{sc}{
	name={Soft Computing},
	description={Paradigma de computación por el que se hace uso de técnicas de resolución de problemas que manejan información incompleta, inexacta o con ruido}
}
\newglossaryentry{som}{
	name={SOM},
	description={\textit{Self-Organizing Maps} o Mapas Autoorganizados, son un tipo de \acrshort{ann} que aprenden una representación topológica de los datos de entrada de una forma no supervisada}
}
\newglossaryentry{sumo}{
	name=SUMO,
	description={(\textit{Simulation of Urban MObility}). Entorno de micro y mesosimulación multiagente desarrollado por el instituto de sistemas de transporte del DLR (Centro Aeroespacial Alemán). Url: \url{http://www.dlr.de}}
}
\newglossaryentry{tcpip}{
	name=TCP/IP,
	description={Protocolo de comunicación desarrollado en 1970 para permitir la comunicación entre dispositivos a través de una red de comunicaciones}
}
\newglossaryentry{traas}{
	name=TraaS,
	description={(\textit{TraCI as a Service}). \glsentryshortpl{api} basado en \glsentryshortpl{soap} para interactuar con el simulador \gls{sumo} cuando está funcionando en modo servidor. Url: \url{http://traas.sourceforge.net/cms/}}
}
\newglossaryentry{traci}{
	name=TraCI,
	description={(\textit{Traffic Control Interface}). Término que sirve tanto para denominar al protocolo de comunicación ofrecido por \gls{sumo} en modo servidor para la interacción remota con la simulación como para denominar a la librería desarrollada para abstraer dicho protocolo cuando se trabaja desde \gls{python}. Url: \url{http://www.sumo.dlr.de/wiki/TraCI}}
}
\newglossaryentry{torcs}{
	name=TORCS,
	description={(\textit{The Open Racing Car Simulator}). Software de simulación creado en un principio como videojuego de carreras y que ha evolucionado hacia plataforma de simulación de técnicas de \glsentrylongsp{ai} aplicadas a la conducción. Url: \url{http://torcs.sourceforge.net/}}
}

% Acronyms
\newacronym{aasim}{AASIM}{Autonomous Agent SIMulation Package}
\newacronym{acl}{ACL}{Agent Communications Language}
\newacronym[longsp=Sistema Avanzado de Ayuda a la Conducción,longplsp=Sistemas Avanzados de ayuda a la Conducción, longplural=Advanced Driver Assistance Systems]{adas}{ADAS}{Advanced Driver Assistance System}
\newacronym[
	longsp=inteligencia artificial
]{ai}{AI}{artificial intelligence}
\newacronym[
	longsp=red neuronal artificial,
	longplsp=redes neuronales artificiales,
	longplural=artificial neural networks
]{ann}{ANN}{artificial neural network}
\newacronym{api}{API}{Application Programming Interface}
\newacronym{bdi}{BDI}{Belief-Desire-Intention}
\newacronym{bsd}{BSD}{Berkeley Software Distribution}
\newacronym[
	longsp=autómata celular,
	longplsp=autómatas celulares,
	longplural=cellular automata
]{ca}{CA}{cellular automaton}
\newacronym{can}{CAN}{Controller Area Network}
\newacronym[longsp=Procesamiento Complejo de Eventos,longplsp=Procesamientos Complejos de Eventos,longplural=Complex Event Processings]{cep}{CEP}{Complex Event Processing}
\newacronym[longsp=Computación Evolutiva]{cev}{EC}{Evolutionary Computation}
\newacronym[
	longsp=inteligencia computacional
]{ci}{CI}{computational intelligence}
\newacronym[
	longsp=red convolucional,
	longplsp=redes convolucionales,
	longplural=convolutional neural networks
]{cnn}{CNN}{convolutional neural network}
\newacronym{coa}{COA}{centroid of area}
\newacronym{dlc}{DLC}{Discretional Lane Change}
\newacronym{dvu}{DVU}{Driver-Vehicle Unit}
\newacronym{dvo}{DVO}{Driver-Vehicle Object}
\newacronym[
	longsp=sistema de control borroso,
	longplsp=sistemas de control borroso,
	longplural=fuzzy control systems
]{fcs}{FCS}{fuzzy control system}
\newacronym[
	longsp=lógica borrosa
]{fl}{FL}{fuzzy logic}
\newacronym{flowsim}{FLOWSIM}{Fuzzy LOgic motorWay SIMulation}
\newacronym[
	longsp=algoritmo genético,
	longplsp=algoritmos genéticos,
	longplural=genetic algorithms
]{ga}{GA}{genetic algorithm}
\newacronym{gps}{GPS}{Global Positioning System}
\newacronym[longsp=Modelo Oculto de Markov,longplsp=Modelos Ocultos de Markov,longplural=Hidden Markov Models]{hmm}{HMM}{Hidden Markov Model}
\newacronym{idm}{IDM}{Intelligent Driver Model}
\newacronym{imu}{IMU}{Intertial Measurement Unit}
\newacronym{its}{ITS}{intelligent transport system}
\newacronym{kqml}{KQML}{Knowledge Query and Manipulation Language}
\newacronym{soap}{SOAP}{Simple Object Access Protocol}
\newacronym{ghr}{GHR}{Gazis-Herman-Rothery}
\newacronym{gm}{GM}{Generalized Model}
\newacronym{insia}{INSIA}{Instituto Universitario de Investigación del Automóvil}
\newacronym[
	longsp=Reconocimento del Entorno Intravehicular,
	longplsp=Reconocimientos del Entorno Intravehiculares,
	longplural=Intra-vehicular Context Awarenesses
]{ivca}{IvCA}{Intra-vehicular Context Awareness}
\newacronym{lda}{LDA}{Latent Dirichlet Allocation}
\newacronym{lidar}{LiDAR}{Light Detection and Ranging}
\newacronym{lolimot}{LOLIMOT}{LOcal LInear MOdel Tree}
\newacronym{lrn}{LRN}{local response normalization}
\newacronym{lstm}{LSTM}{Long-Short Term Memory}
\newacronym[
	longsp=sistema multiagente,
	longplsp=sistemas multiagente,
	longplural=multiagent systems
]{mas}{MAS}{multiagent system}
\newacronym[
	longsp=perceptrón multicapa,
	longplsp=perceptrones multicapa,
	longplural=multilayer perceptrons
]{mlp}{MLP}{multilayer perceptron}
\newacronym[
	longsp=aprendizaje automático
]{ml}{ML}{machine learning}
\newacronym{mlc}{MLC}{Mandatory Lane Change}
\newacronym[longsp=Procesamiento de Lenguaje Natural]{nlp}{NLP}{Natural Language Processing}
\newacronym{osrf}{OSRF}{Open Source Robotics Foundation}
\newacronym[longsp=Software Abierto]{oss}{OSS}{Open Source Software}
\newacronym{relu}{ReLU}{Rectified Linear Unit}
\newacronym[
	longsp=red neuronal recurrente,
	longplsp=redes neuronales recurrentes,
	longplural=recurrent neural networks
]{rnn}{RNN}{recurrent neural network}
\newacronym[
	longsp=sistema de recomendación,
	longplsp=sistemas de recomendación,
	longplural=recommender systems
]{rs}{RS}{recommender system}
\newacronym{ros}{ROS}{Robot Operating System}
\newacronym[
	longplural=Support Vector Machines
]{svm}{SVM}{Support Vector Machine}
\newacronym{v2i}{V2I}{Vehicle-to-Infraestructure}
\newacronym{v2v}{V2V}{Vehicle-to-Vehicle}