% Glossary terms
\newglossaryentry{dai}{
	name={Inteligencia Artificial Distribuida},
	description={Rama de la \gls{ai} donde se estudian las técnicas de aplicación, coordinación y distribución de conocimiento en un entorno multiagente.},
}

\newglossaryentry{esys}{
	name={Sistema Experto},
	description={Sistema que emula a un humano en la toma de decisiones de un dominio en el que es experto.},
	plural={Sistemas Expertos}
}

\newglossaryentry{fcs}
{
	name=Sistema de Control Difuso,
	description={Sistema para realizar tareas que basa su funcionamiento en inferencia difusa. En general es sinónimo de \gls{fis}, sólo que remarcando el objeto de su funcionamiento (tareas de control).}
}

\newglossaryentry{fis}
{
	name=Sistema de Inferencia Difusa,
	description={Sistema que basa su funcionamiento en la aplicación de reglas lógicas con un motor de inferencia basado en la teoría de la lógica difusa.}
}

\newglossaryentry{fl}
{
	name=Lógica Difusa,
	description={Extensión de la lógica tradicional donde los valores no son únicamente verdaderos ($1$) o falsos ($0$) sino que pueden tomar un grado de verdad o falsedad en todo el intervalo $[0, 1]$.}
}

\newglossaryentry{hc}
{
	name=Hard Computing,
	description={Forma de denominar a la computación clasíca en contraposición al término \gls{sc}.}
}

\newglossaryentry{mas}
{
	name=Sistema Multiagente,
	description={Sistema que emplea al menos dos agentes que interactúan entre sí para solucionar el problema para el que está diseñado.}
}

\newglossaryentry{mlp}
{
	name=Perceptrón Multicapa,
	plural=Perceptrones Multicapa,
	description={Topología de red neuronal \textit{feed-forward} donde las redes están organizadas en capas de tal manera que todas las neuronas de una capa están conectadas con todas las siguientes.}
}

\newglossaryentry{sc}
{
	name={Soft Computing},
	description={Paradigma de computación por el que se hace uso de técnicas de resolución de problemas que manejan información incompleta, inexacta o con ruido.}
}

% Acronyms
\newacronym{acl}{ACL}{Agent Communications Language}
\newacronym{ai}{IA}{Inteligencia Artificial}
\newacronym[\glslongpluralkey={Redes Neuronales Artificiales}]{ann}{RNA}{Red Neuronal Artificial}
\newacronym{ci}{IC}{Inteligencia Computacional}
\newacronym[\glslongpluralkey={Redes neuronales convolucionales}]{cnn}{CNN}{Red neuronal convolucional}
\newacronym[\glslongpluralkey={Algoritmos Genéticos}]{ga}{AG}{Algoritmo Genético}
\newacronym{kqml}{KQML}{Knowledge Query and Manipulation Language}
\newacronym{lstm}{LSTM}{Long-Short Term Memory}
\newacronym{ml}{ML}{Machine Learning}
\newacronym{nlp}{NLP}{Natural Language Programming}
\newacronym{relu}{ReLU}{Rectified Linear Unit}
\newacronym{rs}{RS}{Recommender Systems}
\newacronym{som}{SOM}{Mapas Auto-Organizados}