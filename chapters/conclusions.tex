\chapter{Conclusiones}
\label{ch:conclusions}

¿Pierde rendimiento el sistema cuando se aplican los modelos a escenarios significativamente diferentes de los escenarios de test? Si sí, un trabajo futuro y algo para escribbir en conclusiones sería hablar de este defecto y de cómo subsanarlo.

\section{Aportaciones}
\label{ch:conclusions:contributions}

\section{Fururas líneas de investigación}
\label{ch:conclusions:future-work}

¿A lo mejor se podría tirar por el campo de las V2X desde esta tesis?

¿Entrar en el tema de la mesosimulación?

Intersection model (lo he visto nombrar por primera vez en http://elib.dlr.de/89233/1/SUMO\_Lane\_change\_model\_Template\_SUMO2014.pdf. Habrán creado también un concepto así rotondas?

No sé, pero me parece lógico tratar de realizar una disocuiación entre vehículo y conductor en lugar de contemplar el binomio vehículo/conductor como uno sólo. Es más, creo que resultaría interesante evaluar comportamientos de conductores sobre diferentes tipologías de vehículos. La librería de todas formas debería soportar esta disociación (y se debería indicar).

Hemos dejado fuera comportamientos interesantes de estudiar: cruces, rotondas (Driving behavior at a roundabout: Av hierarchical Bayesian regression analysis), ...