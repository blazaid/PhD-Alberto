\chapter{Modelos de comportamiento}
\label{ch:sota-behavior-models}

Detección de patrones de eficiencia y agresividad de subyacen en los comportamientos de éstos
Estudio de la efectividad de los sistemas de asistencia para mejorar la eficiencia de conducción
Estudio de los sistemas de asistencia para analizar el comportamiento del conductor.

¿A qué nos referimos cuando hablamos del "comportamiento"? Según el diccionario de la real academia, \enquote{comportamiento} se define como \textit{manera de comportarse}, siendo \enquote{comportar} \textit{actual de una manera determinada}. Sin embargo, no queda claro qué factores constituyen el comportamiento al volante.

Atendiendo a la literatura, \TODO un par de párrafos para hablar de comportamiento

¿Cómo se define el comportamiento de un conductor?

En~\cite{michon1985critical} describen la conducción como una trea separada en $3$ niveles jerárquicos, el de estrategia, el de maniobra y el de control. En la figura~\ref{fig:three-levels-of-human-driving}. se ve mejor.
\begin{figure}
	\centering
\smartdiagram[descriptive diagram]{
	{Estrategia, {Planificación de la ruta, decisiones generales, \ldots}},
	{Maniobra, {Control del vehículo, decisiones de cambio de carril, \ldots}},
	{Control, {Cambio de marcha, presión sobre el acelerador, \ldots}},
}
	\caption{Esta es una figura de mierda (porque no se ve lo que se quiere mostrar) pero que me demuestra que con cuatro imágenes me pongo en las 40 páginas sin pestañear. La idea es cambiarla por otra más como la figura 2 de~\cite{michon1985critical}. Con algún que otro color bonico a poder ser.}
	\label{fig:three-levels-of-human-driving}
\end{figure}

SUMO usa (al menos así lo indican en el paper del 2002) el modelo Gipps\cite{krajzewicz2002sumo}. No sé si ellos han hecho una extensión del modelo o están referenciando la extensión y ellos sólo la implementan. En el paper del 2012 citan que el modelo car-following que usan por defecto es el desarrollado por Stefan Krauß\cite{jin2016evaluation}, debido a su simplicidad y su velocidad de ejecución.

El modelo ha probado ser válido, pero tiene algunos defectos, por lo que existe un API para implementar otros modelos. En la actualidad están incluidos en el sistema los modelos IDM\cite{treiber2000congested} (\textit{Intelligent Driver Model}), el modelo de tres fases de Kerner\cite{kerner2008testbed} y el modelo de Wiedemann\cite{wiedemann1974simulation}.

La práctica totalidad de modelos para el comportamiento de conductores en tráfico se basan en una estructura jerárquica de control\cite{michon1985critical}. Por lo que veo este paper es de 1985, así que afirmar eso 30 años después es una barrabasada. Hay que buscar algo más para apoyar eso o para indicarlo como una anécdota.

En el paper \cite{al2001framework} describen un framework para la modelización de comportamiento de conductores dentro de simuladores.

En~\cite{tang2014new} los autores proponen un modelo de car following teniendo en cuenta comunicación entre vehículos.

\TODO Hoy he descubierto que hay una cosa que se llama naturalistic traffic study o naturalistic traffic observation (\gls{nds}) que es un estudio naturalístico, es decvir, observar en el mundo real con detalle y extraer datos de ello. A lo mejor hablando así queda como más profesional :D.

Los \gls{nds} basan su funcionamiento en la captura masiva de datos de conducción, normalmente involucrando una gran cantidad de sensores, para analizar el comportamiento del conductor, las características del vehículo, la vía, etcétera. La cantidad de sensores y la velocidad de captura hacen que la tarea de analizar y extraer conclusiones  sea una tarea prácticamemte imposible para un humano, por lo que es necesario el uso de técnicas de análisis de datos que suelen recaer en los campos de la estadística y del aprendizaje automático.

En~\cite{sekizawa2007modeling} describen modelos supervisados offline para capturar el comportamiento del conductor basados en auto-regresión a trozos. Más adelante lo extienden en~\cite{terada2010multi}, aunque los datos de entrenamiento son extraídos directamente de simulaciones, no del \enquote{mundo real\textsuperscript{TM}}.

En~\cite{bando2013unsupervised} describen otro modelo no supervisado offline basado en un modelo bayesiano no paramétrico para la clusterización, combinado con un LDA (Latent Dirichlet Allocation, sea lo que sea esto) para la clusterización a más alto nivel. (\TODO ¿este método usa datos reales?)

Estos trabajos (\cite{sekizawa2007modeling}, \cite{terada2010multi} y \cite{bando2013unsupervised}) tienen la desventaja de ser computacionalmente muy costosos y con poca precisión en el caso de variar mucho los escenarios de entrenamiento y de test.

En~\cite{maye2011bayesian} se presenta un modelo online donde se infiere el comportamiento del conductor haciendo uso de una IMU (Intertial Measurement Unit) y una cámara. Primero de la IMU se sacan datos que se separan en fragmentos para luego relacionarlos con las imágenes obtenidas de la cámara. (\TODO ¿Hacia dónde mira la cámara?). Los modelos propuestos en~\cite{johnson2011driving} y~\cite{van2013driver} también se apoyan en el funcionamiento de clasificar las segmentaciones de una IMU, pero con técnicas distintas y sin cámara (\TODO Que digo yo, ¿qué clasifican exactamente? es más, dependiendo de qué clasifican, para qué vale la camarita del~\cite{maye2011bayesian}?). Además hacen uso de señales externas y umbrales de activación para hacer más efectiva la clasificación (\TODO corroborar).

En el artículo~\cite{bender2015unsupervised} se usa también un modelo no supervisado online con una aproximación bayesiana para identificar los puntos de cambio sin depender de parámetros externos (e.g. umbrales o señales). Se basa también en (1) segmentar los datos de conducción y (2) asignar estos datos a clases que se corresponde n con comportamientos de conducción. Tiene la ventaja de ser más eficiente y rubusto que los anteriores.

La idea de estos métodos desde el~\cite{sekizawa2007modeling} hasta el~\cite{bender2015unsupervised} creo que es el de un sistema que traduzca datos en crudo a datos de más alto nivel. Esto es debido a que la cantidad de datos que se pueden generar en un sólo coche (no digamos ya una flota de ellos) es tan grande que para determinados sistemas disponer de información de más alto nivel haría más sencillo su desempeño (por ejemplo, \gls{adas} que funcionasen con datos de \enquote{adelantando} que sus valores de giro, aceleración en una ventana temporal).

En~\cite{satzoda2013towards}, haciendo uso de la información combinada de bus CAN, cámaras, GPS e información digitalizada el mapa donde se circula se determina una amplio abanico de información crítica en diferentes condiciones de la carretera. La información que sacan es: número de cambios de carril a la derecha, a la izquierda, tiempo en autopista y carretera urbana, distancia recorrida, velocidades medias en autopista y urbano, paradas, giros a la derecha, a la izquierda, incormporaciones y salidas de autopsta, tiempo gastado en un sólo carril, curvas a la izquierda, curvas a la derecha y distancia media al carril central

En~\cite{al2001framework} describen un framework para la modelización de comportamiento de conductores dentro de simuladores. Se basa en cuatro unidades de funcionamiento interconectadas, la de percepción (percibe el entorno en términos locales y globales), la de emoción (cómo responde emocionalmete al entorno), la de decision-making (investiga posibles acciones que podrían servir a las necesidades del módulo emocional) y la de decision-implementation (intenta implementar las decisiones cuando emergen condiciones de tráfico lo suficientemente seguras para llevarlas a cabo). Tengo que volver a leerlo después de hacer una primera introducción en el tema de agentes, porque me parece poner nombrecitos a un tipo de agente que funciona de esa misma manera, pero lo mismo no.

En~\cite{terroso2015complex} analizan lo que ellos denominan el concepto del \gls{ivca}, lo que viene a ser el contexto definido \textbf{dentro} del vehículo, llegando a intentar predecir no sólo el número de ocupantes (ese es un problema sobradamente superado) sino la tipología o clase de ocupante (e.g. niños, adultos, ... \TODO buscar cuáles son las tipologías). Para ello hacen uso de un \gls{cep}\footnote{Un \gls{cep} es un método por el cual se lee un flujo de información compuesta de flujos de distintas fuentes (de ahí el \textit{complex}) para detectar eventos o patrones que pueden indicar la presencia de situaciones a analizar lo más rápido posible.} para procesar los datos de los diferentes sensores del vehículo y así detectar y analizar patrones característicos.

El artículo~\cite{munoz2010human} es muy interesante ya que para la competición \textit{2010 Simulated Car Racing Championship} desarrollaron un controlador no para minimizar el tiempo en realizar las carreras, sino para hacerlo lo más parecido posible a cómo se comporta un humano. Para ello hicieron uso de redes neuronales para calcular trayectorias y de un proceso de aprendizaje a partir de información extraída de un conductor real en el simulador \gls{torcs}.

Aunque \gls{torcs} es usado como entorno de simulación en diversos concursos e investigaciones, se trata de un juego, y es que los juegos son un sandbox perfecto como entorno de simulación, ya que presentan una abstracción del dominio sobre el que trabajar. Otros trabajos en esta línea (entrenamiento de redes neuronales en este simulador) son los de~\cite{munoz2009controller} y~\cite{van2009robust}, el primero entrenando perceptrones multicapa (\TODO verificar) con un backpropagation haciendo uso de un dataset proporcionado por un bot y otro haciendo uso de un algoritmo evolutivo multiobjetivo para optimizar la red de acuerdo a un conjunto de datos proporcionado por un conductor real. Sin embargo, este tipo de modelos se encuentran más cercanos al nivel de control que al nivel de maniobra descritos en la figura~\ref{fig:three-levels-of-human-driving}.

En~\cite{van2013driver} hacen uso de os sensores de inercia de un coche (a saber qué coche tienen) para construir un perfil de conductor. Concluyen que frenar y girar caracterizan mejor a los conductores frente a acelerar.

\section{lane change y car following}

MOBIL --> http://www.traffic-simulation.de/MOBIL.html, http://www.mtreiber.de/publications/MOBIL\_TRB.pdf y http://link.springer.com/chapter/10.1007\%2F978-3-540-77074-9\_19\#page-1

\section{car-following}

En un modelo de tipo car-following los vehículos están representados por una tupla del estilo $(x_n, v_n, a_n, t_n)$ donde $x_n$ es la localización espacial, $v_n$ la velocidad, $a_n$ la aceleración y $t_n$ el momento en el tiempo del vehículo $n$. El modelo es una serie de reglas o ecuaciones que actualizan estos valores a lo largo de $t_n$. Cambiar la definición a algo que sea más claro o citar esta que sale en~\cite{Aghabayk2015}.

En~\cite{Aghabayk2015} realizan un estado de la cuestión en materia de modelos de tipo car-following. El autor divide en dos categorías los modelos, los clásicos y los basados en inteligencia artificial.

\begin{enumerate}
	\item \textbf{Clásicos}. Son aquellos modelos donde la acción que realiza el vehículo viene determinada por una o más ecuaciones en función de las acciones que toma el vehículo delantero. Se pueden considerar limitados porque se centran en los resultados del comportamiento y no en el comportamiento en sí. Además se basan en que determinadas variables (e.g. tiempo de reacción) son las mismas para cada conductor.
	\begin{enumerate}
		\item \textbf{De estímulo-respuesta}. La respuesta está directamente relacionada con el estímulo creado por el comportamiento del coche de delante, generalmente una respuesta de tipo aceleración o deceleración aplicada con un pequeño retardo $t$. Los métodos de esta clase de modelo ssuelen ser sencillos, pero tiene dos problemas principales, (i) que los modelos no capturan los comportamintos de diferentes tipos de conductores o vehículos y (ii) supone que el conductor es capaz de observar los más mínimos cambios en el coche delantero cuando esto en realidad no es así (4-7).
		\item \textbf{De distancia segura}. También denominados de prevención de colisión, basan su funcionamiento en dejar una distancia segura entre el coche delantero y el actual. El principal problema de estos modelos es que no se corresponden con la realidad, ya que un conductor obtiene información de muchas más fuentes y reaccionan en consecuencia y por tanto el comportamiento exhibido en estas simulaciones no se corresponde con el comportamiento real~\cite{Pipes1953}.
		\item \textbf{De avance deseado}. Similares al anterior, estos modelos basan su funcionamiento en tratar de mantener una distancia entre el parachoces trasero del coche frontal y el coche actual. Como en el caso anterior, este modelo no modela corectamente el comportamiento de un conductor debido a que hay muchos parámnetros que no se miden para determinar la distancia a mantener en cada momento.
		\item \textbf{Psicofísicos}. En este modelo, la suposición es que un conductor es capaz de detectar los cambios de velocidad en el vehículo frontal a partir del cambio en el ángulo visual con éste (8-10).
	\end{enumerate}
	\item \textbf{Basados en \acrlongsp{ai}}. Son todos aquellos modelos que están basados en técnicas del campo de la \acrshort{ai}.
\end{enumerate}

Nosotros nos centramos en \acrshort{ai}, seguramente haya que recortar un poco lo de arriba para nombrarlo por encima. Quizá apuntando a los distintos tipos y ya. Según~\cite{Aghabayk2015}, los modelos basados en \acrlongsp{ai} funcionan en base a dos técnicas principalmente, las \acrlongplsp{ann} y la \acrlongsp{fl}.

\subsection{Modelos basados en lógica difusa}

Los modelos de car following que se basan en la lógica difusa suelen apoyarse en la convicción de que la información que maneja el conductor a la hora de tomar decisiones proviene de un análisis no demasiado detallado de la situación que le rodea, y que por tanto que su razonamiento parte de conceptos imprecisos y vagos que llevan a una respuesta no demasiado bien definida.

\cite{Kikuchi1992} fueron los primeros en aplicar lógica difusa a modelos de conducción. Usaron el modelo GHR (\cite{Chandler1958}). Las entradas al modelo eran distancia al coche delantero, diferencia de velocidades, aceleración y deceleración del coche delantero. La aceleración y deceleración se toman como entradas diferentes porque postulaban que el comportamiento antes ambos casos era diferente (aunque creo que esto es equivalente a usar particiones no simétricas de la variable lingüística. Como salida, la aceleración/decelearación del coche actual.

Otros trabajos que trabajan con lógica difusa: (Chakroborty y Kikuchi, 1999, 2003; Das, Bowles, Houghland, Hunn y Zhang, 1999; Gao, Hu, y Dong, 2008; Gonzalez-Rojo, Slama, Pereira y Mora-Camino, 2000, 2002; Hatipkarasulu y Wolshon, 2003; McDonald et al., 1997; Rekersbrink, 1995; Won, Lee, Lee, y Kim, 2007; Zheng y McDonald, 2005, Wu, Brackstone y McDonald, 2003). Ninguno de estos estudios considera diferentes tipologías de vehículos para el car following

Problemas:

\begin{enumerate}
	\item ¿Qué reglas usan los humanos para modelar su comportamiento? Desconocerlas implica modelos no realistas. En (Wu, Brackstone, y McDonald, 2003) intentna suplir este problema con encuestas a conductores.
	\item Los problemas inherentes de los controladores difusos. ¿Cómo validar las funciones de pertenencia? ¿cómo determinar las reglas difusas?
\end{enumerate}

\subsection{Modelos basados en redes neuronales artificiales}

Las redes neuronales se han aplicado mucho sobre el campo de las ITS en general, y sobre la conducción autónoma y el análisis del comportamiento de los conductores (Dougherty, 1995).

En (Fix y Armstrong, 1990) implementaron un controlador basado en redes neuronales para al comportamiento del car following en un microsimulador entrenando dicho modelo previamente con datos extraídos de un conductor en dicho simulador.

En (Dougherty, Kirby y Boyle, 1993) usan redes para determinar el nivel de congestión en la vía.

(Hongfei, Zhicai y Anning, 2003) son los primers en usar datos reales de un coche instrumentado usando el método Five-Wheel-System, que está especificado en su paper de aquella manera y que no me entero de nada. A partir de las entradas correspondientes a velocidad relativa, espacio relativo, velocidad y velocidad deseada (para ello, clasifican al conductor de agresivo, normal, conservador) determinan la aceleración/deceleración del vehículo. No lo aplican a ningún simulador, sólo que los valores se ajustan.

(Panwai and Dia, 2007) redes neuronales usando el dataset de (Manstetten, Krautter y Schwab, 1997) desarrollan una red neuronal para mantener la distancia con el siguiente vehículo. Este modelo sí se evaluó en el simulador AIMSUN, y los resultados muestran una buena correspondiencia entre los datos y la realidad. No replica sin embargo el comportamiento de frenar hasta parar o de acelerar desde parado.

Problemas:

\begin{enumerate}
	\item Es imposible determinar por qué la red funciona como está funcionando.
	\item Los clásicos de los problemas de redes, el no aprendizaje y la especialización.
\end{enumerate}

\subsection{Otras técnicas}

\subsection{Aproximaciones híbridas}

(Li, 2003) y (Ma, 2004) usan aproxmiaciones de fuzzy neural networks y neurofuzzy respectivamente. No proveen sin embargo de documentación y no se investiga la aplicación de estos modelos a microsimuladores de tráfico.

(Aghabayk, Forouzideh y Young, 2013) usan el modelo de árboles lineales locales (LOLIMOT, (Nelles, 2001)) que no deja de ser una aproximación neuro-fuzzy del comportamiento. Intenta incorporar imperfecciones perceptuales en un modelo de car following. El modelo está basado usando datasets reales y los resultados indican que se ajusta lo predicho con la realidad, pero no hay pruebas realizadas en microsimuladores.


\section{lane-changing}

(Gipps, 1986, A Model for the Structure of Lane-Changing Decisions) propone un framework para el problema de cambio de carril que incluye numerosos factores, entre ellos señales de tráfico, tipos de vehículo (e.g. camiones) o \enquote{urgencia} en el camio de carril (e.g. proximidad a una salida o giro). El pricipal problema de las reglas de este modelo es que asumen que el cambio de carril ocurre sin forzar a los vehículos del carril de destino a modificar su comportamiento como disminuir la velocidad o parar.

En (Fritzsche, 1994, A model for traffic simulation) se describe un modelo de microsimulación para analizar cuellos de botella (e.g. un accidente donde se bloquea uno de los carriles). Es un caso típico donde los vehículos no pueden cambiar de carril sin la participación activa del resto de vehículos (colaboración). El modelo lo describe de una maera muy sucinta y no zonsidera comportamiento colaborativo en el cambio de carril.

Yousif and Hunt (1995) developed a microscopic simulation model for the investigation of lane changing behaviour on multi-lane unidirectional roadways. The rules pertaining to the desire and the possibility to change lane are based on similar logic to that described by Gipps (1986). Again, the assumption of the model is that if the available gap in the target lane is smaller than a given acceptable limit, no lane changing will take place. The main concern of the study is the relationship between lane utilisation and traffic flow on dual-carriageway roads under normal flow conditions (i.e. without incidents) and the model is adequate for this purpose. However, it could not produce realistic results when incidents or lane closures affect the flow conditions.

(Barcelo et al., 1996, PETRI: A parallel environment for a real-time traffic management and information system) describen el simulador AIMSUN. El comportamiento de cada vehículo en la simulación es modelado a través de múltiples modelos de comportamiento (e.g. car following, lane changing, gap acceptance). El modelo de cambio de carril es el usado en el modelo de Gipps, aunque el propio simulador permite la modelización de incidentes por lo que debe existir alguna variación del modelo original o un nuevo modelo para ese caso concreto.

(Wagner et al., 1997, Realistic multi-lane traffic rules for cellular automata) describe \enquote{un modelo de microsimulación mínimo para traproducir características macroscópicas en el flujo de tráfico}. El objetivo: definir reglas realísticas para modelar el cambio de carril usando carreteras con múltiples carriles. Para el cambio de carril, describen una serie de reglas para describir \enquote{cuando} cambiar de carril, y una \enquote{regla restrictiva de seguridad} que especifica que un coche que quiere cambiar de carril no moleste al coche de detrás en el carril objetivo. El modelo fue capaz de reproducir de forma satisfactoria las características de uso de los carriles en carreteras de múltiples carriles con diferentes flujos de tráfico sobre condiciones normales (i.e. sin incidentes) de tráfico.

En (Hunt and Lyons, 1994, Modelling of dual-carriageway lane-changing using neural networks) desarrollan un modelo de decisión usando redes neuronales. El modelo funciona a partir de presentarle una entrada visual del entorno alrededor del vehículo que quiere cambiar de carril. Sin embargo, no considera la cooperación entre vehículos.

En (Yang and Koutsopoulos, 1996, A Microscopic Traffic Simulator for Evaluation of Dynamic Traffic Management Systems) presentan el simulador MITSIM, desarrollado por el MIT (creo) en el que se habla específicamente de comportamiento colaborativo en cambio de carriles haciendo uso de lo que denominan "courtesy yielding function" (algo así como función de cesión de paso de cortesía), la cual se usa para para hacer espacio al vehículo que va a incorporarse al carril. Sin embargo, los detalles de dicho proceso no están especificados en el paper.

En (Modelling lane changing and merging in microscopic traffic simulation, 2002) (simulador SITRAS) dicen que según el comportamiento de (Gipps, 1986), el cambio de carril no ocurre nunca en una situación de congestión, y por tanto en este tipo de situaciones el vehículo tiene que forzar su movimiento hacia el carril. Como la interacción entre conductores en ese tipo de maniobras requiere comportamiento complejo de toma de decisiones, éstos pueden ser modelados con técnicas de agentes autónomos. También aquí hablan de los DVOs (driver-vehicle objects). Tienen características individuales como (i) tipo de vehículo, (ii) magnitudes físicos (tamaño, velocidad máxima, ...), (iii) tipo de conduictor y (iv) nivel de conocimiento de la red (porque afecta en la elección de ruta). Tienen un objetivo, llegar del origen al destino tan rápido como puedan. Esto implica un conjunto de decisiones a hacer en intervalos periódicos durante su funcionamiento (i) selección de ruta cada vez que se entra en un nuevo tramo y (ii) cálculo de la aceleración en cada intervalo).

En (Chaib-draa and Levesque, 1996, ) proponen un framework para trabajar con tres tipos diferentes de situaciones (rutina, familiar y no familiar) en sistemas multiagente, demostrando la aplicación en escenarios de microsimulacinó urbana). Su modelo se basa en una estructura jerárquica definida por los niveles de comportamiento humano y de técnicas de razonamiento propuestas por (Rasmussen, 1986, ) (skill-rule-knowledge). El comportamiento basado en habilidades (skills) se refiere a las actividades completamente automatizadas (percepción--ejecución) usadas típicamente en situaciones rutinarias. comportamiento basado en reglas se refiere a situaciones esteorotipadas (percepción--reconocimiento de la situación--planificación--ejecución) aplicable en su mayoría en situacionesfamiliares. El comportamiento basado en el conocimiento se refiere a actividades conscientes que implican trabajo de resolución de problemas y toma de decisiones (perception--reconocimiento de la situación--toma e decisión--planificación de la ejecución) que suelen ser necesarias en situaciones poco familiares. En SITRAS se ven estas diferencias claramente en el cálculo de la aceleración. (i) si no hay ninguna otra restriccinó, llegar a la máxima velocidad es acelerar hasta la máxima velocidad (skill), (ii) si hay ua luz roja más adelante, se va frenando hasta para el coche (rule) y (ii) si se recibe información de coches alrededor (por ejemplo se está incorporando un nuevo coche a nuestro carril) se requiere un conocimiento más complejo (knowledge). Los DVO aquí tienen las siguientes debilidades: no tienen memoria (sólo planean el segundo siguiente de acuerdo a la información actual) y tienen poco contacto directo con los demás DVOs de alrededor (saben del de delante y del de detrás, pero no de los lados).

\section{gap-acceptance}

¿Qué coño es esto?

\section{merging}

Prestar atención al "merging", un caso especial del lane-changing.