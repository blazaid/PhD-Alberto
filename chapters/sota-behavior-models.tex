\chapter{Modelos de comportamiento}
\label{ch:sota-behavior-models}

Detección de patrones de eficiencia y agresividad de subyacen en los comportamientos de éstos
Estudio de la efectividad de los sistemas de asistencia para mejorar la eficiencia de conducción
Estudio de los sistemas de asistencia para analizar el comportamiento del conductor.

¿A qué nos referimos cuando hablamos del "comportamiento"? Según el diccionario de la real academia, \enquote{comportamiento} se define como \textit{manera de comportarse}, siendo \enquote{comportar} \textit{actual de una manera determinada}. Sin embargo, no queda claro qué factores constituyen el comportamiento al volante.

Atendiendo a la literatura, \TODO un par de párrafos para hablar de comportamiento

¿Cómo se define el comportamiento de un conductor?

En~\cite{michon1985critical} describen la conducción como una trea separada en $3$ niveles jerárquicos, el de estrategia, el de maniobra y el de control. En la figura~\ref{fig:three-levels-of-human-driving}. se ve mejor.
\begin{figure}
	\centering
\smartdiagram[descriptive diagram]{
	{Estrategia, {Planificación de la ruta, decisiones generales, \ldots}},
	{Maniobra, {Control del vehículo, decisiones de cambio de carril, \ldots}},
	{Control, {Cambio de marcha, presión sobre el acelerador, \ldots}},
}
	\caption{Esta es una figura de mierda (porque no se ve lo que se quiere mostrar) pero que me demuestra que con cuatro imágenes me pongo en las 40 páginas sin pestañear. La idea es cambiarla por otra más como la figura 2 de~\cite{michon1985critical}. Con algún que otro color bonico a poder ser.}
	\label{fig:three-levels-of-human-driving}
\end{figure}

SUMO usa (al menos así lo indican en el paper del 2002) el modelo Gipps\cite{krajzewicz2002sumo}. No sé si ellos han hecho una extensión del modelo o están referenciando la extensión y ellos sólo la implementan. En el paper del 2012 citan que el modelo car-following que usan por defecto es el desarrollado por Stefan Krauß\cite{jin2016evaluation}, debido a su simplicidad y su velocidad de ejecución.

El modelo ha probado ser válido, pero tiene algunos defectos, por lo que existe un API para implementar otros modelos. En la actualidad están incluidos en el sistema los modelos IDM\cite{treiber2000congested} (\textit{Intelligent Driver Model}), el modelo de tres fases de Kerner\cite{kerner2008testbed} y el modelo de Wiedemann\cite{wiedemann1974simulation}.

La práctica totalidad de modelos para el comportamiento de conductores en tráfico se basan en una estructura jerárquica de control\cite{michon1985critical}. Por lo que veo este paper es de 1985, así que afirmar eso 30 años después es una barrabasada. Hay que buscar algo más para apoyar eso o para indicarlo como una anécdota.

En el paper \cite{al2001framework} describen un framework para la modelización de comportamiento de conductores dentro de simuladores.

En~\cite{tang2014new} los autores proponen un modelo de car following teniendo en cuenta comunicación entre vehículos.

\TODO Hoy he descubierto que hay una cosa que se llama naturalistic traffic study o naturalistic traffic observation (\gls{nds}) que es un estudio naturalístico, es decvir, observar en el mundo real con detalle y extraer datos de ello. A lo mejor hablando así queda como más profesional :D.

Los \gls{nds} basan su funcionamiento en la captura masiva de datos de conducción, normalmente involucrando una gran cantidad de sensores, para analizar el comportamiento del conductor, las características del vehículo, la vía, etcétera. La cantidad de sensores y la velocidad de captura hacen que la tarea de analizar y extraer conclusiones  sea una tarea prácticamemte imposible para un humano, por lo que es necesario el uso de técnicas de análisis de datos que suelen recaer en los campos de la estadística y del aprendizaje automático.

En~\cite{sekizawa2007modeling} describen modelos supervisados offline para capturar el comportamiento del conductor basados en auto-regresión a trozos. Más adelante lo extienden en~\cite{terada2010multi}, aunque los datos de entrenamiento son extraídos directamente de simulaciones, no del \enquote{mundo real\textsuperscript{TM}}.

En~\cite{bando2013unsupervised} describen otro modelo no supervisado offline basado en un modelo bayesiano no paramétrico para la clusterización, combinado con un LDA (Latent Dirichlet Allocation, sea lo que sea esto) para la clusterización a más alto nivel. (\TODO ¿este método usa datos reales?)

Estos trabajos (\cite{sekizawa2007modeling}, \cite{terada2010multi} y \cite{bando2013unsupervised}) tienen la desventaja de ser computacionalmente muy costosos y con poca precisión en el caso de variar mucho los escenarios de entrenamiento y de test.

En~\cite{maye2011bayesian} se presenta un modelo online donde se infiere el comportamiento del conductor haciendo uso de una IMU (Intertial Measurement Unit) y una cámara. Primero de la IMU se sacan datos que se separan en fragmentos para luego relacionarlos con las imágenes obtenidas de la cámara. (\TODO ¿Hacia dónde mira la cámara?). Los modelos propuestos en~\cite{johnson2011driving} y~\cite{van2013driver} también se apoyan en el funcionamiento de clasificar las segmentaciones de una IMU, pero con técnicas distintas y sin cámara (\TODO Que digo yo, ¿qué clasifican exactamente? es más, dependiendo de qué clasifican, para qué vale la camarita del~\cite{maye2011bayesian}?). Además hacen uso de señales externas y umbrales de activación para hacer más efectiva la clasificación (\TODO corroborar).

En el artículo~\cite{bender2015unsupervised} se usa también un modelo no supervisado online con una aproximación bayesiana para identificar los puntos de cambio sin depender de parámetros externos (e.g. umbrales o señales). Se basa también en (1) segmentar los datos de conducción y (2) asignar estos datos a clases que se corresponde n con comportamientos de conducción. Tiene la ventaja de ser más eficiente y rubusto que los anteriores.

La idea de estos métodos desde el~\cite{sekizawa2007modeling} hasta el~\cite{bender2015unsupervised} creo que es el de un sistema que traduzca datos en crudo a datos de más alto nivel. Esto es debido a que la cantidad de datos que se pueden generar en un sólo coche (no digamos ya una flota de ellos) es tan grande que para determinados sistemas disponer de información de más alto nivel haría más sencillo su desempeño (por ejemplo, \gls{adas} que funcionasen con datos de \enquote{adelantando} que sus valores de giro, aceleración en una ventana temporal).

En~\cite{satzoda2013towards}, haciendo uso de la información combinada de bus CAN, cámaras, GPS e información digitalizada el mapa donde se circula se determina una amplio abanico de información crítica en diferentes condiciones de la carretera. La información que sacan es: número de cambios de carril a la derecha, a la izquierda, tiempo en autopista y carretera urbana, distancia recorrida, velocidades medias en autopista y urbano, paradas, giros a la derecha, a la izquierda, incormporaciones y salidas de autopsta, tiempo gastado en un sólo carril, curvas a la izquierda, curvas a la derecha y distancia media al carril central

En~\cite{al2001framework} describen un framework para la modelización de comportamiento de conductores dentro de simuladores. Se basa en cuatro unidades de funcionamiento interconectadas, la de percepción (percibe el entorno en términos locales y globales), la de emoción (cómo responde emocionalmete al entorno), la de decision-making (investiga posibles acciones que podrían servir a las necesidades del módulo emocional) y la de decision-implementation (intenta implementar las decisiones cuando emergen condiciones de tráfico lo suficientemente seguras para llevarlas a cabo). Tengo que volver a leerlo después de hacer una primera introducción en el tema de agentes, porque me parece poner nombrecitos a un tipo de agente que funciona de esa misma manera, pero lo mismo no.

En~\cite{terroso2015complex} analizan lo que ellos denominan el concepto del \gls{ivca}, lo que viene a ser el contexto definido \textbf{dentro} del vehículo, llegando a intentar predecir no sólo el número de ocupantes (ese es un problema sobradamente superado) sino la tipología o clase de ocupante (e.g. niños, adultos, ... \TODO buscar cuáles son las tipologías). Para ello hacen uso de un \gls{cep}\footnote{Un \gls{cep} es un método por el cual se lee un flujo de información compuesta de flujos de distintas fuentes (de ahí el \textit{complex}) para detectar eventos o patrones que pueden indicar la presencia de situaciones a analizar lo más rápido posible.} para procesar los datos de los diferentes sensores del vehículo y así detectar y analizar patrones característicos.

El artículo~\cite{munoz2010human} es muy interesante ya que para la competición \textit{2010 Simulated Car Racing Championship} desarrollaron un controlador no para minimizar el tiempo en realizar las carreras, sino para hacerlo lo más parecido posible a cómo se comporta un humano. Para ello hicieron uso de redes neuronales para calcular trayectorias y de un proceso de aprendizaje a partir de información extraída de un conductor real en el simulador \gls{torcs}.

Aunque \gls{torcs} es usado como entorno de simulación en diversos concursos e investigaciones, se trata de un juego, y es que los juegos son un sandbox perfecto como entorno de simulación, ya que presentan una abstracción del dominio sobre el que trabajar. Otros trabajos en esta línea (entrenamiento de redes neuronales en este simulador) son los de~\cite{munoz2009controller} y~\cite{van2009robust}, el primero entrenando perceptrones multicapa (\TODO verificar) con un backpropagation haciendo uso de un dataset proporcionado por un bot y otro haciendo uso de un algoritmo evolutivo multiobjetivo para optimizar la red de acuerdo a un conjunto de datos proporcionado por un conductor real. Sin embargo, este tipo de modelos se encuentran más cercanos al nivel de control que al nivel de maniobra descritos en la figura~\ref{fig:three-levels-of-human-driving}.

En~\cite{van2013driver} hacen uso de os sensores de inercia de un coche (a saber qué coche tienen) para construir un perfil de conductor. Concluyen que frenar y girar caracterizan mejor a los conductores frente a acelerar.

\section{lane change y car following}

MOBIL --> http://www.traffic-simulation.de/MOBIL.html, http://www.mtreiber.de/publications/MOBIL\_TRB.pdf y http://link.springer.com/chapter/10.1007\%2F978-3-540-77074-9\_19\#page-1