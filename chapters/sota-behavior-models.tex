\chapter{Modelos de comportamiento}
\label{ch:sota-behavior-models}

El objetivo que persigue la simulación de tráfico es hacer cada vez más realistas los modelos generados. En un simulador de tipo \glsentrylongsp{mas} donde cada uno de los agentes modela, entre otros, a conductores, el realismo aumenta cuanto más se parecen el comportamiento\sidenote{
	¿A qué nos referimos cuando hablamos del \textit{comportamiento al volante}? \textit{comportar} (segun la RAE) se define como \textit{actuar de una manera determinada}. Por tanto, comportamiento al volante lo tratamos en este estudio como las acciones o tareas que se ejecutan durante la conducción, es decir, la manera de actuar de un individuo mientras conduce.
} del agente al del conductor real.

Conducir implica la ejecución de múltiples tareas en paralelo, cada una de ellas pertenecientes a un nivel cognitivo. Además, las acciones no están limitadas a la interacción con el vehículo; el conductor ha de tener en cuenta otros factores como, por ejemplo las señales, los peatones o los \glspl{adas}.

\begin{figure}
	\centering
	\begin{tikzpicture}
	\tikzset{
		centered/.style = { align=center, anchor=center },
		arrow/.style = { black!20, arrows={-Triangle} },
		dblarrow/.style = { black!20, arrows={Triangle-Triangle} },
	}
	
	\matrix (m) [
			matrix of nodes,
			column sep      = 2em,
			row sep         = 1em,
			column 1/.style = { nodes = { font=\sffamily\scriptsize, fill=black!20, centered}},
			column 2/.style = { nodes = { font=\sffamily, centered, fill=orange!20, text width=3cm }},
			column 3/.style = { nodes = { font=\sffamily, centered, text width=2.2cm }},
			column 4/.style = { nodes = { font=\sffamily, centered }},
		] {
			& Nivel estratégico & planes generales & $\gg s.$\\
			Entorno & Nivel táctico  & patrones de acción & $s.$ \\
			Entorno  & Nivel de control & comportamientos automáticos & $ms.$ \\
		};
		\draw[arrow] (m-1-2) -> (m-1-3);
		\draw[arrow] (m-2-1) -> (m-2-2);
		\draw[dblarrow] (m-2-2) -> (m-2-3);
		\draw[arrow] (m-3-1) -> (m-3-2);
		\draw[dblarrow] (m-3-2) -> (m-3-3);
		\draw[dblarrow] (m-2-2) -> (m-1-2);
		\draw[dblarrow] (m-3-2) -> (m-2-2);
	\end{tikzpicture}
	\caption{Los tres niveles jerárquicos que describen la tarea de conducción según~\cite{michon1985critical}: \textit{estrategia} (i.e. las decisiones generales), la \textit{maniobra} (i.e. decisiones durante la conducción de más corto plazo) y \textit{control} (i.e. automatismos).}
	\label{fig:three-levels-of-human-driving}
\end{figure}

\cite{michon1985critical} divide en tres los niveles de abstracción de las tareas: el de \textbf{control}, que se ocupa de las tareas de más bajo nivel destinadas éstas a mantener la conducción como son la aceleración o los cambios de marcha, el de \textbf{maniobra} o táctico, donde sus tareas son las encargadas de mantener la interacción con el entorno como los cambios de carril o el control de las señales y demás estímulos externos, y el \textbf{estratégico}, que engloba las tareas de más alto nivel como el razonamiento y la planificación de rutas (ver figura~\ref{fig:three-levels-of-human-driving})\sidenote{Algunos estudios llegan incluso a definir intervalos temporales de tiempo de razonamiento para las tareas de cada nivel. Por ejemplo, en \cite{Alexiadis2004} se establecen los siguientes tiempos: alrededor de $30$ segundos para las tareas del nivel de planificación, de $5$ a $30$ segundos para las tareas de nivel táctico y por debajo de los $5$ segundos para las tareas de control.}.

El comportamiento de un conductor al volante tiene una relación directa con el nivel de abstracción de maniobra o táctico. Ésta se puede concebir como la encargada de planificar acciones a corto plazo para conseguir objetivos a corto plazo. Las tareas de control son automáticas e influyen poco o nada en las tomas de decisión relacionadas con tareas del estilo de cuánto acelerar en esta situación o cuándo cambiar de carril. Las tareas estratégicas estan a un nivel más alto de abstracción (e.g. la ruta a seguir hasta mi destino) y tampoco afectan demasiado al comportamiento en situaciones concretas\sidenote{No obstante algunos trabajos han demostrado que en ocasiones la planificación de la ruta sí afecta a decisiones normalmente asociadas el nivel táctico como por ejemplo la preferencia de un conductor por uno u otro carril de la vía \cite{Wei2000, Toledo2003}.}.

El resto del capítulo introducirá los modelos de comportamiento más conocidos y hará especial hincapié en el estado más reciente de modelos basados en técnicas de la \glsentrylong{ci}\sidenote{
	Otros usos de las técnicas de la \gls{ci} como la caracterización de los conductores no serán explorados. Algunos estudios de interés son \cite{sekizawa2007modeling, terada2010multi} donde se hace uso de regresión parcial sobre datos extraídos de simuladores para la caracterización del comportamiento y \cite{DiazAlvarez2014} donde se aplican redes neuronales a datos de conducción reales para la predicción del consumo y la identificación de comportamientos agresivos.
}.

En~\cite{sekizawa2007modeling} describen modelos supervisados offline para capturar el comportamiento del conductor basados en auto-regresión a trozos. Más adelante lo extienden en~\cite{terada2010multi}, aunque los datos de entrenamiento son extraídos directamente de simulaciones, no del \enquote{mundo real\textsuperscript{TM}}.

\section{Tipos de maniobra}

En el nivel táctico del comportamiento, las tareas que se realizan durante la conducción son las relacionadas a circular con el vehículo dentro del flujo de tráfico, sin bajar demasiado de detalle. En la literatura, estas tareas se centran en dos clases generales de problema diferentes (figura \ref{fig:behavior-model-classification}): el de \textbf{aceleración}, que se ocupa de modelar el comportamiento de un conductor en el carril en el que se encuentra y el de \textbf{cambio de carril}, encargado de decidir y ejecutar los cambios de carril.

\begin{figure}
	\centering
	\begin{tikzpicture}
	\tikzset{every concept/.style={minimum size=2cm, text width=2cm}}
	\path[mindmap,concept color=MidnightBlue, text=white]
	node[concept] {modelos} [clockwise from=300]
	child[level distance=100, concept color=RoyalBlue] {
		node[concept] {aceleración} [clockwise from=30]
		child[concept color=Peach] {
			node[concept] {free-flow}
		}
		child[concept color=Peach] {
			node[concept] {car-following}
		}
	}
	child[level distance=100, concept color=RoyalBlue] {
		node[concept] {lane-changing} [clockwise from=210]
		child[concept color=Peach] {
			node[concept] {lane-selection}
		}
		child[concept color=Peach] {
			node[concept] {gap-acceptance}
		}
	};
	\end{tikzpicture}
	\caption{Las diferentes tareas para modelar el comportamiento de un conductor al volante. Están clasificadas en dos tipos, de aceleración, encargadas de definir cómo acelera el conductor en casos con y sin vehículo delantero y de lane-changing, encargadas de las tareas relacionadas con el cambio de carril.\TODO{Volver a echar un ojo a modelos de aceleración porque lo he cambiado.}}
	\label{fig:behavior-model-classification}
\end{figure}

Dentro de éstas, los autores en función del alcance y el objetivo del estudio identifican diferentes clases de sub-problemas. Algunos de éstos pueden ser selección decarril, cambio de carril, adelantamiento, adaptación a velocidad de vehículo frontal, etcétera (\cite{Aycin1999}). Mencionaremos estos tipos de problema a lo largo de la sección.

\subsection{Modelos de aceleración}

Se ocupan de gestionar el comportamiento del conductor sobre la aceleración (positiva o negativa) en un entorno lineal como lo es un carril de tráfico.

Las primeras menciones sobre modelos de aceleración se atribuyen a Reuschel (1950) y \cite{Pipes1953} por sus trabajos sobre el concepto de \textit{\idx{car-following}}.

\newthought{Un vehículo está en una situación de \idx{car-following}} cuando su velocidad está condicionado por el vehículo que se encuentra frente a él. El primer trabajo concreto es el de \cite{Pipes1953}, en el que el comportamiento responde a tratar de mantener un espacio variable en función de la velocidad\sidenote{En este modelo en concreto, el espacio viene determinado a partir de la ecuación de la velocidad cuando el tiempo no baja de $1,02$ segundos.}. Este modelo se puede considerar de una clase que denominaremos \textit{mantenimiento de medida} dado que su objetivo es manmtener constantemente una distancia segura. Otros trabajos trabajan con el mantenimiento de otras medidas como distancia relativa al parachoque delantero o trasero.

Posteriormente, en $1958$, se presentó el modelo GM\sidenote{Debido a que el desarrollo del modelo se realizó dentro de la \textit{General Motors Corporation}.} (\cite{Chandler1958}), el cual sirvió como base para el desarrollo de numerosos modelos posteriores. Este modelo se caracteriza por el uso del concepto \textit{estímulo $\leftarrow$ respuesta}, donde la acción (respuesta) del vehículo es debida a la activación de un estímulo tras pasar un tiempo de retardo  $\tau$. En el modelo de \cite{Chandler1958} la respuesta es el cambio de tasa de aceleración en función de la variación de la distancia al vehículo delantero. Algunas modificaciones sobre el algoritmo original son, entre otras, la asimetría en la tasa de cambio de aceleración y deceleración o la inclusión de segundos coches delanteros (\cite{Gazis1959}, \cite{Bexelius1968}).

\begin{figure}
	\missingfigure[figheight=4cm]{Una clasificación entre los tres tipos identificados de car-following: mantenimiento de medidas, estímulo-respuesta y psicofísicos}
	\caption{Los tres tipos generales de modelo de \textit{car-following}: basados en mantenimiento de medidas, basados en estímulo-respuesta y psicofísicos.}
	\label{fig:car-following-there-different-models}
\end{figure}

Los métodos de estas dos clases de modelo suelen ser sencillos de implementar, pero tienen un problema principal: suponen que el conductor es capaz de percibir todo cambio, incluso el más mínimo, en el coche delantero cuando esto en realidad no es así. En $1974$ apareció una nueva clase de modelos de \textit{\idx{car-following}}, denominados posteriormente como \textbf{psicofísicos} (\cite{wiedemann1974simulation} and \cite{Leutzbach1988}) donde se introduce el concepto de \textit{umbral perceptual}\sidenote{El \textit{umbral perceptual} de una medida es el límite a partir del cual se percibe un cambio en dicha medida. Mediante el uso de umbrales perceptuales, se limitan las acciones de los coches a cambios perceptibles en los coches delanteros.} como medida para superar la limitación de los otros dos tipos de modelo.

Sucesivos trabajos sobre el concepto de \textit{umbral perceptual} y los modelos psico-físicos llevaron a conclusión de que el \textit{\idx{car-following}} no era más que una clase más de un conjunto más amplio de modelos de aceleración (ver imagen~\ref{fig:acceleration-model-classes}). En \cite{wiedemann1992microscopic} se proponen hasta cuatro clases diferentes de modelos de aceleración en función de las posiciones y velocidades relativas entre el vehículo sujeto y el siguiente: \textit{free-flow}, donde el vehículo no se ve afectado por el comportamiento del siguiente vehículo y se basa en velocidad que quiere alcanzar sin impedimentos (más allá de los impuestos por el tipo y condición de la vía), \textit{car-following} cuando el comportamiento del vehículo se ve influenciado por el vehículo delantero, obligando a disminuir la velocidad deseada en el conductor en cuestión, \textit{approaching} como situación intermedia entre las dos anteriores y \textit{emergency} cuando la situación es crítica (e.g. colisión inminente). Otros autores posteriormente (e.g. \cite{Toledo2003} o \cite{Liu2013}) diferencian otras situaciones como el \textit{close-following} o \textit{stop-and-go}

\begin{figure}
	\missingfigure[figheight=4cm]{Diferentes clases de modelos de aceleración}
	\caption{El modelo \textit{car-following} sólo es uno entre muchas clases de modelos de aceleración.}
	\label{fig:acceleration-model-classes}
\end{figure}

\newthought{Agrupar en un mismo modelo} las diferentes clases de modelos de aceleración es un trabajo que se empezó a desarrollar a partir de los años $1990$. Los principales problemas son la complejidad de estos modelos, ya que aumentar las clases implica la generalización de multitud de factores. Algunos trabajos a este respecto son el modelo de Gipps (\cite{Gipps1981}) el cual agrupa las clases \textit{\idx{free-flow}} y \textit{\idx{car-following}}, el modelo de Yang et. al (\cite{Yang1996}) que agrupa las de \textit{emergency}, \textit{car-following} y \textit{free-flow} o el modelo Optimal Velocity (\cite{Bando1998}) que agrupa las de \textit{\idx{free-flow}}, \textit{\idx{car-following}} y \textit{stop-and-go}.

Una característica de todos los modelos hasta el momento es que no capturan los comportamientos de los diferentes tipos de conductor o vehículo. Sin embargo, el comportamiento puede variar en función de los comportamientos de los vehículos que se encuentran en su entorno (\cite{Tordeux2010}). Algunos estudios recientes como los de Simonelli et al. (2009), Colombaroni and Fusco, 2013 o Zheng et al., 2013 se puede decir que incorporan el comportamiento del entorno, pero únicamente porque entrenan redes neuronales con información tanto del coche como del entorno y por tanto éstas pueden haber aprendido detalles de éste. Más adelante se hablará de los modelos basados en técniacs de la glsentrylong{ci}.

...

SUMO usa (al menos así lo indican en el paper del 2002) el modelo Gipps\cite{krajzewicz2002sumo}. No sé si ellos han hecho una extensión del modelo o están referenciando la extensión y ellos sólo la implementan. En el paper del 2012 citan que el modelo car-following que usan por defecto es el desarrollado por Stefan Krauß\cite{jin2016evaluation}, debido a su simplicidad y su velocidad de ejecución. El modelo ha probado ser válido, pero tiene algunos defectos, por lo que existe un API para implementar otros modelos. En la actualidad están incluidos en el sistema los modelos IDM\cite{treiber2000congested} (\textit{Intelligent Driver Model}), el modelo de tres fases de Kerner\cite{kerner2008testbed} y el modelo de Wiedemann\cite{wiedemann1974simulation}.

(Barcelo et al., 1996, PETRI: A parallel environment for a real-time traffic management and information system) describen el simulador AIMSUN. El comportamiento de cada vehículo en la simulación es modelado a través de múltiples modelos de comportamiento (e.g. car following, lane changing, gap acceptance). El modelo de cambio de carril es el usado en el modelo de Gipps, aunque el propio simulador permite la modelización de incidentes por lo que debe existir alguna variación del modelo original o un nuevo modelo para ese caso concreto.

\subsection{Modelos de cambio de carril}

El tráfico real no está compuesto por un sólo carril, sino por varios. El cambio de carril ocurre cuando un vehículo se mueve del carril que está usando a un carril destino, ya sea porque quiere mejorar su circulación (e.g. quiere realizar un adelantamiento) o porque su ruta lo requiere (e.g. está próxima la rampa de salida que quiere tomar en una autopista).

Este comportamiento mejora la velocidad media del flujo de la vía en situaciones de poca carga de tráfico, aunque pueden afectar al tráfico en formas de ondas de choque (\cite{Sasoh2002}, \cite{Jin2006}) e interferir en éste. En situaciones de carga de tráfico media-alta o de congestión pueden llegar a interferir en el tráfico incluso más aun que los modelos de \textit{\idx{car-following}} (\cite{Laval2006}).

\begin{figure}
	\missingfigure[figheight=4cm]{Dos ilustraciones, una con la seleccción de cambio de carril y otra con el cambio.}
	\caption{El cambio de carril se divide tradicionalmente en una operación que involucra dos pasos. La selección de carril (\textit{lane-selection}) al que cambiarse y la ejecución del cambio (\textit{merging}).}
	\label{fig:lane-selection-plus-merging}
\end{figure}

Los modelos de cambio de carril o \textit{\idx{lane-change}} se ocupan de determinar cuándo un vehículo quiere desplazarse de un carril a otro (denominado \textbf{\idx{lane-selection}}) y de la ejecución del mismo (denominado \textbf{merging}). Esta división fue introducida en \cite{Sparmann1978}, donde el autor además distinguía entre cambios hacia la izquierda (motivados por obstrucciones como por ejemplo coches lentos) y hacia la derecha (motivados por, por ejemplo, no obstrucciones), estando determinada la ejecución del cambio por el espacio en el carril objetivo.

En \cite{Gipps1986} se introduce el concepto de cambio de carril \textbf{obligatorio}, ejecutado cuando es obligatorio abandonar el carril actual o acceder al carril objetivo y \textbf{discrecional} cuando el cambio se ve motivado para mejorar la situación actual de conducción (ver figura~\ref{fig:lane-change-mandatory-vs-discretional}). Este trabajo propone un framework para el problema de cambio de carril al aproximarse a un cambio de dirección. El modelo identifica tres distancias que caracterizan el comportamiento del conductor en función de cómo de lejos está dicho punto: (i) textbf{lejos}, en el que no existe condicionamiento en la decisión de cambio de carril, (ii) textbf{medio}, donde el conductor empieza a ignorar los cambio que dan ventaja de velocidad si no hacia carriles distanciados del de salida y (iii) \textbf{cerca} donde los vahículos deben estar en el carril de cambio de salida. El modelo tiene los problemas de que los factores son evaluados secuencialmente (por lo que no se evalúan factores más bajos en la jerarquía si no es necesario) y de que supone que el cambio de carril ocurre sin forzar a los vehículos del carril de destino a modificar su comportamiento como disminuir la velocidad o parar. El modelo de \cite{Hidas2002} advierte de esta situación indicando que el cambio en situaciones de congestión ha de ser o bien forzado o bien a través de colaboración.

\begin{figure}
	\missingfigure[figheight=4cm]{Dos ilustraciones, una con un cambio de carril obligatorio (que se acabe el carril por ejemplo) y otro discrecional (que se vea un vehículo lento como un camión y el coche adelantando).}
	\caption{Los cambios de carril se clasifican en la literatura como aquellos necesarios para continuar con la conducción u \textbf{obligatorios} y aquellos útiles para mejorar la situación de conducción o \textbf{discrecionales}.}
	\label{fig:lane-change-mandatory-vs-discretional}
\end{figure}

En \cite{wiedemann1992microscopic} se desarrolla un framework que tiene en cuenta los cambios de carril lento a rápido (debido, por ejemplo a alguna obstrucción como accidente o un vehículo lento) y de rápido a lento (por ejemplo debido a las condiciones de la ruta). Ya que el modelo está afectado por la influencia del entorno, divide esta en dos tipos: (i) actual, las características de los vehículos de alrededor y (ii) potencial, la estimación de las características del entorno en momentos posteriores al actual.

En \cite{Hidas2002} se presenta un modelo similar al de \cite{Gipps1986} apoyándose también en los dos tipos de cambio de carril, \textbf{obligatorio} y \textbf{discrecional}. Se presentan una serie de factores (también evaluados en secuencia) que provocan situaciones de cambio de carril. Cuando las situaciones son del tipo de mejora de la situación de conducción, se dispara un cambio discrecional, pero cualquier situación que dispara un cambio obligatorio descarta toda decisión de cambio discrecional por considerarse prioritario.

Más trabajos en la línea de los anteriores son \cite{Halati1997, Yang1996, Ahmed1999}, los cuales fijan la separación jerárquica entre cambios obligatorios y cambios discrecionales, o \cite{Toledo2003, Wei2000} donde salvan dicha limitación.

\cite{Ahmed1999} propone un modelo probabilístico que divide la secuencia de cambio de carril en tres fases: decisión de camboi, selección de carril y ejecución (i.e. aceptación del hueco y merge). Además, a las clases de cambio de carril \textbf{obligatorio} y \textbf{discrecional} definidas en \cite{Gipps1986} añade una nueva, \textbf{forzado}, definida para situaciones de mucha congestión de tráfico, cuando se abre un hueco suficiente para ejecutar el cambio. Se usa \gls{mitsim} como plataforma en la que realizar los tests del modelo.

En \cite{Toledo2003, Toledo2009} Plantean un modelo en el que, manteniendo la clasificación de los cambios de carril en obligatorio y discrecional, separan el adelantamiento en selección de carril y gap-acceptance. El modelo es probabilístico y clasifica las entradas en cuatro grupos de variables: las de vecindad (e.g. huecos y velocidades), las de planificación de ruta (e.g. distancia a la salida objetivo), las relacionadas con la experiencia (e.g. evitar un determinado carril en un determinado tramo) y las de estilo de conducción.

Tanto \cite{Ahmed1999} como \cite{Toledo2003, Toledo2009} prueban sus modelos en \gls{mitsim}


\newthought{La viabilidad en un cambio de carril} se determina haciendo uso de modelos denominados \textit{\idx{gap acceptance}}, donde los vehículos calculan si caben o no en un determinado huevo. Estos modelos fueron creados inicialmente para situaciones emplazadas en intersecciones aunque posteriormente se comprobó su utilidad para determinar si es posible o no el cambio a un carril basándose principalmente en el espacio hueco existente en el carril destino.

\begin{equation}
f_{g_l}(t) = \twopartf {0} {g_l(t) < g^{crit}_l(t)} {1} {g_l(t) \geq g^{crit}_l(t)}
\label{eq:gap-acceptance-model}
\end{equation}

En la ecuación~\ref{eq:gap-acceptance-model} se describe el modelo típico de un modelo de \idx{gap accceptance}. En un momento $t$, el cambio a un carril $l$ es viable ($f_{g_l}(t) = 1$) o no ($f_{g_l}(t) = 0$) dependiendo de si el espacio en el carril destino $g_l(t)$ es mayor o menor que un \enquote{hueco crítico} (en inglés \idx{critical gap}) $g^{crit}_l(t)$. Diferentes autores determinan el hueco crítico basándose en diferentes parámetros (e.g. \cite{Miller1972} o \cite{Cassidy1995})

En \cite{Gipps1986} y \cite{Ahmed1996} dividen sin embargo el hueco crítico en dos, hasta el vehículo delantero y hasta el vehículo trasero. Ambos deben ser aceptables y en dichos huecos influye además la velocidad relativa respecto a los vehículos delantero y trasero. Otras variaciones a la hora de determinar el hueco crítico incluyen la variación del tamaño en función de si es obligatorio o discrecional (\cite{Toledo2003}), velocidades relativas (\cite{Ahmed1999}) o cooperación entre conductor realizando el cambio y conductores en carril destino (\cite{Ahmed1999}, \cite{Hidas2002}).

...


\newthought{Un cambio de carril no tiene por qué involucrar al conductor que lo ejecuta}....

...

En \cite{Fritzsche1994} se describe un modelo de microsimulación para analizar cuellos de botella (e.g. un accidente donde se bloquea uno de los carriles). Es un caso típico donde los vehículos no pueden cambiar de carril sin la participación activa del resto de vehículos (colaboración). El modelo lo describe de una maera muy sucinta y no considera comportamiento colaborativo en el cambio de carril.

En (Yang and Koutsopoulos, 1996, A Microscopic Traffic Simulator for Evaluation of Dynamic Traffic Management Systems) presentan el simulador MITSIM, desarrollado por el MIT (creo) en el que se habla específicamente de comportamiento colaborativo en cambio de carriles haciendo uso de lo que denominan "courtesy yielding function" (algo así como función de cesión de paso de cortesía), la cual se usa para para hacer espacio al vehículo que va a incorporarse al carril. Sin embargo, los detalles de dicho proceso no están especificados en el paper.

...

Sparmann (\cite{Sparmann1978}) propone modelo psico-físico a las velocidades y distancias relativas.

\subsection{Modelos mixtos}

Los modelos de aceleración y de cambio de carril han sido tratados tradicionalmente como modelos independientes, siendo estos primeros mucho más estudiados que los segundos\sidenote{Este hecho es motivado por la dificultad en la captura de datos en los cambios de carril y, por tanto, por su escasez.}.

\begin{figure}
	\includegraphics{toledo-2007-behavior-model-tree}
	\caption{Estructura del modelo de comportamiento de los vehículos en \cite{Toledo2007}. El agente comprueba constantemente se desea o no cambiar de carril, y una vez decidido comprueba la viabilidad. En cualquier caso, toda decisión finaliza con la comprobación de la aceleración. Fuente: \cite{Toledo2007}.\TODO{Quizá la imagen es muy grande y quedaría mejor si la hiciésemos más pequeña.}}
	\label{fig:toledo-2007-behavior-model-tree}
\end{figure}

Sin embargo, a partir de los años $90$ se ha tendido al desarrollo de modelos de comportamiento que combinan los modelos de aceleración y de cambio de carril (\cite{Ma2004}). Un ejemplo actual de este tipo de modelos es el descrito en \cite{Toledo2007}, el cual se basa en el concepto de \enquote{objetivo a corto plazo} para elaborar un \enquote{plan a corto plazo} apoyándose en un árbol de decisión que determina la accinó a realizar (ver figura~\ref{fig:toledo-2007-behavior-model-tree}).

...

\section{La \glsentrylong{ci} en los modelos de conducción}

Hasta ahora, los modelos que hemos explorado en la descripción de maniobras pertenencen al dominio del  \glsentrylong{hc}, es decir, están basados en fórmulas matemáticas y reglas de la lógica convencional con parámetros que se ajustan a partir de la observación de datos reales.

Sin embargo, desde mediados de los años $90$ empezó a crecer el interés por las técnicas de la \gls{ci} debido, entre otras a las siguientes razones:

\begin{itemize}
	\item A mediados del interés volvió el interés de la \gls{ai} debido a los éxistos cosechados por las técnicas de la \gls{ci}, y por tanto se comenzaron a usar en todas las áreas, incluída la de las its.
	\item El rápido desarrollo de la tecnología ha hecho posible la existencia de conjuntos de datos masivos (i.e. mayor capacidad de captura y almacenamiento), con más calidad (e.g. sensores más precisos) y más cantidad de fuentes (e.g. GPS, acelerómetros, giroscopios, \ldots). Esto no sólo permite el ajuste de los modelos existentes o las pequeñas modificaciones, sino que son una fuente de datos muy interesante para técnicas de \glsentrylongsp{ml}, rama que pertenece a la \gls{ci}.
\end{itemize}

Algunos autores incluso sugieren que el futuro de las \gls{its} pasa por cambiar del paradigma hacia técnicas basadas en \gls{ml}. Es decir, pasar del desarrollo convencional de sistemas al desarrollo con técnicas basadas en procesado y aprendizaje de datos (\cite{Zhang2011}).

Las técnicas de \gls{ci} se usan principalmente para dos áreas: la caracterización y el modelado\sidenote{
	No es en estas áreas exclusivamente. La \gls{ci} se usa prácticamente en cualquier problema que tenga que ver con detección de patrones, predicción, planificación, etcétera. Por ello, es fácil encontrar aplicaciones de sus técnicas a prácticamente cualquier tema. Por ejemplo, en \cite{terroso2015complex} hacen uso de un \gls{cep} que procesa los diferentes sensores del vehículo para detectar y analizar patrones característicos con técnicas tales como clasificadores basados en \gls{fl} y, de esta manera, predecir no sólo el número de ocupantes, sino la tipología o clase de ocupante (e.g. niños, adultos, \ldots).
}.

\newthought{La caracterización} de conductores es interesante porque permite identificar perfiles de conducción y clasificar a los conductores de acuerdo a indicadores extraídos de su manera de conducir. Por ejemplo, en \cite{van2013driver} hacen uso de datos extraídos de sensores de inercia para construir un perfil de conductor, concluyendo que frenar y girar son mejores indicadores para la caracterización que la aceleración.

En~\cite{bando2013unsupervised} describen otro modelo no supervisado offline basado en un modelo bayesiano no paramétrico para la clusterización, combinado con un LDA (Latent Dirichlet Allocation, sea lo que sea esto) para la clusterización a más alto nivel. (\TODO ¿este método usa datos reales?)

Estos trabajos (\cite{sekizawa2007modeling}, \cite{terada2010multi} y \cite{bando2013unsupervised}) tienen la desventaja de ser computacionalmente muy costosos y con poca precisión en el caso de variar mucho los escenarios de entrenamiento y de test.

En~\cite{maye2011bayesian} se presenta un modelo online donde se infiere el comportamiento del conductor haciendo uso de una IMU (Intertial Measurement Unit) y una cámara. Primero de la IMU se sacan datos que se separan en fragmentos para luego relacionarlos con las imágenes obtenidas de la cámara. (\TODO ¿Hacia dónde mira la cámara?). Los modelos propuestos en~\cite{johnson2011driving} y~\cite{van2013driver} también se apoyan en el funcionamiento de clasificar las segmentaciones de una IMU, pero con técnicas distintas y sin cámara. Además hacen uso de señales externas y umbrales de activación para hacer más efectiva la clasificación (\TODO corroborar).

En el artículo~\cite{bender2015unsupervised} se usa también un modelo no supervisado online con una aproximación bayesiana para identificar los puntos de cambio sin depender de parámetros externos (e.g. umbrales o señales). Se basa también en (1) segmentar los datos de conducción y (2) asignar estos datos a clases que se corresponde n con comportamientos de conducción. Tiene la ventaja de ser más eficiente y rubusto que los anteriores.

La idea de estos métodos desde el~\cite{sekizawa2007modeling} hasta el~\cite{bender2015unsupervised} creo que es el de un sistema que traduzca datos en crudo a datos de más alto nivel. Esto es debido a que la cantidad de datos que se pueden generar en un sólo coche (no digamos ya una flota de ellos) es tan grande que para determinados sistemas disponer de información de más alto nivel haría más sencillo su desempeño (por ejemplo, \gls{adas} que funcionasen con datos de \enquote{adelantando} que sus valores de giro, aceleración en una ventana temporal).

En~\cite{satzoda2013towards}, haciendo uso de la información combinada de bus CAN, cámaras, GPS e información digitalizada el mapa donde se circula se determina una amplio abanico de información crítica en diferentes condiciones de la carretera. La información que sacan es: número de cambios de carril a la derecha, a la izquierda, tiempo en autopista y carretera urbana, distancia recorrida, velocidades medias en autopista y urbano, paradas, giros a la derecha, a la izquierda, incormporaciones y salidas de autopsta, tiempo gastado en un sólo carril, curvas a la izquierda, curvas a la derecha y distancia media al carril central


\newthought{}

En \cite{Dia2002} Los parámetros relativos al comportamiento, es decir, los que definen características, forma de razonar, etcétera son extraídos de encuestas a conductores reales. De acuerdo a los autores, cada conductor con sus propias características su forma de razonar sus perceciones y sus objetivos se puede modelar como un agente.

\newthought{En microsimulación}, las técnicas dominantes de los modelos de conducción son las \glsentrylong{ann} y la \glsentrylong{fl}. Las primeras debido a ser una de las técnicas principales en la rama del \gls{ml}, y la segunda por ser una manera sencilla y cercana a la manera de razonar del ser humano.

...

Para el cambio de carril

Los modelos estudiados no tienen en cuenta las inconsistencias y la incertidumbre de las percepciones y decisiones de un conductor humano \cite{McDonald1997}. Dichos modelos pertenencen a la clase \glsentrylong{hc}, esto es, valores fijos, ecuaciones y reglas de la lógica convencional para representar el conocimiento, las percepciones y las decisiones de los conductores. La única manera que tienen los autores de añadir incertidumbre en los modelos es mediante la introducción de términos aleatorios.

Sin embargo, los conductores basan sus decisiones en sus percepciones y éstas, aunque imprecisas, no tienen por qué seguir una distribución aleatoria con distribución clásica (e.g. normal). Las técnicas basadas en \gls{ci} palian los inconvenientes típicos de estas técnicas al pertenencer sus técnicas a soluciones de la clase \glsentrylong{sc}.

\cite{Das2009} una nueva metodología para cambio de carril basada en lógica difusa. El simulador se denomina Autonomous Agent SIMulation Package (AASIM). La motivación es que los sistemas difusos son capaces de modelar sistemas no lineales complejos como  reglas de la forma \texttt{if \ldots then \ldots} y que además la lógica difusa es ideal para modelar la incertidumbre del mundo real y por tanto de las percepciones de los conductores. Clasifican los cambios de carril en MLC y DLC. En MLC las reglas tienen en cuenta la distancia al siguiente punto característico (e.g. una salida) (\TODO{a lo mejor en la ilustración del principuio podemos ponerle nombre aesto y asi'erferirnos a ello en el resto del texto}) y el número de cambios de carril necesarios. En DLC deciden si cambiar o no basándose en el nivel de satisfacción del conductor y en el nivel de congestión en los carriles adjacentes.

\cite{McDonald1997, Wu2003} desarrollan otro modelo de simulación denominado Fuzzy LOgic motorWay SIMulation (FLOWSIM) con similares características donde establecen conjuntos difusos para el modelo. Dso categorías de cambio de carril, al lento (principalmente para evitar incordiar a los vehículos que se aproximan por detrás a velocidades superiores, usan dos variables, presión del vehículo trasero y satisfacción en el gap del carril destino) y al rápido (para ganar velocidad, variables: velocidad ganada con el cambio y oportunidad, es decir, seguridad y confort con el cambio).

\TODO{Aquí hay que meter el cuadro ese de las reglas difusas para ejemplo}

\TODO{Echar un vistazo a la tabla resumen}

\subsection{El enfoque de agentes}

...

\cite{Hidas2002} --> También aquí hablan de los DVOs (driver-vehicle objects) (¿esto es un clon de \glspl{dvu}?. Tienen características individuales como (i) tipo de vehículo, (ii) magnitudes físicas (tamaño, velocidad máxima, ...), (iii) tipo de conduictor y (iv) nivel de conocimiento de la red (porque afecta en la elección de ruta). Tienen un objetivo, llegar del origen al destino tan rápido como puedan. Esto implica un conjunto de decisiones a hacer en intervalos periódicos durante su funcionamiento (i) selección de ruta cada vez que se entra en un nuevo tramo y (ii) cálculo de la aceleración en cada intervalo).

En (Chaib-draa and Levesque, 1996, ) proponen un framework para trabajar con tres tipos diferentes de situaciones (rutina, familiar y no familiar) en sistemas multiagente, demostrando la aplicación en escenarios de microsimulacinó urbana). Su modelo se basa en una estructura jerárquica definida por los niveles de comportamiento humano y de técnicas de razonamiento propuestas por (Rasmussen, 1986, ) (skill-rule-knowledge). El comportamiento basado en habilidades (skills) se refiere a las actividades completamente automatizadas (percepción--ejecución) usadas típicamente en situaciones rutinarias. comportamiento basado en reglas se refiere a situaciones esteorotipadas (percepción--reconocimiento de la situación--planificación--ejecución) aplicable en su mayoría en situaciones familiares. El comportamiento basado en el conocimiento se refiere a actividades conscientes que implican trabajo de resolución de problemas y toma de decisiones (perception--reconocimiento de la situación--toma e decisión--planificación de la ejecución) que suelen ser necesarias en situaciones poco familiares. En SITRAS se ven estas diferencias claramente en el cálculo de la aceleración. (i) si no hay ninguna otra restriccinó, llegar a la máxima velocidad es acelerar hasta la máxima velocidad (skill), (ii) si hay ua luz roja más adelante, se va frenando hasta para el coche (rule) y (ii) si se recibe información de coches alrededor (por ejemplo se está incorporando un nuevo coche a nuestro carril) se requiere un conocimiento más complejo (knowledge). Los DVO aquí tienen las siguientes debilidades: no tienen memoria (sólo planean el segundo siguiente de acuerdo a la información actual) y tienen poco contacto directo con los demás DVOs de alrededor (saben del de delante y del de detrás, pero no de los lados).

En~\cite{al2001framework} describen un framework para la modelización de comportamiento de conductores dentro de simuladores. Se basa en cuatro unidades de funcionamiento interconectadas, la de percepción (percibe el entorno en términos locales y globales), la de emoción (cómo responde emocionalmete al entorno), la de decision-making (investiga posibles acciones que podrían servir a las necesidades del módulo emocional) y la de decision-implementation (intenta implementar las decisiones cuando emergen condiciones de tráfico lo suficientemente seguras para llevarlas a cabo). Tengo que volver a leerlo después de hacer una primera introducción en el tema de agentes, porque me parece poner nombrecitos a un tipo de agente que funciona de esa misma manera, pero lo mismo no. En \cite{Kuge2000} proponen otro framework de este palo.


En \cite{Das} se realiza una simulación de comportamiento de vehículos en autopista. Los agentes basan su comportamiento en un sistema difuso donde las reglas definen la conclusión en autopista (i.e. car-following y lane-changing). Llaman a este simulador AASIM (Autonomous Agent Simulator).,

En \cite{Ehlert2001} (ver si este otro paper del autor cuenta lo mismo y me quito de una de las dos referencias: \cite{Ehlert2001-2}), simulación donde los agentes son de tipo reactivo. Además, poseen diferentes estuilos de conduccińo. El agente cntinuamente va realizando decisiones de control para mantenerse en la via y llegar a su destino.


\subsection{Modelos basados en lógica difusa}

Los modelos de conducción basados en \glsentrylong{fl} parten de la hipótesis de que la información que maneja el conductor a la hora de tomar decisiones proviene de un análisis no demasiado detallado de la situación que le rodea; es decir, la percepción y el comportamiento humanos son estímulos percibidos de manera aproximada. Por tanto, el resultado debe ser fruto de un proceso de razonamiento que tenga en cuenta esa imprecisión en los estímulos.

El primer trabajo documentado en \gls{fl} es \cite{Kikuchi1992}, donde los autores aplicaron la lógica difusa sobre un modelo de aceleración de tipo \textit{\idx{car-following}}. Utilizaron el modelo \gls{ghr}\sidenote{
	El modelo \gls{ghr} (\cite{Chandler1958}) es el modelo más conocido antes de la introducción del modelo de Gipps. Desarrollado a finales de los años $50$, calcula el valor de la aceleración $a$ en un instante $t$ como:
	
	\begin{equation}
	a(t) = c v^m(t) \frac{\Delta v(t - \tau)}{\Delta x^l(t - \tau)}
	\label{eq:ghr-car-following-model}
	\end{equation}
	
	Siendo $t$ es el instante actual, $a(t)$ la aceleración del vehículo, $\delta v(t)$ y $\delta x(t)$ son la velocidad y distancia relativas al siguiente coche respectivamente, $v$ la velocidad del vehículo y $c$, $m$, $l$ y $\tau$ constantes, siendo ésta última el tiempo de reacción del conductor.
} como base y determinaron las entradas al modelo como valores de pertenencia a conjuntos difusos. Las entradas del modelo eran las distancias y velocidades relativas entre el vehículo modelado y el delantero y las variaciones en la aceleración del vehículo delantero, y como salidael cambio en la tasa de aceleración sobre el vehículo modelado. Más adelante los mismos autores aplicaron el mismo razonamiento sobre el modelo de General Motors (Modelo GMC)~\cite{Chakroborty1999}.

Otros trabajos en la línea de investigación de~\cite{Kikuchi1992} y \cite{Chakroborty1999}
...

Data-driven approaches have already been used in developing a fully adaptive cruise control system (Simonelli et al., 2009; Bifulco et al., 2013)

Otros trabajos que trabajan con lógica difusa: (Calibrating the membership functions of the fuzzy inference system: instantiated by car-following data, A FUZZY LOGIC MODEL OF FREEWAY DRIVER BEHAVIOR, The Modeling and Simulation of the Car-following Behavior Based on Fuzzy Inference, Fuzzy parameters estimation for car-following modelling, A Fuzzy Logic Approach for Car-Following Modelling, Variable response time lag module for car-following models, Development of a fuzzy logic based microscopic motorway simulation model, Establishment of car following theory based on fuzzy-based sensitivity parameters. Advances in multimedia modelling, Application of fuzzy systems in the car-following behaviour analysis, The validation of a microscopic simulation model: A methodological case study). Ninguno de estos estudios considera diferentes tipologías de vehículos para el car following

Problemas:

\begin{enumerate}
	\item ¿Qué reglas usan los humanos para modelar su comportamiento? Desconocerlas implica modelos no realistas. En (The validation of a microscopic simulation model: A methodological case study) intenta suplir este problema con encuestas a conductores.
	\item Los problemas inherentes de los controladores difusos. ¿Cómo validar las funciones de pertenencia? ¿cómo determinar las reglas difusas?
\end{enumerate}

\subsection{Modelos basados en redes neuronales artificiales}

Las redes neuronales se han aplicado mucho sobre el campo de las ITS en general, y sobre la conducción autónoma y el análisis del comportamiento de los conductores (A review of neural networks applied to transport).

En (Modelling human performance with neural networks) implementaron un controlador basado en redes neuronales para el comportamiento del car following en un microsimulador entrenando dicho modelo previamente con datos extraídos de un conductor en dicho simulador.

En (The  use  of  neural  networks  to  recognise  and
predict traffic congestion) usan redes para determinar el nivel de congestión en la vía.

(Develop a car-following model using data collected by ‘five-wheel
system’) son los primers en usar datos reales de un coche instrumentado usando el método Five-Wheel-System, que está especificado en su paper de aquella manera y que no me entero de nada. A partir de las entradas correspondientes a velocidad relativa, espacio relativo, velocidad y velocidad deseada (para ello, clasifican al conductor de agresivo, normal, conservador) determinan la aceleración/deceleración del vehículo. No lo aplican a ningún simulador, sólo que los valores se ajustan.

(Neural agent car-following models) redes neuronales usando el dataset de (Traffic simulation supporting urban control system development) desarrollan una red neuronal para mantener la distancia con el siguiente vehículo. Este modelo sí se evaluó en el simulador AIMSUN, y los resultados muestran una buena correspondiencia entre los datos y la realidad. No replica sin embargo el comportamiento de frenar hasta parar o de acelerar desde parado.

A Modified Car-Following Model Based on a Neural Network Model of the Human Driver Effects


Problemas:

\begin{enumerate}
	\item Es imposible determinar por qué la red funciona como está funcionando.
	\item Los clásicos de los problemas de redes, el no aprendizaje y la especialización.
\end{enumerate}

...

En (Hunt and Lyons, 1994, Modelling of dual-carriageway lane-changing using neural networks) desarrollan un modelo de decisión usando redes neuronales. El modelo funciona a partir de presentarle una entrada visual del entorno alrededor del vehículo que quiere cambiar de carril. Sin embargo, no considera la cooperación entre vehículos.

Simonelli et al. (2009) have applied neural networks to develop a real-time learning mode for capturing car-following behavior taking into consideration individual drivers’ characteristics. Bifulco et al. (2013) extended the work of Simonelli et al. (2009) into a framework for reproducing spacing in adaptive cruise control applications. While in this research we have used data derived from the same experiment as Simonelli et al. (2009), the scope and level of complexity of the studies is very different. While all studies adopt a data-driven approach, in this paper the objective is to create a simple and practical methodology for speed estimation using car-following models for use in a microscopic traffic simulator.

\subsection{Otras técnicas}

En \cite{Hou2011} proponen una técnica basada en modelos ocultos de Markov para la identificación de posibles cambios de carril. Los conductores conducen vehículos en un simulador de autopistas. A partir del ángulo de giro del volante el modelo es capaz de estimar, con una precisión del $0,95$, si el conductor va a realizar un cambio a la derecha, a la izquierda o si va a mantenerse en el carril. Creo que \cite{Berndt2008} es también por el estilo. De hecho puede que el primero es haya \enquote{inspirado} mucho.

\subsection{Aproximaciones híbridas}

(Simulation of car-following decision using fuzzy neural network system) y (Toward an integrated car-following and lane-changing model based on neural-fuzzy approach) usan aproximaciones de fuzzy neural networks y neuro-fuzzy respectivamente. No proveen sin embargo de documentación y no se investiga la aplicación de estos modelos a microsimuladores de tráfico.

(Exploring a local linear model tree approach to car-following) usan el modelo de árboles lineales locales (LOLIMOT, (Nonlinear system identification: From classical approaches to neural networks and fuzzy models)) que no deja de ser una aproximación neuro-fuzzy del comportamiento. Intenta incorporar imperfecciones perceptuales en un modelo de car following. El modelo está basado usando datasets reales y los resultados indican que se ajusta lo predicho con la realidad, pero no hay pruebas realizadas en microsimuladores.

En~\cite{Ma2004}, bajo la suposición de que el ser humano toma múltiples decisiones relacionadas basándose en su percepción (imprecisa) del entorno propone un método masado en un sistema de inferencia difusa para tomar decisiones tanto para el problema del \textit{car following} como para el del \textit{lane changing}, calibrando y ajustando dicho controlador mediante el uso de redes neuronales (aproximación Neuro-Fuzzy).

Modeling car-following behavior via artificial neural networks (Colombaroni and Fusco, 2013; Chong et al., 2013; Zheng et al., 2013)