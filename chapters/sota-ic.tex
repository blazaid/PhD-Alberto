\chapter{\glsentrylong{ic}}
\label{ch:sota-ic}

El comportamiento de una persona se ve influenciado por una gran cantidad de variables. Identificar las relaciones entre éstas es en la mayoría de las ocasiones una tarea que va de lo muy difícil a lo imposible, más auń si añadimos que en ocasiones estas variables son imposibles de medir o incluso de identificar.

La \gls{ic} engloba un conjunto de técnicas que facilitan enormemente estas tareas. En este capítulo se ofrece una perspectiva de la literatura actual sobre las técnicas de la inteligencia computacional que son de interés para esta tesis.

\section{Inteligencia artificial vs. inteligencia computacional}

¿Qué es la \gls{ic}? Para entender el significado de éste término tenemos que entender la evolución del significado de inteligencia artifical desde los inicios de la investigación en el campo.

Inteligencia computacional: habilidad de un ordenador de aprender una tarea específica a partir de datos u observaciones experimentales (esto es sacado de la wikipedia. Hay que buscar más definiciones en la literatura). Destacar \textbf{tarea específica} y \textbf{a partir de datos experimentales}.

Algunos autores lo consideran sinónimo de Soft Computing, pero no como no hay definición comúnmente aceptada, pues na.

La cosa es que hay procesos matemáticos muy complejos, con mucha interdependencia entre variables y factores que hacen que los problemas sean muy difíciles de modelar, más aún cuando el problema en su naturaleza es estocástico (poner algún ejemplo de algo en la naturaleza que se comporte de forma estocástica~\cite{siddique2013computational}).

But generally, computational intelligence is a set of nature-inspired computational methodologies and approaches to address complex real-world problems to which mathematical or traditional modelling can be useless for a few reasons: the processes might be too complex for mathematical reasoning, it might contain some uncertainties during the process, or the process might simply be stochastic in nature.[1] Indeed, many real-life problems cannot be translated into binary language (unique values of 0 and 1) for computers to process it. Computational Intelligence therefore provides solutions for such problems.



\section{Inteligencia computacional}

Qué es la inteligencia artificial. Qué es la inteligencia computacional.
Diferencias entre inteligencia artificial clásica e inteligencia computacional. Diferentes puntos de vista (soft computing, machine learning, ...)
Técnicas de entrenamiento de modelos: supervisado, no supervisado, semisupervisado, refuerzo, ...
Técnias de funcionamiendo: online y offline
Técnicas de la inteligencia computacional usadas en esta tesis (redes neuronales artificiales(perceptrón multicapa, recurrentes y lstm), lógica difusa y computación evolutiva)
¿Qué técnicas se usan actualmente y sobre qué problemas?
Detección de patrones de eficiencia y agresividad de subyacen en los comportamientos de éstosç
Estudio de la efectividad de los sistemas de asistencia para mejorar la eficiencia de conducción
Estudio de los sistemas de asistencia para analizar el comportamiento del conductor.