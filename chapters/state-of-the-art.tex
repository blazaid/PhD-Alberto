\chapter{Estado de la cuestión}
\label{ch:state-of-the-art}

Simuladores, micro y macro
Comparativa de simuladores y por qué se ha elegido SUMO
Sistemas multiagente
Qué es la inteligencia artificial.
Diferencias entre inteligencia artificial clásica e inteligencia computacional. Diferentes puntos de vista (soft computing, machine learning, ...)
Técnicas de la inteligencia computacional usadas en esta tesis (redes neuronales artificiales(perceptrón multicapa, recurrentes y lstm), lógica difusa y computación evolutiva)



Aplicar técnicas de la inteligencia computacional (o de la rama subsimbólica de la IA o del softcomputing) sobre el área de la conducción. -- ¿Qué técnicas se usan actualmente y sobre qué problemas?

Concretamente -- Sobre el estudio de la efectividad de sistemas de asistencia encaminados a mejorar la eficiecia y para el análisis del comoprtamiento del conductos (detección de patrones de eficiencia y agresividad de subyacen en los comportamientos de éstos). -- Aquí hay dos cosas. Por un lado Estudio de la efectividad de los sistemas de asistencia para mejorar la eficiencia de conucción y estudio de los sistemas de asistencia para analizar el comportamiento del conductor.


Estudio y aplicación de técnicas de la rama subsimbólica de la inteligencia artificial sobre el área de la conducción. Concretamente para el estudio de la efectividad de sistemas de asistencia encaminados a mejorar la eficiencia y para el análisis de comportamiento de conductor (detección de los modelos y patrones de eficiencia y agresividad que subyacen en los comportamientos de los conductores).

El núcleo de la Tesis consiste en el estudio y la aplicación de técnicas de la rama subsimbólica de la Inteligencia Artificial sobre el área de la conducción, concretamente para el estudio de la efectividad de sistemas de asistencia encaminados a mejorar la eficiencia y para el análisis del comportamiento de conductor (detección de los modelos y patrones de eficiencia y agresividad que subyacen en los comportamientos de los conductores).
