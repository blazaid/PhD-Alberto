\chapter{Estado de la cuestión}
\label{ch:state-of-the-art}

Se pretende desarrollar un método para el análisis de la eficiencia de los conductores, realizando para ello un modelo del perfil de conducción a partir de técnicas de Inteligencia Computacional y aplicándolo en un entorno multiagente de donde obtener el resto de parámetros. Así, una vez configurado el entorno multiagente, se podrá simular el tráfico y el comportamiento de los conductores dentro de éste cuando su marcha está condicionada por factores como el tráfico, semáforos, etcétera.

Es para ello necesario un análisis previo del estado de la cuestión en torno a varios conceptos y su relación: la inteligencia computacional y sus técnicas, agentes y sistemas multiagente y por último, modelos de conductor y simuladores de tráfico. Esta sección se encarga de recoger el resultado de la revisión de la literatura en estos temas y sirve de base para el desarrollo de la tesis.

\section{Simulación de tráfico}

El tráfico es un sistema cuyo funcionamiento depende de multitud de factores con una organización caótica. Debido a esto, obtener modelos exactos es una tarea prácticamente imposible y es por ello que la mayoría del trabajo cuyo objetivo es la predicción se realice en base a simuladores.

Hoy en día la oferta de simuladores de tráfico existente es amplia. NO SÉ QUÉ COJONES PONER AQUÍ DE QUE SE OFERTAN CON DIFERENTES LICENCIAS, TANTO ABIERTAS COMO CERRADAS.

A lo mejor aquí hay que especificar que hablando de simuladores de tráfico, sólo vamos a centrarnos en temas de simulación de vehículos utilitarios (coches o como cojones se llamen), sin entrar en detalles de simulación de otro tipo de vehículos o de peatones.

Dentro de simulación de tráfico que no es directamente simulación de coches, se  puede hablar (para nombrar y hacer un par de párrafos con refeerncias) de la simulación para la evaluación de sistemas de señalización inteligentes\cite{jin2016evaluation}, estimación de emisiones\cite{quaassdorff2016microscale}

He encontrado un smiulador que es relativamente reciente y que puede merecer la pena mencionar (más párrafos, yuhu!). es el X10, y cero que tiene que ver con algo de IBM, y algo también de inteligencia porque hablan de preferencias para el conductor en temas deruta, evlocidad y no se qué hostias. Dejo aquí los enlaces: \url{http://x10.sourceforge.net/documentation/presentations/X10DayTokyo2015/x10daytokyo15-mizuta.pdf}, \url{http://ieeexplore.ieee.org/xpl/login.jsp?tp=\&arnumber=6365068\&url=http\%3A\%2F\%2Fieeexplore.ieee.org\%2Fxpls\%2Fabs\_all.jsp\%3Farnumber\%3D6365068}, \url{http://informs-sim.org/wsc12papers/includes/files/pos148.pdf}, \url{https://www.researchgate.net/profile/Tsuyoshi\_Ide2/publication/236173272\_X10-based\_massive\_parallel\_large-scale\_traffic\_flow\_simulation/links/00b4951b89905784a7000000.pdf}


AQUÍ ESTARÍA MUY BIEN UNA IMAGEN COMO LA QUE APARECE EN~\cite{krajzewicz2002sumo} SOBRE LOS DIFERENTES SIMULADORES.


\subsection{Microsimulación vs. macrosimulación}

Los dos puntos de vista principales a la hora de abordar un modelo de flujo de tráfico son los micromodelos y los macromodelos. Cuando se trabaja con simulaciones, dependiendo de con qué modelo se trabaje hablamos entonces de microsimulaciónes y macrosimulaciónes, o simlpemente simulaciones \textit{micro} y \textit{macro} respectivamente. Sus características son las siguientes:

\begin{itemize}
	\item \textbf{Micromodelos}. Su objetivo es estudiar desde un punto de vista de granularidad fina (e.g. vehículos o peatones) las micropropiedades del flujo de tráfico (e.g. cambio de carril, aproximaciones a vehículos delanteros o adelantamientos) para evaluar su comportamiento. Tiene dos principales ventajas, la posibilidad de estudiar el tráfico como un todo a partir de sus elementos más simlpes (ofreciendo una representacińo más fiel de este) y la posibilidad de estudiar cada elemento por separado. Sin embargo, la principal desventaja de este tipo de modelos es que cada elemento de la simulación requiere de cómputo y por tanto simulaciones con alto contenido de elementos pueden llegar a ser inviables.
	\item \textbf{Macromodelos}. Este tipo de modelos centran su esfuerzo en estudiar el flujo de tráfico como un todo, explorando sus macropropiedades (e.g. evolución del tráfico, efectos onda, velocidad media o flujo en vías). Su ventaja principal es que a nivel macroscópico permiten estudiar propiedades que a nivel microscópico requeriría una cantidad ingente de recursos. Sin embargo, es imposible obtener información precisa de un elemento en particular del tráfico.
\end{itemize}

Aunque esta es la categorización típica de modelos, en la literatura aparecen otros tipos de modelo con granularidades que pueden considerarse no pertenecientes a ninguno de estos dos conjuntos.

Por ejemplo, los \textbf{sub-micromodelos} (\TODO buscar referencia) donde la granularidad baja hasta el punto que los elementos tómicos del modelo son las propias partes del vehículo o los conductores.

Otro ejemplo son los \textbf{mesomodelos}\cite{munoz2001integrated}\cite{casas2011need} donde se trata de amortiguar los problemas interentes a la complejidad en los micromodelos y a la falta de resolución en los macromodelos (\TODO añadir una figura para el modelo citado en~\cite{munoz2001integrated}, donde se hace uso de ventanas en al vía para analizar el comportamiento micro, dejando el modelo fuera de esa ventana en la granularidad macro).

\subsection{Simulación discreta vs. continua}

La discertización aquí se refiere tanto a tiempo como a espacio.

Si hablamos de tiempo, en general las simulaciones de tiempo continuo trabajan sobre macrosimulación, por lo que es de menor interés para nosotros que los simuladores discretos de tiempo, donde se cuantifica el tiempo de la simulación, generalmente a una frecuencia de 1Hz.

Si hablamos de espacio, nos interesan más los simuladores continuos, ya que los discretos pierden demasiado detalle, y nos interesa más el comportamiento en cada instante t que la colocación relativa aproximada de los vehículos en cada instante t (cuantificado o no)

\subsection{Comparativa de simuladores}

En este apartado se facilita la comparativa realizada para la elección de simulador sobre el que basar los escenarios a plantear en las simulaciones de los modelos de conductor.

Quizá se pudea hablar del \enquote{SMARTEST} project: \url{http://www.its.leeds.ac.uk/projects/smartest/}

Simuladores de pago:

Quadstone paramics (microscopic)
VISSUM (macroscopic)
VISSIM (microscopic)
AIMSUN

Simuladores gratuitos:

Matsim
SUMO (microscopic)
Repast
MAINSIM
Synchro

Ni puta idea:

CUBE
SATURN
PARAMICS
TRANSIMS

En definitiva, el simulador que más se adapta a nuestras necesidades y el que se usará como simulador base en el desarrollo de esta tesis será SUMO\footnote{Para más información acerca de su funcionamiento, los artículos \cite{krajzewicz2002sumo}, \cite{behrisch2011sumo} y \cite{krajzewicz2012recent} ofrecen suficiente información sobre su arquitectura, su funcionamiento y sus últimas actualizaciones, aunque el proyecto ha evolucionado bastante desde la última publicación.} (\textit{\textbf{S}imulation of \textbf{U}rban \textbf{MO}bility}). Funciona como simulación de tiempo discreto variable (siendo el mínimo 1ms.) en un escenario espacial continuo.


\section{Sistemas multiagente}


\section{Inteligencia computacional}

Qué es la inteligencia artificial.
Diferencias entre inteligencia artificial clásica e inteligencia computacional. Diferentes puntos de vista (soft computing, machine learning, ...)
Técnicas de entrenamiento de modelos: supervisado, no supervisado, semisupervisado, refuerzo, ...
Técnias de funcionamiendo: online y offline
Técnicas de la inteligencia computacional usadas en esta tesis (redes neuronales artificiales(perceptrón multicapa, recurrentes y lstm), lógica difusa y computación evolutiva)
¿Qué técnicas se usan actualmente y sobre qué problemas?
Detección de patrones de eficiencia y agresividad de subyacen en los comportamientos de éstosç
Estudio de la efectividad de los sistemas de asistencia para mejorar la eficiencia de conducción
Estudio de los sistemas de asistencia para analizar el comportamiento del conductor.

\section{Recolección de datos}



\section{Modelos de comportamiento}

¿Cómo se define el comportamiento de un conductor?

SUMO usa (al menos así lo indican en el paper del 2002) el modelo Gipps\cite{krajzewicz2002sumo}. No sé si ellos han hecho una extensión del modelo o están referenciando la extensión y ellos sólo la implementan. En el paper del 2012 citan que el modelo car-following que usan por defecto es el desarrollado por Stefan Krauß\cite{jin2016evaluation}, debido a su simplicidad y su velocidad de ejecución.

El modelo ha probado ser válido, pero tiene algunos defectos, por lo que existe un API para implementar otros modelos. En la actualidad están incluidos en el sistema los modelos IDM\cite{treiber2000congested} (\textit{Intelligent Driver Model}), el modelo de tres fases de Kerner\cite{kerner2008testbed} y el modelo de Wiedemann\cite{wiedemann1974simulation}.

La práctica totalidad de modelos para el comportamiento de conductores en tráfico se basan en una estructura jerárquica de control\cite{michon1985critical}. Por lo que veo este paper es de 1985, así que afirmar eso 30 años después es una barrabasada. Hay que buscar algo más para apoyar eso o para indicarlo como una anécdota.

En el paper \cite{al2001framework} describen un framework para la modelización de comportamiento de conductores dentro de simuladores.

En~\cite{tang2014new} los autores proponen un modelo de car following teniendo en cuenta comunicación entre vehículos.

\TODO Hoy he descubierto que hay una cosa que se llama naturalistic traffic study o naturalistic traffic observation que es un estudio naturalístico, es decvir, observar en el mundo real con detalle y extraer datos de ello. A lo mejor hablando así queda como más profesional :D.

En~\cite{sekizawa2007modeling} describen modelos supervisados offline para capturar el comportamiento del conductor basados en auto-regresión a trozos. Más adelante lo extienden en~\cite{terada2010multi}, aunque los datos de entrenamiento son extraídos directamente de simulaciones, no del \enquote{mundo real\textsuperscript{TM}}.

En~\cite{bando2013unsupervised} describen otro modelo no supervisado offline basado en un modelo bayesiano no paramétrico para la clusterización, combinado con un LDA (Latent Dirichlet Allocation, sea lo que sea esto) para la clusterización a más alto nivel. (\TODO ¿este método usa datos reales?)

Estos trabajos (\cite{sekizawa2007modeling}, \cite{terada2010multi} y \cite{bando2013unsupervised}) tienen la desventaja de ser computacionalmente muy costosos y con poca precisión en el caso de variar mucho los escenarios de entrenamiento y de test.

En~\cite{maye2011bayesian} se presenta un modelo online donde se infiere el comportamiento del conductor haciendo uso de una IMU (Intertial Measurement Unit) y una cámara. Primero de la IMU se sacan datos que se separan en fragmentos para luego relacionarlos con las imágenes obtenidas de la cámara. (\TODO ¿Hacia dónde mira la cámara?). Los modelos propuestos en~\cite{johnson2011driving} y~\cite{van2013driver} también se apoyan en el funcionamiento de clasificar las segmentaciones de una IMU, pero con técnicas distintas y sin cámara (\TODO Que digo yo, ¿qué clasifican exactamente? es más, dependiendo de qué clasifican, para qué vale la camarita del~\cite{maye2011bayesian}?). Además hacen uso de señales externas y umbrales de activación para hacer más efectiva la clasificación (\TODO corroborar).

En el artículo~\cite{bender2015unsupervised} se usa también un modelo no supervisado online con una aproximación bayesiana para identificar los puntos de cambio sin depender de parámetros externos (e.g. umbrales o señales). Se basa también en (1) segmentar los datos de conducción y (2) asignar estos datos a clases que se corresponde n con comportamientos de conducción. Tiene la ventaja de ser más eficiente y rubusto que los anteriores.

La idea de estos métodos desde el~\cite{sekizawa2007modeling} hasta el~\cite{bender2015unsupervised} creo que es el de un sistema que traduzca datos en crudo a datos de más alto nivel. Esto es debido a que la cantidad de datos que se pueden generar en un sólo coche (no digamos ya una flota de ellos) es tan grande que para determinados sistemas disponer de información de más alto nivel haría más sencillo su desempeño (por ejemplo, \gls{adas} que funcionasen con datos de \enquote{adelantando} que sus valores de giro, aceleración en una ventana temporal).