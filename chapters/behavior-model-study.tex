\chapter{Estudio de modelos de comportamiento}
\label{ch:behavior-models-study}

...

Creo que además de crear modelos adaptados al conductor de car following y de lane change, estaría bien adaptar los parámetros de los modelos existentes y comparar.

Cosas que se me ocurren. Habría que tener un módulo que se plantea qué acción tomar en función del entorno y qué acciones ocurren alrededor nuestro. Quiero cambiar de carril, está entrando un usuario al carril,

Lo mismo es una chorrada, pero en \cite{Alexiadis2004} el autor se saca de la manga un intervalo temporal para diferenciar las tareas de niveles estratégico, táctico y de control. Estos intervalos podrían ser una variable lingüística y sus particiones variar en función del conductor.

En \cite{Toledo2007} (o \cite{Toledo2007-3}), plantea cosas a mejorar en el estado de la cuestión: (i) más regímenes de conducción y mejor determinación de sus límites y sus transiciones, tanto en modelos de aceleración como en modelos de cambio de carril, mejorando así el realismo de la simulación, (ii) posicionamiento estratégico, es decir, incorporar el ptah planning y cosas por el estilo del comportamiento de planificación al táctico preposicionando el coche en los carriles más convenientes, (iii) más campo de visión, no sólo los delanteros y traseros del carril actual y carril destino, (iv) interdependencia entre modelos y regímenes de modelos, (v) planificación y anticipación,sobre todo esto último en el tema de qué van a hacer los vehículos de alrededor. Si tiramos de algo de aquí, vendría bien mirar el paper porque da referencias a trabajos en estas direcciones.

En un paper que me ha pasado Felipe para revisar (A Novel Method for Predicting Vehicle State in Internet of Vehicle) plantean un modelo basado en árboles de clasificación para identificar el estado del vehículo. El caso es que plantean comportamientos del estilo "evitar jam", "intento de adelantamiento" (que por cierto, se basan únicamente en el tamaño y velocidad del vehículo delantero). Echarle un ojo más adelante.

...

Proponer el modelo en cuatro partes. Car following, free flow, lane exit, decisión de cambio de carril y ejecución de cambio de carril. Todos ellos se ejecutarán con un árbol molón en el que primero se realizarán o no los cambios de carril y luego se decidirá la velocidad del vehículo (yéndose por una de las dos ramas del modelo longitudinal).

\begin{enumerate}
	\item **Car following**. En este caso, el modelo se definirá como controladores difusos ajustados a los conductores con lo que he estado haciendo de representar el controlador difuso como grafo computacional. Las entradas el modelo serán la distancia al vehículo siguiente (hasta un máximo), la diferencia de velocidad con éste y ¿la aceleración actual?. La salida será la aceleración.
	\item **Free flow**. Este caso creo que lo mejor será representarlo como un perceptrón multicapa donde las entradas sean la velocidad máxima de la vía, la velocidad actual del vehículo y la aceleración de esta. Tanto en el caso anterior como en este, la salida deberá ser la aceleración.
	\item **Lane exit**. Este caso es similar al siguiente en entradas, pero la salida es la aceleración a aplicar.
	\item **Decisión de cambio de carril**. Para decidir el cambio de carril debemos saber el estado de los carriles de alrededor, si es posible o no desde esos carriles seguir por la siguente salida y la distancia al comienzo y al final de la siguiente salida. La salida del modelo debe ser "quiero cambiar a la izquierda, a la derecha o no cambiar".
	\item **Ejecución de cambio de carril**. Este es el que hecho y parte de las entradas del entorno y del módulo anterior. Esto lo tengo que describir de forma similar a la del paper publicado.
\end{enumerate}

El orden de los módulos es modelo más grande cuya salida son cuatro valores: la aceleración (como valor de coma flotante) a aplicar, cambio a la izquierda (0 o 1) y cambio a la derecha (0 o 1).

Las entradas pasarán, por un lado a los modelos longitudinales (fx, ff y le) y el resultado será el del menor de los tres y por otro al modelo de cambio de carril.

\section{El vehículo como agente inteligente}

En nuestra tesis el trabajo se basa en la representación del comportamiento de un conductor (\gls{dvu}).

\section{Resultados}
\label{ch:behavior-models-study:results}

Realizar comparativa de modelos existentes contra modelos de parámetros ajustados contra modelos creados.