\chapter{Estudio de modelos de comportamiento}
\label{ch:behavior-models-study}

...

Creo que además de crear modelos adaptados al conductor de car following y de lane change, estaría bien adaptar los parámetros de los modelos existentes y comparar.

Cosas que se me ocurren. Habría que tener un módulo que se plantea qué acción tomar en función del entorno y qué acciones ocurren alrededor nuestro. Quiero cambiar de carril, está entrando un usuario al carril,

Lo mismo es una chorrada, pero en \cite{Alexiadis2004} el autor se saca de la manga un intervalo temporal para diferenciar las tareas de niveles estratégico, táctico y de control. Estos intervalos podrían ser una variable lingüística y sus particiones variar en función del conductor.

En \cite{Toledo2007} (o \cite{Toledo2007-3}), plantea cosas a mejorar en el estado de la cuestión: (i) más regímenes de conducción y mejor determinación de sus límites y sus transiciones, tanto en modelos de aceleración como en modelos de cambio de carril, mejorando así el realismo de la simulación, (ii) posicionamiento estratégico, es decir, incorporar el ptah planning y cosas por el estilo del comportamiento de planificación al táctico preposicionando el coche en los carriles más convenientes, (iii) más campo de visión, no sólo los delanteros y traseros del carril actual y carril destino, (iv) interdependencia entre modelos y regímenes de modelos, (v) planificación y anticipación,sobre todo esto último en el tema de qué van a hacer los vehículos de alrededor. Si tiramos de algo de aquí, vendría bien mirar el paper porque da referencias a trabajos en estas direcciones.

\section{El vehículo como agente inteligente}

En nuestra tesis el trabajo se basa en la representación del comportamiento de un conductor (\gls{dvu}).

\section{Resultados}
\label{ch:behavior-models-study:results}

Realizar comparativa de modelos existentes contra modelos de parámetros ajustados contra modelos creados.