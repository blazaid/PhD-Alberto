\chapter{Agentes inteligentes y sistemas multiagentes}
\label{ch:sota-agents-and-mas}

\section{Concepto de agente, agente inteliegnte y agente racional}

\section{Sistemas multiagente}
\label{ch:sota-agents-and-mas:s:mas}


\section{El vehículo como agente inteligente}

En nuestra tesis el trabajo se basa en la representación del comportamiento de un conductor. Para simplificar el problema asumiremos que los términos \enquote{conductor} y \enquote{vehículo} son equivalentes y que se refieren a la tupla \enquote{(vehículo, conductor)} (Hay un paper que lo denomina DVU como driver-vehicle-unit \cite{Dia2002})

No sé cómo ponerlo, pero esta información puede ser útil aquí. A lo mejor hay que separarlo, no sé. Yo lo pongo aquí porque me ha venido ahora.

\begin{itemize}
	\item \textbf{Percepción}. Característica intrínseca a todo ser vivo para la obtención de información del entorno. Pues para los agentes inteligentes, lo mismo. Puede ser en forma de sensores (e.g. sensor de velocidad, acelerómetro, sensor de distancia, lidar, cámaras, termómetros, gps, \ldots), conocimiento (e.g. controlador difuso, sistema experto, red neuronal, es decir, información procesada que ad valor añadido), datos GIS, comunicación con otros vehículos (e.g. V2V, V2E, \ldots). Un simulador nos ofrece estos sensores casi gratis, pero nuestro problema es que para la generación de modelos necesitamos previamente haber capturado información con dichos sensores y en físico ya es otra cosa.
	\item \textbf{Toma de decisiones}. Un agente racional en un sistema multiagente ha de ser capaz de razonar acerca del mundo, de su propio estado y del estado del resto de agentes.
	\item \textbf{Actuación}. La actuación es la consecuencia natural de las anteriores características. Dado un estado del mundo y un proceso cognitivo surgen acciones a realizar, tras las cuales se llega a un nuevo estado del entorno que provee de información actualizada para seguir actuando.
\end{itemize}
