\cleardoublepage
\begin{fullwidth}
~\vfill
\thispagestyle{empty}
\setlength{\parindent}{0pt}
\setlength{\parskip}{\baselineskip}
\par{
	\textbf{Acerca del código fuente}
	
	La presente tesis lleva consigo numerosas horas de programación, lo que implica varios miles de líneas de código. Sin embargo, esta nota existiria aún con sólo unas pocas decenas. Se ha decidido no proveer de forma impresa el código fuente y en su lugar distribuirlo en formato electrónico, una forma más manejable para su consulta y a la vez respetuosa con el medio ambiente. No obstante sí es posible que existan pequeños fragmentos de código para apoyar explicaciones. En caso de necesitar los fuentes y no ser capaces de conseguirlos, se puede contactar directamente conmigo, el autor, en \href{mailto:alberto.diaz@upm.es}.
	}

\par{
	\textbf{Cómo citar esta tesis}
	
	Si deseas citar esta tesis, lo primero gracias. Me alegro de que te sirva para tu investigación. Si lo deseas, incluye el siguiente código en bibtex:
	
	\textbf{TODO A ver cómo coño meto en el paquete listings caracteres acentuados...}
	}


\end{fullwidth}

