\cleardoublepage
\begin{fullwidth}
~\vfill
\thispagestyle{empty}
\setlength{\parindent}{0pt}
\setlength{\parskip}{\baselineskip}

\par{
	\textbf{Cómo citar esta tesis}
	
	Si deseas citar esta tesis, lo primero gracias. Me alegro de que te sirva para tu investigación. Si lo deseas, incluye el siguiente código en bibtex:
	
	\begin{lstlisting}[escapechar=\%]
	@phdthesis{blazaid20167,
	author = ,
	abstract = {XXX},
	pages = {XXX},
	title = {Modelado de comportamiento de conductores con técnicas de Inteligencia Computacional},
	url = {XXX},
	year = 
	}
	\end{lstlisting}
}

\par{
	\textbf{Acerca del código fuente}
	
	La presente tesis lleva consigo numerosas horas de programación y, por tanto, muchísimas líneas de código. Éste se encuentra en formato electrónico como datos adjuntos a la memoria y no como capítulo o anexo a ésta, una forma más manejable para su consulta y a la vez respetuosa con el medio ambiente. No obstante sí es posible que existan pequeños fragmentos de código para apoyar explicaciones. En caso de necesitar los fuentes y no estar disponibles los datos anexos a la memoria, puedes contactar directamente conmigo en \url{alberto.diaz@upm.es}.
}

\end{fullwidth}

