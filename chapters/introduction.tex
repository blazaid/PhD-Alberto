\chapter{Introducción}
\label{ch:intro}

Es un hecho que la Inteligencia Artificial\footnote{A pesar de existir casi tantas definiciones como autores, podemos definir la inteligencia artificial como el \textit{área que estudia el comportamiento inteligente exhibido en máquinas}.} (IA o AI del inglés \textit{Artificial Intelligence}) como área de conocimiento ha experimentado un creciente interés en los últimos años. Esto no siempre ha sido así, ya que después de un nacimiento muy esperanzador, con mucho optimismo (1956) le siguieron unas épocas de apenas avance (incluso hay una época denominada AI Winter (explicarla mejor)). Sin embargo, en la actualidad es muy difícil encontrar un campo que no se beneficie directamente de la aplicación de técnicas pertenecientes o nacidas en dicho área.

Esto es debido a su característica multidisciplinar ya que, si bien se la define como área perteneciente al campo de la Informática, es transversal a muy diferentes campos, como pueden ser por ejemplo la biología, neurología o la psicología entre otros. [Quizá por aquí se podría hacer una referencia al russel y norvig sobre la separación en los cuatro cuadrantes, aunque quizá mejor en el related work].

Dentro del área de la inteligencia artificial es común diferenciar dos tipos de aproximaciones a la hora de hablar de cómo representar el conocimiento:

\begin{itemize}
	\item \textbf{Inteligencia artificial clásica} o \textit{simbólica}. Postula que el conocimiento como tal es reducible a un conjunto de símbolos junto con operadores para su manipulación. A este tipo de técnicas y soluciones, que basan su funcionamiento en modelos analíticos, se las denominan también \textit{Hard Computing}, y no suelen tolerar factores como la imprecisión y la incertidumbre (e.g. cálculo simbólico o análisis numérico).
	\item \textbf{Inteligencia computacional} o \textit{subsimbólica}. En ésta el conocimiento se alcanza por aproximación. Los esfuerzos se centran en la simulación de los elementos de bajo nivel que subyacen a los comportamientos inteligentes (e.g. redes neuronales artificiales) esperando que de éstos surja la solución de forma espontánea. A este tipo de técnicas se las conoce también como \textit{Soft Computing}, conjunto de soluciones para trabajar sobre información incompleta, imprecisa o con ruido.
\end{itemize}

El límite entre ambos conjuntos no está perfectamente definido, máxime si tenemos en cuenta las diferentes terminologías existentes, las sinergias entre distintas técnicas dentro del área y los diferentes puntos de vista sobre éstas por parte de los autores. Sin embargo, una de las principales diferencias de ambos paradigmas es la forma de solucionar problemas. Mientras que en el primer caso es prácticamente impensable una aproximación diferente a la \textit{top-down} (i.e. se representa la solución, se define el algoritmo y éste lleva a la solución exacta), en el segundo los problemas se resuelven utilizando el paradigma contrario, \textit{bottom-up} (i.e. la solución "aprende" a resolverse dado el problema, sin necesidad de plantear un algoritmo y generando soluciones no necesariamente exactas pero sí lo suficientemente buenas). Revisaremos las diferencias entre conceptos de diferentes autores en el capítulo \ref{ch:state-of-the-art}

Uno de los campos de aplicación es el de los sistemas de transporte inteligentes (ITS, del inglés \textit{Intelligent Transport Systems}). Éstos se definen como un conjunto de aplicaciones orientadas a gestionar el transporte en todos sus aspectos y granularidades (e.g. conducción eficiente, diseño de automóviles, gestión del tráfico o señalización en redes de carreteras) para hacerlos más eficientes y seguros. El interés en tal que en el año 2010 se publicó la directiva 2010/40/UE\cite{parliament2010directive} donde se estableció el marco de implantación de los ITS en la Unión Europea\footnote{En esta directiva, los ITS se definen como \textit{aplicaciones avanzadas que, sin incluir la inteligencia como tal, proporcionan servicios innovadores en relación con los diferentes modos de transporte y la gestión del tráfico y permiten a los distintos usuarios estar mejor informados y hacer un uso más seguro, más coordinado y «más inteligente» de las redes de transporte.}}.

En el caso concreto del comportamiento al volante, es interesante la evaluación de los conductores para conocer su manera de actuar en determinados escenarios, y poder extraer información de estos que nos permiten, por ejemplo, detectar qué factores pueden afectar más o menos a determinados indicadores (e.g. el consumo estimado para una ruta en concreto). Sin embargo, la evaluación en distintos escenarios puede no ser interesante debido a limitaciones existentes, como pueden ser el tiempo, el dinero o la peligrosidad del escenario.

Los simuladores de tráfico\footnote{Concretamente los \textit{micro-simuladores}, donde los conductores y/o vehículos son modelados como agentes independientes. Existe otra categoría dentro de la simulación de tráfico, denominada \textit{macro-simuladores}, que conciben el tráfico como un fluido. Aún así, veremos este concepto más en detalle en el capítulo~\nameref{ch:state-of-the-art}.} son una solución para muchas de estas limitaciones, pero suelen basar su funcionamiento en conductores y vehículos (normalmente concebidos como una única entidad) basandose en modelos de conductor que responden a funciones más o menos complejas, además con pocas o ningunas opciones de personalización. Esto provoca que dichos modelos se adapten poco al modelo de un conductor en concreto.

Esta tesis pretende entrar en el tema de la generación de modelos de conductor para simuladores que respondan al comportamiento de conductores en concreto usando, para ello, técnicas pertenecientes al campo de la Inteligencia Computacional.

Concretamente pretende desarrollar un método para el análisis de la eficiencia de los conductores realizando, para ello, un modelo del perfil de conducción a partir de técnicas de la Inteligencia Computacional y aplicándolo a un entorno multiagente de simulación. Así, una vez configurado el entorno multiagente, se podrá simular el tráfico y estudiar aspectos como el estilo de conducción o el impacto de los sistemas de asistencia.


\section{Motivación}

Los conceptos introducidos al comienzo del capítulo obedecen a una \textit{necesidad} (eufemismo de problema) de la sociedad en la que vivimos, y que afecta tanto a nuestra generación como afectará a las venideras: la eficiencia en la conducción. Dado que es imprescindible saber que existe un problema para arreglarlo, nada mejor que introducir primero algunos hechos conocidos:

\begin{itemize}
	\item En el año 2014, el número de vehículos a nivel mundial asciende a más de $1.200$ millones\cite{oica2014motrate}, con una tendencia ascendente. Reducir en un pequeño porcentaje el consumo durante la conducción evita la emisión de toneladas de gases considerados nocivos para el medio ambiente y el ser humano\footnote{Uno puede argumentar que el parque automovilístico se recicla con nuevos vehículos eléctricos categorizados "de consumo 0". La triste realidad es que estos vehículos consumen la electricidad generada actualmente de una mayoría de centrales de combustibles fósiles y nucleares. Además, mientras que en países desarrollados el crecimiento ha sido en torno al 4-7\%, en países subdesarrollados, donde no existe aun infraestructura para la recarga de vehículos eléctricos, dicho crecimiento ha superado el 120\%.}
	\item Debemos abandonar los combustibles fósiles antes de que éstos nos abandonen a nosotros. Aunque existen diferentes puntos de vista acerca de cuándo se agotarán las reservas de petróleo, lo cierto es que los combustibles fósiles son recursos \textbf{finitos}. Lo más probable es que no se llegue a agotar debido a la ley de la oferta y la demanda, pero hay que recordar que el petróleo se usa como base para la producción de muchos y muy diferentes tipos de productos, como por ejemplo la vaselina, el asfalto o los plásticos.
	\item Independientemente del momento en el que se agoten los recursos, hay que hacer notar que la emisión de gases está correlacionada con el aumento de la temperatura del planeta. De seguir con el ritmo de consumo actual, se teme llegar a un punto de no retorno con consecuencias catastróficas para la vida en el planeta.
	\item Algo más cercano, y aun así íntimamente relacionado. La conducción eficiente afecta directamente a factores correlacionados con el número de accidentes de tráfico. Un factor de sobra conocido es el de la velocidad, factor relacionado no sólo con el número sino con la gravedad de los accidentes\cite{imprialou2016re}.
\end{itemize}

Estos hechos son solo algunos que ponen de manifiesto la necesidad de centrarse en el problema de cómo hacer de la conducción una actividad más eficiente y segura.

La \textbf{conducción eficiente} o \textit{eco-driving} es definida como la aplicación de una serie de reglas de conducción con el objetivo de reducir el consumo de carburante (en el caso de coches de combustión) o de electricidad (en el caso de coches eléctricos).

Ser capaces de identificar o al menos estimar qué conductores son considerados como no eficientes es importante debido a que de esta manera se pueden identificar los hábitos recurrentes en este tipo de perfil y adecuar la formación para eliminar dichos hábitos. Más aún teniendo en cuenta la relación existente entre la peligrosidad y algunas conductas agresivas. Un ejemplo donde la identificación de perfiles no eficientes pueden tener impacto claro económico y social es el de las empresas cuya actividad se basa en el transporte de mercancías o de personas.

Sin embargo, identificar la conducta de un conductor no es fácil, dado que su comportamiento se ve condicionado por numerosos factores como el estado de la ruta, el del tráfico o el estado físico o anímico. Además, la ambigüedad de las situaciones dificulta todavía más la identificación. Por ejemplo, un conductor puede ser clasificado en un momento como agresivo o no eficiente en una situación únicamente porque su comportamiento ha sido condicionado por las malas reacciones por parte de los demás conductores.

El análisis de todos los posibles casos es una tarea prácticamente imposible. Por ello, las simulaciones pueden dar una estimación de los posibles resultados de un estudio en el mundo real. Las simulaciones con sistemas multiagente\footnote{Los sistemas multiagente (SMA o MAS del inglés \textit{Multi-Agent Systems}) son aquellos sistemas compuestos por diversos elementos denominados agentes, los cuales cooperan sobre un entorno para, normalmente, llegar a una solución.} representan a los conductores como agentes permitiendo la evaluación del comportamiento tanto individual como general del sistema en base a sus individuos a través de iteraciones discretas de tiempo. Si dichos agentes son obtenidos a partir de la modelización de conductores a partir de sus datos reales, su comportamiento en la simulación podría ser considerado como fuente de datos para condiciones de tráfico y/o ruta no contempladas en el mundo real. De esta forma, se dispondría de un marco de trabajo para la comparación de diferentes conductores sin necesidad de exponerlos a todos y cada uno de los posibles eventos posibles. También sería posible evaluar sistemas de asistencia evitando los problemas de no comparabilidad de condiciones del entorno entre pruebas.

Es decir, se pretende desarrollar un método para el análisis de la eficiencia de los conductores, realizando para ello un modelo del perfil de conducción a partir de técnicas de Inteligencia Artificial y aplicándolo en un entorno multiagente de donde obtener el resto de parámetros. Así, una vez configurado el entorno multiagente, se podrá simular el tráfico y el comportamiento de los conductores dentro de éste cuando su marcha está condicionada por factores como el tráfico, semáforos, etcétera.

Demostrar que la evaluación de un modelo del conductor en entornos simulados es equivalente a la evaluación de conductores en entornos reales implica que es posible la comparación de dos conductores usando un criterio objetivo, es decir, sin depender del estado del resto de factores a la hora de realizar la prueba de campo. Dicho de otro modo, implicaría que es posible comparar la eficiencia de dos conductores independientemente del estado del tráfico e, incluso, sobre rutas diferentes.

\section{Objetivos}
\label{ch:intro:objectives}

	El objetivo de esta tesis doctoral es la de demostrar la hipótesis~\ref{hyp:hypothesis-1}, quedando dicha demostración dentro de los límites impuestos por los supuestos y erstricciones indicados más adelante.

\begin{hyp}[H\ref{hyp:hypothesis-1}] \label{hyp:hypothesis-1}
	La aplicación de técnicas pertenecientes al campo de la Inteligencia Computacional con datos extraídos de un entorno de micro-simulación permitirá modelar, de manera fiel a la realidad, el comportamiento de los conductores pertenecientes a los grupos más representativos.
\end{hyp}

Por tanto, el objetivo de la tesis es el de simular el comportamiento de conductores en entornos de micro-simulación a partir de su comportamiento en entornos reales usando técnicas de Inteligencia Computacional. Para ello se consideran los siguientes objetivos específicos:

\begin{itemize}
	\item Estudiar y aplicar técnicas de la Inteligencia Computacional (e.g. sub-simbólica) sobre el área de la conducción.
	\item Implementar métodos de generación de modelos personalizados a partir de datos de conductores.
	\item Aplicar modelos de conductores a entornos de simulación multiagente.
	\item Validar los modelos de conductor contra conductores reales.
	\item Estudiar la efectividad de sistemas de asistencia encaminados a mejorar la eficiencia y analizar el comportamiento de conductor.
\end{itemize}

\subsection{Supuestos}

\begin{itemize}
	\item Se supone que el comportamiento de un conductor es el comportamiento en línea y el comportamiento de cambio de carril\footnote{Son conocidos en la literatura como \textit{car-following} y \textit{lane-changing} respectivamente. Entraremos en detalle sobre ambos conceptos en el capítulo~\nameref{ch:state-of-the-art}}.
	\item Los datos de los que extraer el comportamiento se corresponderán con lecturas realizadas durante el día, con buena visibilidad y sin lluvia.
\end{itemize}

\subsection{Restricciones}

\begin{itemize}
	\item La resolución máxima del modelo creado es de 1Hz.
	\item En el caso de los modelos que hacen uso de redes neuronales artificiales, no se pueden exlpicar las razones del comportamiento inferido.
\end{itemize}

\section{Resultados}
\label{ch:intro:results}

\section{Estructura de la tesis}
\label{ch:intro:structure}

La tesis está estructurada de la siguiente manera:

\begin{itemize}
	\item \textbf{\nameref{ch:state-of-the-art}}. Revisión del estado de la cuestión donde se explica en qué punto se encuentra la literatura de los temas en los que se apoya la presente tesis.
\end{itemize}
