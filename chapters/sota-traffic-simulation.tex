\chapter{Simulación de tráfico}
\label{ch:sota-traffic-simulation}

El tráfico es un sistema caótico en el que intervienen un número muy elevado de diferentes variables, muchas de las cuales relacionadas entre si. Debido a esto, obtener modelos exactos de tráfico es una tarea prácticamente imposible y es por ello que la mayoría del trabajo cuyo objetivo es la predicción se realice en base a simuladores.

Los simuladores de tráfico son herramientas de software que, usando diferentes modelos, describen el tráfico como sistema, permitiendo:

\begin{itemize}
	\item Extraer resultados y conclusiones de escenarios de tráfico determinados.
	\item Implementar técnicas de tráfico sin necesidad de alterar el tráfico real.
	\item Reproducir exactamente un escenario.
	\item Introducir modificaciones en puntos determinados (e.g. espaciales o temoprales) de un escenario conocido para evaluar la divergencia en su evolución.
\end{itemize}

Los objetivos en la simulación de tráfico son los de hacer que los modelos se parezcan lo máximo posible a la realidad. En este capítulo vamos a ver cuál es la realidad actual de este tipo de simuladores, cuáles son sus diferentes tipologías y maneras de modelar los diferentes aspectos del tráfico y, posteriormente, realizaremos una evaluación de cuáles son los idóneos para nuestro trabajo.

Limitaremos el estudio no obstante a los simuladores de vehículos, obviando otros tipos de simulación de tráfico que no tienen que ver con esta temática, como por ejemplo los orientados a la evaluación de sistemas de señalización inteligentes (e.g.~\cite{jin2016evaluation}) o a la estimación de emisiones (e.g.~\cite{quaassdorff2016microscale}).

\section{Clasificación de simuladores de tráfico}

Los aspectos simulables y medibles del problema tráfico son muy diversos, dependiendo sobre todo del nivel de complejidad del tráfico\sidenote{Modelar una vía por la que circula un centenar de coches no es lo mismo que modelar una ciudad donde circulan millones.}, de qué queremos medir\sidenote{Evaluar a un conductor en una situación determinada o evaluar la evolución del flujo de tráfico en un cuello de botella causado por un accidente.} y de cómo lo modelamos\sidenote{Un autómata celular se modela de forma diferente a un modelo lineal de vías o carriles.}.

El resto de la sección ofrece una visión de las diferentes categorías en las que se clasifican los simuladores de tráfico.

\subsection{Tipos de simulador en función de la complejidad}

La complejidad en una simulación se refiere al nivel de detalle al que queremos llegar a la hora de modelar nuestra solución. Es evidente que según aumentamos el detalle en la simulación aumenta la cantidad de cálculo. Por ejemplo, si queremos modelar el comportamiento de $10$ billones de canicas cayendo por un tubo es considerablemente más eficiente modelarlas como un fluido con una serie de propiedades características que como una colección de elementos individuales, cada uno con sus propiedades (e.g. masa, aceleración, ...) e interaccionando entre sí.

\begin{figure}
	\centering
	\includegraphics{images/granularities-in-traffic-simulation}
	\caption{Taxonomía clásica de los simuladores en función de la granularidad (complejidad) de la simulación.}
	\label{fig:granularities-in-traffic-simulation}
\end{figure}

En el caso de los simuladores de tráfico es lo mismo. En éstos existe un amplio intervalo de granularidades, desde por ejemplo el flujo de entrada en una autovía hasta el consumo de carburante de un vehículo en ciudad. Lo más común es clasificar los simuladores dentro de dos grandes grupos, los cuales se ilustran en la Figura~\ref{fig:granularities-in-traffic-simulation}:

\begin{itemize}
	\item \textbf{Microsimulación} o simulación de tipo \textbf{micro}. Su objetivo es estudiar, desde un punto de vista de granularidad fina como puede ser vehículos o peatones, las micropropiedades del flujo de tráfico como, por ejemplo, los cambios de carril, las aproximaciones a vehículos delanteros o los adelantamientos, para evaluar su comportamiento. Tiene dos principales ventajas, la posibilidad de estudiar el tráfico como un todo a partir de sus elementos más simples (ofreciendo una representación más fiel de éste) y la posibilidad de estudiar cada elemento por separado. Sin embargo, la principal desventaja de este tipo de modelos es que cada elemento de la simulación requiere de cómputo independiente y por tanto simulaciones con alto contenido de elementos pueden llegar a ser inviables\sidenote{Existen técnicas de computación distribuida que superan ampliamente los límites impuestos por la computación en un único nodo, por ejemplo, el simulador de IBM \textit{Megaffic}. Éste implementa un modelo de granularidad micro donde cada elemento es un agente independiente (i.e. sistema multiagente) usando para ello entornos con cientos de núcleos de proceso que proveen de capacidad suficiente para modelar ciudades enteras como Tokio (ver~\cite{Osogami2012} y~\cite{Suzumura2012}).}.
	\item \textbf{Macrosimulación} o simulación de tipo \textbf{macro}. Este tipo de modelos centran su esfuerzo en estudiar el flujo de tráfico como un todo, explorando sus macropropiedades (e.g. evolución del tráfico, efectos onda, velocidad media o flujo en vías). Su ventaja principal es que a nivel macroscópico permiten estudiar propiedades que a nivel microscópico requeriría una cantidad ingente de recursos. Sin embargo, con este modelo es imposible obtener información precisa de un elemento en particular del tráfico.
\end{itemize}

Aunque esta es la categorización típica de modelos, en la literatura aparecen otros tipos de modelo con granularidades que pueden considerarse no pertenecientes a ninguno de estos dos conjuntos. Este es el caso de los simuladores de tipos \textbf{sub-micro} y los \textbf{meso} (ver figura~\ref{fig:mesoscopic-and-submicroscopic-simulation}). 

\begin{figure}
	\centering
	\includegraphics{images/mesoscopic-and-submicroscopic-simulation}
	\caption{Aproximaciones alternativas de modelos en función de la complejidad. Ejemplo de mesosimulación como ventana de microsimulación dentro de un flujo en un macrosimulador (e.g. \cite{munoz2001integrated}) y ejemplo de submicrosimulación donde se modelan componentes internos del vehículo.}
	\label{fig:mesoscopic-and-submicroscopic-simulation}
\end{figure}

Los \textbf{sub-micromodelos} especifican granularidades por debajo del nivel de \enquote{vehículo} o \enquote{peatón}. Por ejemplo, en (\cite{Minderhoud1999}) trabaja a nivel de funcionamiento del control de crucero inteligente de un vehículo en función del entorno del vehículo.

Por otro lado los \textbf{mesosimuladores} (e.g.~\cite{munoz2001integrated} o~\cite{casas2011need}) nacen para amortiguar los problemas inherentes a la complejidad en los micromodelos y a la falta de resolución en los macromodelos.

Dado que en nuestro discurso trabajaremos en la evaluación de modelos de comportamiento de conductores, nos ceñiremos al uso de simuladores que modelen un nivel de granularidad \textbf{micro}.

\subsection{Tipos de simulador en función del espacio y el tiempo}

Existen otras dos dimensiones que generan dos agrupaciones cada una dependiendo de cómo evolucionan los factores espacio y tiempo a lo largo de la simulación. Sin embargo, el tiop de simulador en funcińo de la complejidad (i.e. \textit{micro} vs. \textit{macro}) determina en gran medida la evolución de estos factores.

En el caso del tiempo, si éste transcurre en forma de intervalos variables pero discretos se habla de \textbf{simulación de tiempo discreto} o de \textit{eventos discretos}. Si por el contrario el tiempo es un factor más de un modelo de ecuaciones, generalmente diferenciales, estamos hablando de una \textbf{simulación de tiempo continuo}. En general, las simulaciones de tiempo continuo trabajan sobre macrosimulación, por lo que es de menor interés para nosotros que los simuladores de tiempo discreto, donde se cuantifica el tiempo de la simulación (e.g. pasos a una frecuencia de $1Hz$ o $10Hz$).

En el caso del espacio, la clasificación es similar. Si la simulación se mueve por un espacio discreto, hablamos de una \textbf{simulación de espacio discreto}, y en caso de que el espacio sea continuo, de \textbf{simulación de espacio continuo}. En este caso, para nuestro estudio consideramos que los discretos pierden demasiado detalle. Dado un instante $t$, independientemente de si el simulador es de tiempo discreto o continuo, nos es más interesante conocer la situación exacta del vehículo y no una situación aproximada en una separación discreta del espacio de al simulación.

\subsection{Modelos en microsimulación}

Dentro de la microsimulación, he visto que hay diferentes aproximaciones. Autómatas celulares, sistemas de partículas, sistemas multiagentes. Buscarmás y describir aquí. El caso es que por lo que veo, de esostres sólo nos interesan los mas. Explicarlos aquí.


\section{Elección de software para las simulaciones}

En este apartado se facilita la comparativa realizada para la elección de simulador sobre el que basar los escenarios a plantear en las simulaciones de los modelos de conductor.

Hoy en día existe una oferta muy amplia de simuladores en el mercado, cada uno implementando uno o varios modelos diferentes y bajo diferentes licencias.

Cada simulador tiene sus ventajas e inconvenientes, y es por ello importante realizar un estudio previo para conocerlos y no llevarse sorpresas una vez se llega a estadios más avanzados del estudio. No se trata de realizar una comparativa en busca del mejor simlador de tráfico del mercado, sino en encontrar el simulador que más se adecúa a los criterios concretos para los propósitos de esta tesis.

\subsection{Entornos de simulación a estudiar}

El siguiente listado muestra la lista de simuladores de tráfico sobre los que se realizarán las comparativas. Por motivos de espacio no se han incluido todos los simuladores encontrados en la listeratura, sino que se han seleccionado únicamente aquellos que (i) aun existen y se pueden adquirir, y (ii) son entornos de microsimulación.

\begin{enumerate}
	\item \textbf{AIMSUN}. Entorno de simulación de granularidad micro, meso y macro desarrollado por la empresa \textit{Transport Simulation Systems}. Url: \url{http://www.aimsun.com/}.
	\item \textbf{TSIS-CORSIM}. Entorno de microsimulación compuesto de dos simuladores para distintos modos de tráfico (NETSIM para entornos urbanos y FRESIM para entornos interurbanos) desarrollado dentro de la Universidad de Florida por el centro \textit{McTrans}. Url: \url{http://mctrans.ce.ufl.edu/featured/tsis/}.
\end{enumerate}

Simuladores de pago:

Quadstone paramics (microscopic)
VISSUM (macroscopic)
VISSIM (microscopic)
ARCHISIM

Simuladores gratuitos:

Matsim
SUMO (microscopic)
Repast
MAINSIM
Synchro

Ni puta idea:

CUBE
SATURN
PARAMICS
TRANSIMS


\subsection{Criterios de selección}

Los criterios se muestran ordenados alfabéticamente:

\begin{enumerate}
	\item \textbf{Activo}. Si el simulador está activaemnte desarrollado o si por el contrario se trata de un proyecto con poca actividad por parte de sus autores. Es interesante hacer uso de un simulador que esté siendo activamente desarrollado porque eso favorece la aparición de parches y mejoras sobre el software.
	\item \textbf{Extensibilidad}. Si el simulador permite extender sus funcionalidades de alguna manera. Aunque se puede considerar que si es Open Software, es posible modificar su comportamiento para adcuarlo a los modelos desarrollados, es mejor que el propio software ofrezca los mecanismos necesarios para la integración sin necesidad de tocar el núcleo.
	\item \textbf{Granularidad}. Si el simulador es de tipo micro, meso o macro. Para nuestras necesidades es necesario un simulador que implemente microsimulación, ya que es el único tipo de granularidad que permite evaluar el comportamiento de un conductor independientemente del resto de la simulación.
	\item \textbf{Licencia}. Especifica con qué tipo de licencia se distribuye el software. Es preferible una licencia de tipo Open Software (\TODO hay que ver si esto está bien dicho o no) ya que de esta manera es posible modificar el software en caso de encontrar algún error o falta de funcionalidad que el fabricante no tenga pensado codificar.
	\item \textbf{Sistema operativo}. Sobre qué sistemas operativos está soportado el entorno de simulación. Es imprescindible que el software se ejecute sobre sistemas operativos GNU/Linux por la configuración de los sistemas sobre los que se trabaja, aunque es interesante también su funcionamiento en entornos tipo OS-X.
	\item \textbf{Tipo de simulación}. Qué modelo interno usa el motor para la simulación (e.g. automatas celurares, sistemas multiagentes, ...).
	particle system simulation).
\end{enumerate}

\subsection{Comparativa}

Al haber tal cantidad de simuladores, la comparativa se ha realizado descartando aquellos simuladores que no cuentan con características necesarias o que son de una tipología no deseada. A continuación se enumera la lista de razones por las que se han descartado simuladores junto con aquellos afectados por la decisión:

\begin{enumerate}
	\item Debe ser (o al menos soportar) un entorno de microsimulación.
	\item Debe ofrecer un entorno de simulación de tráfico general, no sólo casos particulares como congestión o colisiones.
	\item \ldots
\end{enumerate}

\begin{center}
	\footnotesize
	\begin{tabular}{lllll}
		\toprule
		& Aimsun & \acrshort{sumo} & TSIS-CORSIM & \\
		\midrule
		Activo & sí & sí & sí & \na \\
		\addlinespace
		Extensibilidad & \na & sí & no & \na \\
		\addlinespace
		Granularidad & & & & \\
		\quad Micro          & sí     & sí             & \na  & \na \\
		\quad Meso           & sí     & no             & \na  & \na \\
		\quad Macro          & sí     & no             & \na  & \na \\
		\addlinespace
		Licencia & & & & \\
		\quad Propietaria    & sí     & sí             & \na  & \na \\
		\quad Open Software  & no     & no             & \na  & \na \\
		\quad Compatible GPL & no     & no             & \na  & \na \\
		\addlinespace
		Sistema operativo & & & & \\
		\quad GNU/Linux      & sí     & no             & \na  & \na \\
		\quad OS X           & sí     & no             & \na  & \na \\
		\quad Windows        & sí     & sí             & \na  & \na \\
		\addlinespace
		Tipo de simulación   & \na    & \acrshort{mas} & italics & upright, caps \\
		\bottomrule
	\end{tabular}
\end{center}

\subsection{Entorno seleccionado: \acrshort{sumo}}

En definitiva, el simulador que más se adapta a nuestras necesidades y el que se usará como simulador base en el desarrollo de esta tesis será \gls{sumo}\sidenote{Sus principales publicaciones son~\cite{krajzewicz2002sumo}, \cite{behrisch2011sumo} y \cite{krajzewicz2012recent}.}. \gls{sumo} es un entorno de microsimulación de código abierto\sidenote{Licenciado bajo la \gls{gpl}, concretamente la versión $3.0$.} que implementa un modelo discreto en el tiempo y continuo en el espacio.

Además de simulación clásica, \gls{sumo} provee de una interfaz gráfica (se puede ver un pantallazo en la figura~\ref{fig:sumo-simulator}) donde se puede ver el comportamiento de cada vehículo durante la simulación. Es interesante para obtener de un vistazo información acerca del funcionamiento del modelo en concreto a controlar. Otras de las características que el simulador ofrece son las siguientes:

\begin{figure}
	\includegraphics{sumo-simulator}
	\caption{Ejemplo de pantalla del simulador \gls{sumo}. Además de entorno de simulación propiamente dicho, \gls{sumo} provee de una interfaz gráfica que permite una visualización general, de zonas y de elementos en concreto a la vez que permite la variación de configuración de la simulación durante el desarrollo de la misma.}
	\label{fig:sumo-simulator}
\end{figure}

\begin{itemize}
	\item Multimodalidad permitiendo modelar no sólo tráfico de vehículos sino de peatones, bicicletas, trenes e incluso de barcos.
	\item Vehículos de diferentes tipologías, Simulación con y sin colisiones de vehículos.
	\item Diferentes tipos de vehículos y de carreteras, cada una con diferentes carriles y éstas con diferentes subdivisiones de subcarriles (diseño conceptual para permitir las simulaciones )
\end{itemize}

Al estar licenciado bajo la licencia \gls{gpl}, su distribución implica a su vez la distribución de su código fuente. Esto permite la modeificación de su comportamiento y el desarrollo de nuevos modelos integrados dentro del simulador. Sin embargo nosotros no haremos uso de esta característica, sino que usaremos \gls{sumo} como aplicación servidor y el módulo \gls{traci} como aplicación cliente desde donde gestionar todos los aspectos de cada simulación.

\subsection{La interfaz \glsentrylong{traci}}